

\vspace*{-5cm}
\begin{small}
\section*{{\normalsize \begin{center}GhostML : a mini-ML with global references and ghost terms\end{center}}}
%%%%%%%%%%%%%%%%%%%%%%%%%%%%%%%%%%%%%%%%%%%%%%%%%%%%%%%%%%%%%%%%%%%%%%%%%%%%%%%%
\newcommand{\grtitle}[1]{ & & &\textbf{[#1]}    \\}
\newcommand{\grhead}[3]{#1 & ::= & #2 & \text{#3} \\}
\newcommand{\grcase}[2]{& | & #1 & \text{#2} \\}
\newpage
\section*{GhostML syntax}
\label{GhostML syntax }
\begin{figure}[!ht]\vspace*{-1cm}
\begin{displaymath}	 	
  \begin{array}{lll@{\hspace*{2.5cm} }l}
		 \grhead{prog}{typedecl^{*} \quad vardecl^{*} \quad t}{program}
		 \grhead{typedecl}{\texttt{type} ~ id ~ \alpha, ..., \alpha ~ = \tau}{type declaration}
		 \grhead{vardecl}{\texttt{val} ~ id : \texttt{ref} ~ \tau}{global reference declaration}
    \end{array} 
\end{displaymath} \vspace*{-0.5cm}
\caption*{ \textbf{GhostML Programs} \hfill}
\end{figure}

\begin{figure}[!ht]\vspace*{-1cm}
\begin{displaymath}
 \begin{array}{lll@{\hspace*{2.5cm} }l}
  	\grhead{\tau}{\alpha}{type variable}
  	\grcase{\varepsilon ~ (\tau,\dots, \tau)}{datatype constructor}
  	\grcase{\tau \rightarrow \tau}{function type}
  	\grcase{~\texttt{Unit} ~|~\texttt{int} ~|~ \texttt{bool} ~|~ \texttt{Prop} ~|~ ...}{build-in types}
	\grhead{\sigma}{\forall \overline{\alpha} . \tau}{type scheme}
  \end{array} 
\end{displaymath} \vspace*{-0.5cm}
\caption*{ \textbf{GhostML   types and schemes} \hfill}
\end{figure}

\begin{figure}[!h]\vspace*{-0.5cm}
\begin{displaymath}
  \begin{array}{rrl@{\hspace*{4cm} }l}
  \grhead{\varepsilon}{\quad}{data variable}
  \grcase{\texttt{ghost}}{ghost variable} \\
  
  \grhead{v}{x}{variable}
  \grcase{op}{build-in constants and operands \ ( \textit{1, true, +}, $\vee$, ...)}
  \grcase{C(v,\dots, v)}{constructor application}
  \grcase{(\texttt{fun} \ \varepsilon x_{\tau} \rightarrow t)}{function}
  \grcase{ \texttt{rec} \ f \ \varepsilon x_{\tau}   =  t}{recursive function} \\

  \grhead{t}{v}{value}
  \grcase{v ~ v}{application}
  \grcase{\texttt{let}~\varepsilon x_{\sigma} = t \ \texttt{in} \ t}{local binding}
  \grcase{!~x}{global reference access}
  \grcase{x := v}{global reference assignment}
  \grcase{\mathsf{if}~v~\mathsf{then}~t~\mathsf{else}~t }{conditional expression}
  \grcase{\mathsf{match} ~v ~ \mathsf{with} ~ c	}{pattern-matching}
  \end{array} 
\end{displaymath} \vspace*{-0.5cm}
\caption*{ \textbf{GhostML Values \& Terms} \hfill}
\end{figure}

\begin{figure}[!h]
\begin{displaymath}
  \begin{array}{rrl@{\hspace*{4.5cm}}l}
  \grhead{c}{p -> t}{base case}
  \grcase{p -> t  ~|c}{branch case} \\
  \grhead{p}{x}{variable}
  \grcase{C(x,\dots, x)}{constructor}
  \end{array} 
\end{displaymath} \vspace*{-0.5cm}
\caption*{ \textbf{GhostML Patterns \& Cases} \hfill}
\end{figure}
\end{small}
\newpage

\label{GhostML operational semantics}
\section*{GhostML  operational semantics}


\definecolor{gray}{gray}{0.2}
\newcommand{\stepone}[4]{#1{\color{gray}_{|#3}}   ~~\leadsto~~ #2{\color{gray}_{|#4}}}
\newcommand{\sbst}[3]{#1 [#2 \mapsfrom #3] }
\reservestyle{\command}{\texttt}
\command{let[\;let\;], fun[fun\;], rec[rec\;],  ghost[ghost\;],  in[\;in\;],  if[\;if\;],  then[\;then\;],  else[\;else\;],
 unit[\;unit\;], match[\;match\;], with[\;with\;]}
\typicallabel{E-IF-False-True-Again}

	\infax[E-AppFun]
		{\stepone{(\<fun> \varepsilon x_{\tau} \rightarrow t)v}{\sbst{t}{x}{v}}{\mu}{\mu}}
	
	\infax[E-AppRec]
		{\stepone{(\<rec> f \ \varepsilon x_{\tau} \rightarrow t)  v}
		{t [f \mapsfrom (\<rec> f \ \varepsilon x_{\tau} \rightarrow t), x \mapsfrom v]}{\mu}{\mu}}
		
	\infax[E-Let]
		{\stepone{\<let> \varepsilon x_{\sigma} = v_{1} \<in> t}{\sbst{t}{x}{v}}{\mu}{\mu}}	
	
 	\infax[E-IF-True]
  		{\stepone{\<if> \mathtt{true} \<then> t_{1} \<else> t_{2}} {t_{1}}{\mu}{\mu}}
	
	\infax[E-IF-False]
  		{\stepone{\<if> \mathtt{false} \<then> t_{1} \<else> t_{2}} {t_{2}}{\mu}{\mu}}

	\infax[E-Assign]
  		{\stepone{x := v}{unit}{\mu}{\sbst{\mu}{x}{v}}}

	\infrule[E-Deref] 
		{\mu(x) = v}
		{\stepone{x}{v}{\mu}{\mu}} 
  
  	\infrule[E-Let-Context]
   		{\stepone{t_{1}}{t_{1}'}{\mu}{\mu'}}
    	{\stepone{\<let> \varepsilon x_{\sigma} = t_{1} \<in> t_{2}}{\<let> \varepsilon x_{\sigma} = t_{1} \<in>  t_{2}'}{\mu}{\mu'}}
   	
	\infrule[E-Case-Match]
		{ v_{i}~\mathrm{ filters}~  x_{i}~ \mathrm{for}~  i \in 1 ...  n }
		{\stepone{\<match> C(v_{1}, \ldots, v_{n}) \<with>  | C(x_{1}, \ldots, x_{n}) \rightarrow t ~| c }
		{\\ t[x_{i} \mapsfrom v_{i}]^{i \in 1...n}}{\mu}{\mu}}
 
\infrule[E-Case-Fail]
		{ C \neq C' }
		{\stepone{\<match> C(v_{1}, \ldots, v_{n}) \<with>  | C'(x_{1}, \ldots, x_{n}) \rightarrow t ~| c  }
		{ \\ \<match> C(v_{1}, \ldots, v_{n})  \<with> c }{ \mu }{ \mu}}
\newpage
\label{GhostML typing }

\section{GhostML Typing}
\newcommand{\tstep}[3]{#1 \vdash  #2 : #3}

	\infrule[T-Var]
		{\Gamma(x) = \forall \overline{\alpha}.\tau}
		{\tstep{\Gamma}{\varepsilon x}{\sigma[\overline{\alpha}} \mapsfrom \overline{\tau}]}

    \infrule[T-op]
		{\Gamma(\texttt{op}) = \forall \overline{\alpha}.\tau}
		{\tstep{\Gamma}{\texttt{op}}{\sigma[\overline{\alpha}} \mapsfrom \overline{\tau}]}
		
		
Let us illustrate it with a small example (written in why3 language) 
 of a function that runs though an integer list  and returns the sum of its elements.   
	\begin{whycode}   
var s : ref int
let rec sumlist l =
    match l with
      | Nil -> ()
      | Cons x xs -> s := x + !s; sumlist xs 
    end
    \end{whycode}
Here the specification consist of saying that if before each recursive call of \texttt{sumlist} 
variable \texttt{s} contained the sum of the already seen elements of the list \texttt{l},
then at the end of each call it must contain the sum of 

 \begin{whycode}
let rec iter_0 l (ghost before: list int) =
    variant  { l }
    requires { !s = sum before }
    ensures  { !s = sum ((reverse l) ++ before) }
    match l with
      | Nil -> ()
      | Cons x xs -> s := x + !s; iter_0 xs (Cons x before)
    end

  \end{whycode}
 