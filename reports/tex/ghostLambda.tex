\section{GhostLambda: Lambda-Calculus with Ghost Code}
%cccccccccccccccccccccccccccccccccccccccccccccccccccccccccccccccccccccccccccccc
%Annotating a program with logical formulae predicting statically it dynamic behaviour is a 
%keystone of the deductive software verification approach.
%
%Typically, these logical formulae are assertions about program assigned variables,
% loop invariants, recursive function variants, etc. 
% 
% However,  it is often useful to annotate programs with some data that appear in program 
%  that appear in assertions but do not affect the program dynamic behaviour in any way. 
%   
% When a correct-by-construction executable code is extracted
% from the specified program, this , called \textit{ghost code} disappear  together with
% specification. 
 
In this section we describe a language where one can annotate programs with such \textit{ghost code}.
We start by formalizing \textit{ghost}$\lambda$\textit{-calculus}, a tiny language of simply typed $\lambda$-calculus enriched with ghost variables and ghost expressions.
We then define ghost code \textit{erasure}, which transforms a well-typed ghostLambda term  to a term of standard  $\lambda$-calculus.
Finally we state and proof a few basic preservation properties of such translation.

\subsection{g$\lambda$-calculus syntax and semantics}
The syntax and small-step operational semantics of \glam is summarized below.\vspace{-0.5cm}
\begin{figure}[!h]
  	 \begin{flushright}  	 \textbf{Syntax}  	 \end{flushright}
\begin{displaymath}
 \begin{array}{lll@{\hspace*{2.5cm} }l}
	\grtitle{\texttt{t}}{\textsc{terms:}}
  	\grcase{()}{unit}
  	\grcase{\glvar}{variable}
  	\grcase{\lambda \glvar.\texttt{t}}{abstraction}
  	\grcase{\texttt{t} \texttt{t}}{application}
  	\grcase{\texttt{ghost t}}{ghost term} 
  	\grtitle{\texttt{v}}{\textsc{values:}}
  	\grcase{()}{unit}
  	\grcase{\lambda \glvar.\texttt{t}}{abstraction}
  	\grtitle{\mathfrak{B}}{\textsc{ghost status :}}
  	\grcase{\top ~~(*true*)}{ghost code} 
   	\grcase{\bot~~(*false*)}{physical code} 
  	\grtitle{\tau}{\textsc{types:}}
  	\grcase{\texttt{unit}}{unit type}
  	\grcase{\tau^{\mathfrak{B}} \rightarrow \tau}{function type}
  \end{array} 
\end{displaymath} 

  	 \begin{flushright}  	 \textbf{Evaluation}  	 \end{flushright}
  	\infax[E-AppFun]    	{\stepone{(\lambda x^{b}_{\tau}.\texttt{t})v}{\sbst{\texttt{t}}{x^{b}_{\tau}}{v}}}
	\infax[E-DeGhost]	 	{\stepone{\texttt{ghost t}}{\texttt{t}}}
%	\infrule[E-Context]{{\texttt{t}} \rightarrow {\texttt{t}'}}{{\texttt{E[t]}} \rightarrow {\texttt{E[t']}}}
%	\begin{center}
% 	 where ~ \texttt{E}   $::= \square$  | \texttt{E}~\texttt{t} |	 \texttt{v} \texttt{E}
% 	 \end{center}	 
	\infrule[T-AppLeft]{{\texttt{t}_{1}} \rightarrow {\texttt{t}'_{1}}}{{\texttt{t}_{1} \texttt{t}_{2} } \rightarrow {\texttt{t}'_{1} \texttt{t}_{2}}}
	\infrule[T-AppLeft]{{\texttt{t}_{2}} \rightarrow {\texttt{t}'_{2}}}{{\texttt{v}_{1} \texttt{t}_{2} } \rightarrow {\texttt{v}_{1} \texttt{t}'_{2}}}	
\caption{ \textbf{\glam syntax and semantics} \hfill}
\end{figure}

\subsubsection*{Free variables, scope and equivalence of terms} \textit{Admit}.

\subsection{Typing Relation}


%	
% 	 	  \hrule
%  	 \begin{flushright}  	 \textbf{Typing}  	 \end{flushright}
	\infrule[T-Unit]	 	{}{\tystepone{()}{unit}{\bot}} 
	
	\infrule[T-Var]	 	{}{\tystepone{x^{\mathfrak{B}}_{\tau}}{\tau}{\mathfrak{B}}}
	
	\infrule[T-Ghost]	 {\tystepone{\texttt{t}}{\tau}{\mathfrak{B}}}{\tystepone{\texttt{ghost t}}{\tau}{\top}}
	
	\infrule[T-Abs]	 	 {\tystepone{\texttt{t}}{\tau_{2}}{\mathfrak{B}_{2}}}
										{\tystepone{\lambda x_{\tau_{1}}^{\mathfrak{B}_{1}}.\texttt{t}}{\tau_{1}^{\mathfrak{B}_{1}} \rightarrow \tau_{2}}{\mathfrak{B}_{2}}}
										
	\infrule[T-App]	 
	{\tystepone{\texttt{t}_{1}}{\tau_{2}^{\mathfrak{B}_{2}} \rightarrow \tau_{1}}{\mathfrak{B}_{1}} 
	\qquad  {\tystepone{\texttt{t}_{2}}{\tau_{2}}{\mathfrak{B'}_{2}}}
	\qquad \mathfrak{B}_{2} \Rightarrow \mathfrak{B'}_{2}}
	{\tystepone{\texttt{t}_{1}~ \texttt{t}_{2} }{\tau_{1}}{\mathfrak{B}_{1} \vee (\neg \mathfrak{B}_{2} \wedge \mathfrak{B'}_{2} ) }}
										


\subsubsection{Properties of typing}
\begin{lemma}[Inversion of typing relation]~\\
\vspace{-0.5cm}
\begin{enumerate}
	\item 
	if $\tystepone{\texttt{ghost t}}{\tau_{1}}{\mathfrak{B_{1}}}$ 
	then	$ \mathfrak{B_{1}} = \top $ 
	and $\tystepone{\texttt{t}}{\tau_{1}}{\mathfrak{B_{2}}} $.
%	with $\mathfrak{B_{2}} = \{\bot, \top\}$.
	
 	\item
 	if $\tystepone
 			{\lambda x_{\tau_{2}}^{\mathfrak{B}_{2}}.\texttt{t}}
 			{\tau_{1}}
 			{\mathfrak{B_{1}}}$
 	then $\tau_{1} = \tau_{2}^{\mathfrak{B}_{2}} \rightarrow \tau_{11}$
 	for some $\tau_{11}$ with \mbox{$\tystepone{\texttt{t}}{\tau_{11}}{\mathfrak{B_{1}}} $}.
  
 \item 
 	If ${\tystepone{\texttt{t}_{1}~ \texttt{t}_{2} }{\tau_{1}}{\mathfrak{B}_{1}}}$
	then there exist $\tau_{11}$, $\tau_{2}$, $\mathfrak{B_{2}}$ and 	
	$\mathfrak{B'_{2}}$
	such that \\ 
	$\tystepone
		{\texttt{t}_{1}}
		{\tau_{2}^{\mathfrak{B}_{2}} \rightarrow \tau_{11}}
		{\mathfrak{B_{1}}}$
		and 
	${\tystepone{\texttt{t}_{2}}{\tau_{2}}{\mathfrak{B'}_{2}}}$ with \\
	 \quad
	\mbox{$\vDash \mathfrak{B}_{1} \vee (\neg \mathfrak{B}_{2} \wedge \mathfrak{B'}_{2} ) \wedge (\mathfrak{B}_{2} \Rightarrow \mathfrak{B'}_{2})$}. 
	
	In particular, if 
 		${\tystepone{\texttt{t}_{1}~ \texttt{t}_{2} }{\tau_{1}}{\bot}}$
then $\vDash \mathfrak{B_{2}} \Leftrightarrow \mathfrak{B'_{2}} $.
\end{enumerate}
\end{lemma} 
\begin{proof}
Straightforward from definition of the typing relation.
\end{proof}

\begin{lemma}[Progress] Admit.
\end{lemma}
\begin{lemma}[Preservation] Admit.
\end{lemma}
\begin{theorem}[Soundness] Admit.
\end{theorem}
%--------------------------------------------------------------------------%
\subsection{Ghost Code Erasure}
\newcommand{\e}{\mathcal{E}}
\theoremstyle{remark}
\newtheorem{dfn}[theorem]{Definition}

\qquad Once we formally defined the simply typed lambda-calculus enriched with ghost expressions, our goal is to show that terms which are not ghost 
themselves have the same computational behaviour as their translation to
lambda-calculus, which preserves the structure of terms except for ghost sub-expressions. 

Therefore we need to define at first a type-erasure and a term-erasure translations from g$\lambda$-calculus to standard simply typed  
$\lambda$-calculus. 

%--------------------------------------------------------------------------%
\subsubsection{Target language} 
Standard Simply typed lambda calculus. ...
\subsubsection{Ghost Erasure}


\begin{dfn}[Type-Erasure] 
\label{type-erasure}
We define type-erasure by induction on the structure of types : \vspace{0.2cm} 

\noindent$ \e_{\top}(\tau) = \texttt{unit} $ 

\noindent$ \e_{\bot}(\texttt{unit}) = \texttt{unit}$ 

\noindent$ \e_{\bot}(\tau_{2}^{\mathfrak{B_{2}}} \rightarrow \tau_{1})  
= \e_{\mathfrak{B_{2}}}(\tau_{2}) \rightarrow \e_{\bot}(\tau_{1}) $. \\
\end{dfn} 

\begin{dfn}[Term-Erasure] 
\label{term-erasure}
Let \texttt{t} be a term such that 
~~ $\tystepone{\texttt{t}}{\tau}{\mathfrak{B}}$ ~ holds. 
We define term-erasure function by induction on 
the structure of \texttt{t}\\[0.1cm]
$ \e_{\top}(\texttt{t}_{1}) = \texttt{()}$ where $ \tystepone{\texttt{t}_{1}}{\tau_{1}}{\top}$.\\[0.05cm]
$\e_{\bot}(\texttt{()}) = \texttt{()}$\\[0.05cm]
$\e_{\bot}(x_{\tau}^{\bot}) = x_{\e_{\bot}(\tau)}$\\[0.05cm]  
$\e_{\bot}(\lambda x_{\tau_{2}}^{\mathfrak{B_{2}}}.t) 
= \lambda x_{\e_{\mathfrak{B_{2}}}(\tau_{2})}. \e_{\bot}(t)$\\[0.05cm]  
$\e_{\bot}(\texttt{t}_{1} ~ \texttt{t}_{2}) 
= \e_{\bot}(\texttt{t}_{1})  ~ \e_{\mathfrak{B_{2}}}(\texttt{t}_{2})$
\quad where $ \tystepone{\texttt{t}_{2}}{\tau_{2}}{\mathfrak{B_{2}}}$.
\end{dfn}
As it can be seen, the erasure function is a morphism that preserve
the structure of operational (not ghost) terms and their types ($\sim\e_{\bot}(\star)$), and sends ghost expressions and types to \texttt{()} and \texttt{unit} respectively ($\sim\e_{\top}(\star)$).




%--------------------------------------------------------------------------%
\subsection{Properties of ghost erasure}

\quad Now that we defined the erasure-translation of $g\lambda$-calculus to $\lambda$-calculus, our concern is to show that evaluation result of well-typed operational terms as well as their typing  are preserved under erasure. 
First off we need to state and prove a few basic lemmas. 





%cccccccccccccccccccccccccccccccccccccccccccccccccccccccccccccccccccccccccccccc
%%%%%%%%%%%%%%%%%%%%%%%%%%%%%%%%%%%%%%%%%%%%%%%%%%%%%%%%%%%%%%%%%%%%%%%%%%%%%%%
\subsubsection{Evaluation Preservation}
\begin{lemma}[Substitution under erasure] ~\\
If $\tystepone
		{\texttt{t}_{1}}
		{\tau_{1}}
		{\bot}$  
and 
	\mbox{
		$\tystepone
			{\texttt{v}_{2}}
			{\tau_{2}}
			{\mathfrak{B_{2}}}$}	
hold, \\ 
then
$\e_{\bot}
	(\sbst
		{\texttt{t}_{1}}
		{x^{\mathfrak{B_{2}}}_{\tau_{2}}}
		{\texttt{v}_{2}})
= \sbst
		{\e_{\bot}(\texttt{t}_{1})}
		{x_{\e_{\mathfrak{B_{2}}}(\tau_{2})}}
		{\e_{\mathfrak{B_{2}}}(\texttt{v}_{2})}$
\label{Substitution under erasure}
\end{lemma}

\begin{proof}
By induction on the structure of $\texttt{t}_{1}$.  \\

\noindent\textit{Case} 
$\texttt{t}_{1} = x^{\mathfrak{B_{2}}}_{\tau_{2}}$:\\
In that case,  we can deduce that $\mathfrak{B_{2}} = \bot$. Therefore:

\begin{center}
 $\e_{\bot}
	(\sbst
		{x^{\bot}_{\tau_{2}}}
		{x^{\bot}_{\tau_{2}}}
		{\texttt{v}_{2}}) 
	= \e_{\bot}(\texttt{v}_{2}) $ 
	$= \sbst
			{x_{\e_{\bot}(\tau_{2})}}
			{x_{\e_{\bot}(\tau_{2})}}
			{\e_{\bot}(\texttt{v}_{2})}$ \\[0.08cm]
	$= \sbst
			{\e_{\bot}(x^{\bot}_{\tau_{2}})}
			{x_{\e_{\bot}(\tau_{2})}}
			{\e_{\bot}(\texttt{v}_{2})}$
\end{center}

\noindent\textit{Case} $
\texttt{t}_{1} = y^{\bot}_{\tau'_{2}} \neq x^{\mathfrak{B_{2}}}_{\tau_{2}}$:
\begin{center}
 $\e_{\bot}
	(\sbst
		{y^{\bot}_{\tau'_{2}}}
		{x^{\mathfrak{B_{2}}}_{\tau_{2}}}
		{\texttt{v}_{2}}) 
	= \e_{\bot}(y^{\bot}_{\tau'_{2}}) 
	= y_{\e_{\bot}(\tau'_{2})} $ \\[0.08cm]
$ = \sbst
			{y_{\e_{\bot}(\tau'_{2})}}
			{x_{\e_{\mathfrak{B_{2}}}(\tau_{2})}}
			{\e_{\mathfrak{B_{2}}}(\texttt{v}_{2})}$ 
$ = \sbst
			{\e_{\bot}(y^{\bot}_{\tau'_{2}})}
			{x_{\e_{\mathfrak{B_{2}}(\tau_{2})}}}
			{\e_{\mathfrak{B_{2}}}(\texttt{v}_{2})}$
\end{center}

\noindent\textit{Case} 
$\texttt{t}_{1} 
= \lambda y^{\mathfrak{B'_{2}}}_{\tau'_{2}}. t_{11} \quad \text{ with }   
y^{\mathfrak{B'_{2}}}_{\tau'_{2}} \not\in \texttt{FV}(\texttt{v}_{2}) 
\text{ and } \neq x^{\mathfrak{B_{2}}}_{\tau_{2}} : $
\begin{center}
 $\e_{\bot}
	(\sbst
		{(\lambda y^{\mathfrak{B'_{2}}}_{\tau'_{2}}. t_{11})}
		{x^{\mathfrak{B_{2}}}_{\tau_{2}}}
		{\texttt{v}_{2}})
	= \e_{\bot}[\lambda y^{\mathfrak{B'_{2}}}_{\tau'_{2}}.
	(\sbst
		{ t_{11}}
		{x^{\mathfrak{B_{2}}}_{\tau_{2}}}
		{\texttt{v}_{2}})] $\\[0.08cm]
$	= \lambda y_{\e_{\mathfrak{B'_{2}}}(\tau'_{2})}.
		\e_{\bot}(
		\sbst
			{\texttt{t}_{11}}
			{x^{\mathfrak{B_{2}}}_{\tau_{2}}}
			{\texttt{v}_{2}})] $\\[0.08cm]
$ \stackrel{Ind.Hyp.}{=} 
	\lambda y_{\e_{\mathfrak{B'_{2}}}(\tau'_{2})}.		
	\sbst
			{\e_{\bot}(\texttt{t}_{11})}
			{x_{\e_{\mathfrak{B_{2}}}(\tau_{2})}}
			{\e_{\mathfrak{B_{2}}}(\texttt{v}_{2})}$ \\[0.08cm]
$	= \sbst
			{(\lambda y_{\e_{\mathfrak{B'_{2}}}(\tau'_{2})}.
				\e_{\bot}(\texttt{t}_{11}))}
			{x_{\e_{\mathfrak{B_{2}}}(\tau_{2})}}
			{\e_{\mathfrak{B_{2}}}(\texttt{v}_{2})}$ \\[0.08cm]
$ = 
	\sbst
			{\e_{\bot}(\lambda y^{\mathfrak{B'_{2}}}_{\tau'_{2}}. t_{11}) }
			{x_{\e_{\mathfrak{B_{2}}}(\tau_{2})}}
			{\e_{\mathfrak{B_{2}}}(\texttt{v}_{2})}$
\end{center}

\noindent\textit{Case} 
$\texttt{t}_{1} = \texttt{t}_{11}\texttt{t}_{12} $
\begin{center}
 $\e_{\bot}
	(\sbst
		{\texttt{t}_{11}\texttt{t}_{12}}
		{x^{\mathfrak{B_{2}}}_{\tau_{2}}}
		{\texttt{v}_{2}}) 	$ 
$	= \e_{\bot}
	(\sbst
		{\texttt{t}_{11}}
		{x^{\mathfrak{B_{2}}}_{\tau_{2}}}
		{\texttt{v}_{2}} ~
	\sbst
		{\texttt{t}_{12}}
		{x^{\mathfrak{B_{2}}}_{\tau_{2}}}
		{\texttt{v}_{2}}) 	  $ \\[0.08cm]
$ = \e_{\bot}
	(\sbst
		{\texttt{t}_{11}}
		{x^{\mathfrak{B_{2}}}_{\tau_{2}}}
		{\texttt{v}_{2}}) ~
	\e_{\bot}
	(\sbst
		{\texttt{t}_{12}}
		{x^{\mathfrak{B_{2}}}_{\tau_{2}}}
		{\texttt{v}_{2}}) 	  $ \\[0.08cm]
$ \stackrel{Ind.Hyp.}{=} 
		(\sbst
			{\e_{\bot}({\texttt{t}_{11}})}
			{x_{\e_{\mathfrak{B_{2}}}(\tau_{2})}}
			{\e_{\mathfrak{B_{2}}}(\texttt{v}_{2})})	
		(\sbst
			{\e_{\bot}({\texttt{t}_{12}})}
			{x_{\e_{\mathfrak{B_{2}}}(\tau_{2})}}
			{\e_{\mathfrak{B_{2}}}(\texttt{v}_{2})})	$ \\[0.08cm]
$ = 
	\sbst
		{\e_{\bot}(\texttt{t}_{11} \texttt{t}_{12})}
		{x_{\e_{\mathfrak{B_{2}}}(\tau_{2})}}
		{\e_{\mathfrak{B_{2}}}(\texttt{v}_{2})})	$\\[0.08cm]
\end{center}
 \end{proof}



\theoremstyle{remark}

\begin{lemma}[One-step evaluation under erasure]
For any closed g$\lambda$-term \texttt{t} such that
$\tystepone{\texttt{t}}{\tau}{\bot}$ holds, 
if	$\texttt{t} \rightarrow_{g\lambda} \texttt{t}'$ 
for some term \texttt{t}', 
then either 
\text{$\e_{\bot}(\texttt{t}) \rightarrow_{\lambda} \e_{\bot}(t')$} 
or 		$\e_{\bot}(\texttt{t}) = \e_{\bot}(t')$. 
\end{lemma}

\begin{proof} 
By induction on the evaluation relation of 
$\texttt{t} \rightarrow_{g\lambda} \texttt{t}'$. \\

\noindent\textit{Case} \textsc{E-AppAbs}: \qquad
$ ~\texttt{t} 
= (\lambda x^{\mathfrak{B_{2}}}_{\tau_{2}}.
		\texttt{t}_{1})\texttt{v}_{1}$ ~with~
$ (\lambda x^{\mathfrak{B_{2}}}_{\tau_{2}}.
		\texttt{t}_{1})\texttt{v}_{1} 
	\stackrel{\epsilon\quad}{\rightarrow_{g\lambda}}  
	\sbst
		{\texttt{t}_{1}}
		{x^{\mathfrak{B_{2}}}_{\tau_{2}}}
		{\texttt{v}_{1}}$\\
\phantom{\noindent\textit{Case} \textsc{E-AppAbs}: \qquad }		
$ \tystepone
		{(\lambda x^{\mathfrak{B_{2}}}_{\tau_{2}}.\texttt{t}_{1})\texttt{v}_{1}}	
		{\tau_{1}}
		{\mathfrak{B_{1}}}, \quad 
 	\tystepone
 		{\texttt{v}_{1}}
 		{\tau_{2}}
 		{\mathfrak{B'_{2}}}, $ \\
\phantom{\noindent\textit{Case} \textsc{E-AppAbs}: \qquad }		
$ \mathfrak{B_{1}} = \bot,  \quad 
	\vDash\mathfrak{B_{2}} \Leftrightarrow \mathfrak{B'_{2}} $ 

\begin{center}
	\begin{tabular}{lll}
   & $ \e_{\bot}
  [(\lambda x^{\mathfrak{B_{2}}}_{\tau_{2}}.
  		\texttt{t}_{1})\texttt{v}_{1}]$ &\\
   & $ = \lambda x_{\e_{\mathfrak{B_{2}}}(\tau_{2})}. 
   \e_{\bot}(\texttt{t}_{1}))\e_{\mathfrak{B'_{2}}}(\texttt{v}_{1})$ 
   & (as $ \mathfrak{B_{1}} = \bot $)  \\
&  $ \stackrel{\epsilon~~}{\rightarrow_{\lambda}} 
	   \sbst
   		{\e_{\bot}(\texttt{t}_{1})}
   		{x_{\e_{\mathfrak{B_{2}}}(\tau_{2})}}
   		{\e_{\mathfrak{B'_{2}}}(\texttt{v}_{1})}$ & (head red.) \\
& $ = \e_{\bot}(\sbst{\texttt{t}_{1}}{x^{\mathfrak{B_{2}}}_{\tau_{2}}}{\texttt{v}_{1}}) $ & (by Substitution under erasure lemma)
	\end{tabular}
\end{center}

\noindent\textit{Case} \textsc{E-DeGhost}:  \qquad \\ Trivially verified,
as for any instance of $\tystepone{ghost \texttt{t}_{1}}{\tau_{1}}{\mathfrak{B_{1}}}$, $\mathfrak{B_{1}} = \top $.\\

\noindent\textit{Case} \textsc{E-AppLeft}: \qquad 
 $\texttt{t} = \texttt{t}_{1} \texttt{t}_{2},\quad
 \texttt{t}^{'} = \texttt{t}_{1} \texttt{t}^{'}_{2}, ~ \text{ with }
 \texttt{t}_{1} \rightarrow_{g\lambda} \texttt{t}^{'}_{1}$ \\
\phantom{\noindent\textit{Case} \textsc{E-AppLeft}: \qquad}		
$\tystepone{\texttt{t}_{1}}{\tau_{2}^{\mathfrak{B_{2}}} \rightarrow \tau_{1}}{\mathfrak{B_{1}}}$,  ~
$\tystepone{\texttt{t}_{2}}{\tau_{2}}{\mathfrak{B'_{2}}}, $ \\ 
\phantom{\noindent\textit{Case} \textsc{E-AppLeft}: \qquad}		
$ \mathfrak{B_{1}} = \bot,  \quad 
\vDash\mathfrak{B_{2}} \Leftrightarrow \mathfrak{B'_{2}} $ \\

As $\mathfrak{B_{1}} = \bot$, we can apply induction hypothesis on
$\texttt{t}_{1}$ which gives  
$\e_{\bot}(\texttt{t}_{1}) \rightarrow_{\lambda} 
\e_{\bot}(\texttt{t}_{1}^{'})$.
Then, applying \textsc{E-AppRight} rule, we obtain:
$$
\e_{\bot}(\texttt{t}) 
= \e_{\bot}(\texttt{t}_{1})\e_{\bot}(\texttt{t}_{2})
\rightarrow_{\lambda} 
\e_{\bot}(\texttt{t}^{'}_{1})\e_{\bot}(\texttt{t}_{2})
= \e_{\bot}(\texttt{t}^{'}).$$




\noindent\textit{Case} \textsc{E-AppRight}: \qquad 
 $\texttt{t} = \texttt{v}_{1} \texttt{t}_{2},\quad
 \texttt{t}^{'} = \texttt{v}_{1} \texttt{t}^{'}_{2}, ~ \text{ with }
 \texttt{t}_{2} \rightarrow_{g\lambda} \texttt{t}^{'}_{2}$ \\
\phantom{\noindent\textit{Case} \textsc{E-AppRight}: \qquad}		
$\tystepone{\texttt{v}_{1}}{\tau_{2}^{\mathfrak{B_{2}}} \rightarrow \tau_{1}}{\mathfrak{B_{1}}}$,  ~
$\tystepone{\texttt{t}_{2}}{\tau_{2}}{\mathfrak{B'_{2}}}, $ \\ 
\phantom{\noindent\textit{Case} \textsc{E-AppRight}: \qquad}		
$ \mathfrak{B_{1}} = \bot,  \quad 
\vDash\mathfrak{B_{2}} \Leftrightarrow \mathfrak{B'_{2}} $ \\
If $\mathfrak{B_{2}} = \mathfrak{B'_{2}} = \top $, then
$$\e_{\bot}(\texttt{t}) 
= \e_{\bot}(\texttt{v}_{1})\e_{\top}(\texttt{t}_{2})
= (\e_{\bot}(\texttt{v}_{1}))\texttt{()}
= \e_{\bot}(\texttt{v}_{1})\e_{\top}(\texttt{t}^{'}_{2})
= \e_{\bot}(\texttt{t}^{'}). $$
Otherwise, $\mathfrak{B_{2}} = \mathfrak{B'_{2}} = \bot $.
By induction hypothesis, 
$\e_{\bot}(\texttt{t}_{2}) \rightarrow_{\lambda} 
\e_{\bot}(\texttt{t}_{2}^{'})$.
Then, applying \textsc{E-AppRight} rule, we obtain:
$$
\e_{\bot}(\texttt{t}) 
= \e_{\bot}(\texttt{v}_{1})\e_{\bot}(\texttt{t}_{2})
\rightarrow_{\lambda} 
\e_{\bot}(\texttt{v}_{1})\e_{\bot}(\texttt{t}^{'}_{2})
= \e_{\bot}(\texttt{t}^{'}).$$




%
%
%By canonical forms lemma, $\texttt{v}_{2} = 
%\lambda x^{\mathfrak{B_{2}}}_{\tau_{2}}. \texttt{t}_{2} $.    
 
% $\texttt{t} = \texttt{E t}_{1},\quad
% \texttt{t'} = \texttt{E t'}_{1}, \quad
% \texttt{t}_{1} \rightarrow_{g\lambda} \texttt{t'}_{1}$  

\end{proof}

Now we can prove the main theorem.
\begin{theorem}[Value preservation under erasure]
	For any closed g$\lambda$-term \texttt{t} such that
	$\tystepone{\texttt{t}}{\tau}{\bot}$ holds, 
	if	$\texttt{t} \rightarrow^{\ast}_{g\lambda} \texttt{v}$ for some value \texttt{v}, then
	 \text{$\e(\texttt{t}) \rightarrow^\ast_{\lambda} \e(v)$}. 
\end{theorem}
\begin{proof} By induction on the length of the evaluation of 
$\texttt{t} \rightarrow^{\ast}_{g\lambda} \texttt{v}$. 


We already have proved the base case : indeed, if $\texttt{t} \rightarrow_{g\lambda} \texttt{v}$ then by the one-step evaluation lemma, 
$\e_{\bot}(\texttt{t}) \rightarrow^{0 | 1}_{\lambda} 
\e_{\bot}(\texttt{v})$. 

Now, assume that $\texttt{t} \rightarrow^{1}_{g\lambda} 
\texttt{t}' \rightarrow^{n}_{g\lambda}  \texttt{v}$
for some arbitrary $n \in \mathbb{N}$.
By the progress of typing, 
$ \tystepone{\texttt{t}'}{\tau}{\bot} $, so we can apply induction hypothesis on $\texttt{t}'$ which gives~
$\e(\texttt{t}') \rightarrow^\ast_{\lambda} \e(\texttt{v})$.
By the one-step evaluation lemma again, we have 
$\e_{\bot}(\texttt{t}) \rightarrow^{0 | 1}_{\lambda} 
\e_{\bot}(\texttt{t}')$. 
That is, 
$\e_{\bot}(\texttt{t}) \rightarrow^{\ast}_{\lambda} 
\e_{\bot}(\texttt{v})$.
\end{proof}





\subsubsection{Typing Erasure}
\begin{lemma}[Typing relation under erasure]~ \\
	If  $\tystepone{\texttt{t}}{\tau}{\bot}$  
	then \text{$\vdash_{\lambda}  \e_{\bot}(\texttt{t}) 
	~:~\e_{\bot}({\tau})$}.
\end{lemma}
\begin{proof}
By induction on a derivation of the statement 
\text{$\vdash_{g\lambda}  \e_{\bot}(\texttt{t}) ~:~\e_{\bot}({\tau})$}. 
For a given derivation, we proceed by case analysis on the final typing rule
 used in the proof. 
 
 	\noindent\textit{Case} \textsc{T-Unit}:\quad 
 	$\tystepone{\texttt{()}}{\texttt{unit}}{\bot}$
 	
 	Immediately by definition of $\e_{\bot}$.\\[0.08cm] 
% 		As  $\e_{\bot}(\texttt{()}) = \texttt{()}$ and $ \e_{\bot}(\texttt{unit}) = \texttt{unit}$
% 		 we have immediately $\vdash_{\lambda} \texttt{()} : \texttt{unit}$.
 	\noindent\textit{Case} \textsc{T-Var}:\quad 
 	$\tystepone{x_{\tau}^{\bot}}{\tau}{\bot}$
 	 
 	 $\e_{\bot}(x_{\tau}^{\bot}) = x_{\e_{\bot}(\tau)}$ gives  immediately 
 	 $\vdash_{\lambda} x_{\e_{\bot}(\tau)} : \e(\tau)$.\\[0.08cm] 
 	\noindent\textit{Case} \textsc{T-Abs}:\quad
 		$\tystepone{\lambda x_{\tau_{2}}^{\mathfrak{B}_{2}}.\texttt{t}_{1}}
 		{\tau_{2}^{_{2}} \rightarrow 
 		\tau_{1}}{\bot}~ \text{with} \tystepone{\texttt{t}_{1}}{\tau_{1}}{\bot}	$
 		
 		By induction hypothesis $\vdash_{\lambda} \e_{\bot}(\texttt{t}_{1}) : \e_{\bot}(\tau_{1}) $.
 		There are two cases to consider, depending on whether the parameter 
 		of the abstraction is ghost or not. If $\mathfrak{B_{2} = \top}$ then 
 		$\e_{\bot}(\lambda x_{\tau_{2}}^{\top}. \texttt{t}_{1}) = \lambda x_{\texttt{unit}}. \e(\texttt{t}_{1})$
 		and therefore
 		\infrule[T-Abs]
 		{\vdash_{\lambda} \e_{\bot}(\texttt{t}_{1}) : \e_{\bot}(\tau_{2})}
 		{ \vdash_{\lambda} \lambda x_{\texttt{unit}}.\e_{\bot}(\texttt{t}_{1}) : 
 		\texttt{unit} \rightarrow 	\e_{\bot}(\tau_{1})} 
 	
 		Otherwise $\mathfrak{B_{2} = \bot}$ and again by the rule \textsc{T-Abs} 
 		we obtain :
 		\infrule[T-Abs]
 		{\vdash_{\lambda} \e_{\bot}(\texttt{t}_{1}) : \e_{\bot}(\tau_{1})}
 		{ \vdash_{\lambda} \lambda x_{\e_{\bot}(\tau_{2})}.\e_{\bot}(\texttt{t}_{1}) : 
 		\e_{\bot}(\tau_{2}) \rightarrow 	\e_{\bot}(\tau_{1})}
 		
 		 	
 	\noindent\textit{Case} \textsc{T-App}:\quad
 	$\tystepone{\texttt{t}_{1}~\texttt{t}_{2}}{\tau_{1}}{\bot} $ ~
 	with sub-derivations: \\
	\phantom{{Case} \textsc{T-App}: \qquad}
	$\tystepone{\texttt{t}_{1}}
	{\tau_{2}^{\mathfrak{B}_{2}} \rightarrow \tau_{1}}{\mathfrak{B}_{1}}$ \\
	\phantom{\textit{Case} \textsc{T-App}: \qquad}	
	$ \tystepone{\texttt{t}_{2}}{\tau_{2}}{\mathfrak{B'}_{2}} $, \quad 
%	and constraints: \\ 
%	\phantom{\textit{Case} \textsc{T-App}: \qquad}	
%	$ \mathfrak{B}_{1}\vee (\neg \mathfrak{B}_{2} \wedge \mathfrak{B'}_{2})=\bot,$
%	$ \mathfrak{B}_{2} \Rightarrow \mathfrak{B'}_{2} = \top $ \\
	
%	By lemma's statement, $\texttt{t}_{1}~\texttt{t}_{2}$ 
%	should not be a ghost term. Therefore
%	$\mathfrak{B}_{1}\vee (\neg \mathfrak{B}_{2} \wedge \mathfrak{B'}_{2})=\bot$.
%	From that and from the rule's premise condition 
%	$ \mathfrak{B}_{2} \Rightarrow \mathfrak{B'}_{2} = \top $ 
%	we deduce 

As $\tystepone{\texttt{t}_{1}~\texttt{t}_{2}}{\tau_{1}}{\bot} $, the inversion lemma gives as 
By inversion that $\vDash \mathfrak{B}_{2} \Leftrightarrow \mathfrak{B'}_{2}$. That is, we have two cases to consider.
	
	If \text{$ \mathfrak{B}_{2} = \mathfrak{B'}_{2} = \bot $} 
	then by induction hypotheses \\
	$\vdash_{\lambda} \e_{\bot}(\texttt{t}_{1}) : 
	\e_{\bot}(\tau_{2}) \rightarrow \e_{\bot}(\tau_{1}) $ and
	\text{$\vdash_{\lambda} \e_{\bot}(\texttt{t}_{2}) : \e_{\bot}(\tau_{2}) $}.
	By  \textsc{T-App} rule,
	\infrule[T-App]
	{\vdash_{\lambda} \e_{\bot}(\texttt{t}_{1}) : 
	\e_{\bot}(\tau_{2}) \rightarrow \e_{\bot}(\tau_{1}) 
	\qquad
	\vdash_{\lambda} \e_{\bot}(\texttt{t}_{2}) : \e_{\bot}(\tau_{2}) }
	{\vdash_{\lambda} \e_{\bot}(\texttt{t}_{1}~\texttt{t}_{2}): \e(\tau_{1})}  
	
	If \text{$ \mathfrak{B}_{2} = \mathfrak{B'}_{2} = \top $},
	then by definition of $\e$ we have $\e(\texttt{t}_{2}) = \texttt{()}$ 
	and $\e_{\mathfrak{B'_{2}}}(\tau_{2}) = \texttt{unit}$. 
	By induction hypothesis on $\texttt{t}_{1}$,
	\text{$\vdash_{\lambda} \e_{\bot}(\texttt{t}_{1}) : 
	\texttt{unit} \rightarrow \e_{\bot}(\tau_{1}) $}. 
	Applying \textsc{T-App} rule gives us \\

\hspace*{1.2in}
\infer[\mbox{\textsc{(T-App)}}]
 {\vdash_{\lambda} \e_{\bot}(\texttt{t}_{1}~\texttt{t}_{2}): \e_{\bot}(\tau_{1})}
 {\infer
   {\vdash_{\lambda} \e_{\bot}(\texttt{t}_{1}~\texttt{t}_{2}): \e(\tau_{1})}
   {}
   &
 \infer[\mbox{\textsc{(T-Unit)}}]
  {\vdash_{\lambda} \texttt{()} : \texttt{unit}}
  {}}	\\
  
  	The case of (\textsc{T-Ghost}) as well as any other valid derivation  
  	where a typed term is marked as ghost do not satisfy lemma's requirement, 
  	so these cases are trivially verified.									
\end{proof}

TODO:
Extensions:
- If-then-else
- rec
- let in
- match 
- constructors
- ref global mono
- types polymorphes
- operators

Potential difficulties:
- polymorphism
- schemes
- non interference

-->(E,E')

règles de typage
règles d'effacement



