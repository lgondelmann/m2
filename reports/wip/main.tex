%%%%%%%%%%%%%%%%%%%%%%%%%%%%%%%%%%%%%%%%%%%%%%%%%%%%%%%%%%%%%%%%%%%%%%%%%%%%%%%%
\documentclass[a4paper,12pt,oneside]{article}
\label{packages}
\usepackage[sc]{mathpazo} 
\usepackage[scaled]{helvet} % ss
\usepackage[T1]{fontenc}
\usepackage[english]{babel} 
\usepackage[a4paper,pdftex,colorlinks=true,
	urlcolor=blue,pdfstartview=FitH]{hyperref}
\usepackage{graphicx}
\usepackage{amssymb,amsfonts,amsthm,amsmath}
\usepackage{titlesec}
\usepackage{listings}
\usepackage[usenames,dvipsnames]{xcolor}
\usepackage[dvipsnames]{xcolor,colortbl}
\usepackage{caption} 
\protect\usepackage{semantic}
\usepackage{bcprules, proof}
\usepackage{stmaryrd}
 \usepackage{float} 
\usepackage{changepage}
\usepackage{setspace}
\usepackage{pstricks,pstricks-add,pst-math,pst-xkey}
\usepackage{multicol}
\usepackage[small,nohug,heads=vee]{diagrams}
\usepackage{cleveref}
\crefname{enumi}{position}{positions}

\diagramstyle[labelstyle=\scriptstyle]

\titleformat{\section}[hang]% style du titre
  {\normalfont\Large\bfseries}% police du titre + numéro
  {\thesection}% numérotation
  {0.4in}% espacement numéro/titre <--- valeur à changer
  {}% police spécifique du titre
\titlespacing*{\section}
  {-0.5in}% espacement gauche
  {0.2in}% espacement avant
  {0.2in}% espacement après
  [0.1in]% espacement à droite

\titleformat{\subsection}[hang]% style du titre
  {\normalfont\large\bfseries}% police du titre + numéro
  {\thesubsection}% numérotation
  {0.2in}% espacement numéro/titre <--- valeur à changer
  {}% police spécifique du titre
\titlespacing*{\subsection}
  {-0.44in}% espacement gauche
  {0.2in}% espacement avant
  {0.2in}% espacement après
  [0.1in]% espacement à droite


%\titleformat{\subsubsection}[hang]% style du titre
%  {\normalfont\large\bfseries}% police du titre + numéro
%  {\thesubsubsection}% numérotation
%  {0.1in}% espacement numéro/titre <--- valeur à changer
%  {}% police spécifique du titre
%\titlespacing*{\subsubsection}
%  {-0.30in}% espacement gauche
%  {0.2in}% espacement avant
%  {0.2in}% espacement après
%  [0.1in]% espacement à droite


\RequirePackage{listings}
\RequirePackage{amssymb}

\lstset{
  basicstyle={\ttfamily},
  framesep=2pt,
  frame=single,
  keywordstyle={\color{blue}},
  stringstyle=\itshape,
  commentstyle=\itshape,
  columns=[l]fullflexible,
  showstringspaces=false,
}

\lstdefinelanguage{why3}
{
morekeywords={namespace,predicate,function,inductive,type,use,clone,%
import,export,theory,module,end,in,with,%
let,rec,for,to,do,done,match,if,then,else,while,try,invariant,variant,%
absurd,raise,assert,exception,private,abstract,mutable,ghost,%
downto,raises,writes,reads,requires,ensures,returns,val,model,%
goal,axiom,lemma,forall},%
string=[b]",%
sensitive=true,%
morecomment=[s]{(*}{*)},%
keepspaces=true,
}
%literate=%
%{'a}{$\alpha$}{1}%
%{'b}{$\beta$}{1}%
%{<}{$<$}{1}%
%{>}{$>$}{1}%
%{<=}{$\le$}{1}%
%{>=}{$\ge$}{1}%
% {<>}{$\ne$}{1}%
% {/\\}{$\land$}{1}%
% {\\/}{ $\lor$ }{3}%
% {\ or(}{ $\lor$(}{3}%
% {not\ }{$\lnot$ }{1}%
% {not(}{$\lnot$(}{1}%
% {+->}{\texttt{+->}}{2}%
% % {+->}{$\mapsto$}{2}%
% {-->}{\texttt{-\relax->}}{2}%
% %{-->}{$\longrightarrow$}{2}%
% {->}{$\rightarrow$}{2}%
% {<->}{$\leftrightarrow$}{2}%

\lstnewenvironment{whycode}{\lstset{language=why3}}{}
\lstnewenvironment{ocamlcode}{\lstset{language={[Objective]Caml}}}{}




%\lstset{basicstyle={\ttfamily}}
\let\why\lstinline
\newcommand{\HRule}{\rule{\linewidth}{0.5mm}}
\newtheorem{theorem}{Theorem}[section]
\newtheorem{lemma}{Lemma}[section]
\newtheoremstyle{plain}
{\topsep}
{\topsep}{\itshape}{}{}{.}{ }{\textit{#1 #2} [\textsc{#3}]}

\let\mf\mathfrak

\addtolength{\textwidth}{1.5cm}



\def\nrepeat#1#2{\count0=#1 \loop \ifnum\count0>0 \advance\count0 by -1 #2\repeat}



 \begin{document}
\newcommand{\mem}{_{|\mu}}\newcommand{\memp}{_{|\mu'}}

%%%%%%%%%%%%%%%%%%%%%%%%%%%%%%%%%%%%%%%%%%%%%%%%%%%%%%%%%%%%%%%%%%%%%
\newcommand{\glam}{\textit{ghost}-$\lambda$~}
\newcommand{\gml}{\textit{ghost}-ml~}
%%%%%%%%%%%%%%%%%%%%%%%%%%%%%%%%%%%%%%%%%%%%%%%%%%%%%%%%%%%%%%%%%%%%%

\newcommand{\var}[3]{#1^{#2}_{#3}}
\newcommand{\gvar}[3]{#1^{\mathfrak{B_{#2}}}_{\tau_{#3}}}
\newcommand{\gref}[3]
{#1^{\mathfrak{B_{#2}}}_{\mathtt{ref}~\tau_{#3}}}
\let\rvar\gref
\newcommand{\gvarT}[2]{#1^{\top}_{\tau_{#2}}}
\newcommand{\gvarF}[2]{#1^{\bot}_{\tau_{#2}}}
\newcommand{\gabst}[4]{\lambda \gvar{#1}{#2}{#3}. #4}
\newcommand{\gghost}[1]{\mathtt{ghost}~ #1}
\newcommand{\glet}[5]
{\mathtt{let}~\gvar{#1}{#2}{#3} = #4 ~ \mathtt{in}~ #5}    
\newcommand{\gif}[3]{\mathtt{if}~#1~\mathtt{then}~#2~\mathtt{else}~#3}
\newcommand{\grech}[6]
	{\mathtt{rec}~\var{#1}{\mf{B_{#2}}}{}~\gvar{#3}{#4}{#5}~:~\tau_{#6}. t}
\newcommand{\grec}[4]
	{\mathtt{rec}~\var{f}{\mf{B_{#1}}}{}~\gvar{x}{#3}{#4}:\tau_{#2}.~t}
\newcommand{\gread}[3]{!\gref{#1}{#2}{#3}}	
\newcommand{\gwrite}[4]{\gref{#1}{#2}{#3} := #4}		
%%%%%%%%%%%%%%%%%%%%%%%%%%%%%%%%%%%%%%%%%%%%%%%%%%%%%%%%%%%%%%%%%%%%%

%SEMANTICS
\newcommand{\leval}[4]{~#1_{|#2} \rightarrow_{\lambda} #3_{|#4} ~}
\newcommand{\geval}[4]{~#1_{|#2} \rightarrow_{g\lambda} #3_{|#4} ~}

\newcommand{\levalh}[4]{~#1_{|#2} \stackrel{\epsilon}{\rightarrow}_{\lambda} #3_{|#4} ~}
\newcommand{\gevalh}[4]{~#1_{|#2} \stackrel{\epsilon}{\rightarrow}_{g\lambda} #3_{|#4} ~}

\newcommand{\levalstar}[4]{~#1_{|#2} \rightarrow_{\lambda}^{\star} #3_{|#4} ~}
\newcommand{\gevalstar}[4]{~#1_{|#2} \rightarrow_{g\lambda}^{\star} #3_{|#4} ~}
\newcommand{\gstep}[2]{~#1 ~ {\rightarrow}_{g\lambda} ~ #2~}
\newcommand{\ghead}[2]{~#1~\stackrel{\epsilon}{\rightarrow}~#2~}
\newcommand{\gstar}[2]{~#1 ~ {\rightarrow}^{\star}_{g\lambda} ~ #2~}
\newcommand{\glhead}[2]{#1~\stackrel{\epsilon\quad}
										{\rightarrow_{g\lambda}}~#2}
\newcommand{\stepone}[2]{#1 ~ {\rightarrow}~ #2}
\let\eval\stepone


%%%%%%%%%%%%%%%%%%%%%%%%%%%%%%%%%%%%%%%%%%%%%%%%%%%%%%%%%%%%%%%%%%%%%
\newcommand{\tarr}[3]{\tau^{\mf{B}_{#3}}_{#1} \rightarrow \tau_{#2}}
\newcommand{\tarrS}[4]
	{\tau^{\mf{B}_{#3}}_{#1} 
	\stackrel{\Sigma_#4}{\longrightarrow} \tau_{#2}}

\newcommand{\sbst}[3]{#1 [#2 \mapsfrom #3] }
%%%%%%%%%%%%%%%%%%%%%%%%%%%%%%%%%%%%%%%%%%%%%%%%%%%%%%%%%%%%%%%%%%%%%
\newcommand{\grtitle}[2]{#1 & ::= &  & \textit{#2} \\}
\newcommand{\grhead}[3]{#1 & ::= & #2 & \textit{#3} \\}
\newcommand{\grcase}[2]{&  & #1 & \textit{#2} \\}
%%%%%%%%%%%%%%%%%%%%%%%%%%%%%%%%%%%%%%%%%%%%%%%%%%%%%%%%%%%%%%%%%%%%%

% TYPING
\newcommand{\tystepone}[3]{\vdash_{g\lambda}  #1 : (#2, #3) }
\newcommand{\typrule}[5]{~\vdash_{g\lambda}  #1 : (#2, #3, #4) #5}

% ERASURE 
\newcommand{\e}{\mathcal{E}}
\newcommand{\ebot}[1]{\e_{\bot}(#1)}
\newcommand{\etop}[1]{\e_{\top}(#1)}
\newcommand{\evar}[2]{\e_{#1}(#2)}


% CONSTANTS
\newcommand{\glvar}{\gvar{x}{}{}}
\newcommand{\glref}{\gref{r}{}{}}
\newcommand{\glabst}{\gabst{x}{}{}{t}}
\newcommand{\glapp}{t_1 ~ t_2}
\newcommand{\gltyping}{~\typrule{t}{\tau}{\mf{B}}{\Sigma}{}~}
\newcommand{\vardecl}{\text{var }\glref = v}
\newcommand{\gllet}{\glet{x}{}{}{t}{t}} 
\newcommand{\glif}{\gif{t}{t}{t}}
\newcommand{\glrec}{\grec{}{}{}{}}
\newcommand{\glread}{!\gref{r}{}{}}	
\newcommand{\glwrite}{\gref{r}{}{} := t}		
%%%%%%%%%%%%%%%%%%%%%%%%%%%%%%%%%%%%%%%%%%%%%%%%%%%%%%%%%%%%%%%%%%%%%
\newcommand{\Longdownarrow}{\rotatebox{90}{$\Longleftarrow$}}
%%%%%%%%%%%%%%%%%%%%%%%%%%%%%%%%%%%%%%%%%%%%%%%%%%%%%%%%%%%%%%%%%%%%%
\newcommand{\lgvar}[2]{#1_{#2}}
\newcommand{\lgvarc}{x_{\tau}}

\newcommand{\lgabs}[3]{\lambda \lgvar{#1}{#2}.{#3}}
\newcommand{\lgabsc}{\lgabs{x}{\tau_2}{t_1}}

\newcommand{\lgapp}[2]{#1~#2}
\newcommand{\lgappc}{t_1~t_2}

\newcommand{\lglet}[4]{\text{let } \lgvar{#1}{#2} = #3 \text{ in } #4}
\newcommand{\lgletc}{\lglet{x}{\tau_2}{t_2}{t_1}}

\newcommand{\lgif}[3]{\text{ if } #1 \text{ then } #2 \text{ else } #3}
\newcommand{\lgifc}{\lgif{t_1}{t_2}{t_3}}

\newcommand{\lgrec}[5]{\text{rec } #1~\lgvar{#3}{#4} : #2 = #5}
\newcommand{\lgrecc}{\lgrec{g}{\tau_1}{x}{\tau_2}{t_1}}

\newcommand{\lgand}[2]{#1 \wedge #2}
\newcommand{\lgandc}{\lgand{t_1}{t_2}}

\newcommand{\lgor}[2]{#1 \vee #2}
\newcommand{\lgorc}{\lgor{t_1}{t_2}}

\newcommand{\lgneg}[1]{\neg #1}

\newcommand{\lgexist}[3]{\exists \glvar{#1}{#2}. #3}
\newcommand{\lgexistc}{\lgexist{x}{\tau_2}{f_1}}

\newcommand{\lgforall}[3]{\forall \glvar{#1}{#2}. #3}
\newcommand{\lgforallc}{\lgforall{x}{\tau_2}{f_1}}


\section{GhostLambda: Lambda-Calculus with Ghost Code}
%cccccccccccccccccccccccccccccccccccccccccccccccccccccccccccccccccccccccccccccc
%Annotating a program with logical formulae predicting statically it dynamic behaviour is a 
%keystone of the deductive software verification approach.
%
%Typically, these logical formulae are assertions about program assigned variables,
% loop invariants, recursive function variants, etc. 
% 
% However,  it is often useful to annotate programs with some data that appear in program 
%  that appear in assertions but do not affect the program dynamic behaviour in any way. 
%   
% When a correct-by-construction executable code is extracted
% from the specified program, this , called \textit{ghost code} disappear  together with
% specification. 
%The goal of this section is to present the idea of \textit{ghost} code,
%its use in program specification and to state some properties about its 
%extraction.
%
%Ghost code is a kind of program annotation that serves 
%as a support for program specification inside logical formulae and assertions.
%
%The main idea about ghost code is that as any other kind of
% specification it does not affect the program meaning. That is, ghost
% code does not correspond to any physical entity, and as such can by entirely erased from a program, without altering its formal meaning. 
% 
%In this section we describe a language where one can annotate programs 
%with such \textit{ghost code}.
%
%We start by formalizing the \textit{ghost}$\lambda$\textit{-calculus}, 
%a tiny language of simply typed $\lambda$-calculus enriched with ghost variables and ghost expressions.
%
%We then define ghost code \textit{erasure}, which transforms a well-typed g$\lambda $ term  to a term of standard  $\lambda$-calculus.
%Finally we state and proof a few basic preservation properties of such translation.
%%%%%%%%%%%%%%%%%%%%%%%%%%%%%%%%%%%%%%%%%%%%%%%%%%%%%%%%%%%%%%%%%%%%%%%%%%%%%%%
\subsection{g$\lambda$-calculus syntax and semantics}
\renewcommand{\arraystretch}{1.1}
%The syntax of \gml is summarized below.\vspace{-0.5cm}
\begin{figure}[H]
\begin{adjustwidth}{-3em}{-3em}
\begin{displaymath}
\begin{array}{ll@{\hspace*{-0.1cm}}
							 l@{\hspace*{1.5cm} }
							 l|@{\hspace*{0.5cm}}
							 llll}
	\hline

  \texttt{p} & ::= 
  & & \textsc{programs} &  
  \texttt{v} & ::= 
  & & \textsc{values}\\

	&  \mathtt{var}~r^{\mathfrak{B}}_{\text{ref }\tau}: v~; 
	\quad p       
	& & \textit{reference declaration} &
	& 	\texttt{c}             
	& & \textit{constant}\\ 
 
  & t 
  & & \textit{body} &
	& \gabst{x}{}{}{t}        
	& & \textit{abstraction}\\    

	& & & &	 
	& \mathtt{rec} ~ \gvar{f}{}{} ~ \gvar{x}{}{}. t    
  & &  \textit{recursive function} \\ 
 
  \texttt{t} & ::= 
  & & \textsc{terms} &  
  & & &\\
     
  &     \texttt{c}
  & & \textit{constant} &
	\texttt{c} & ::= 
	& & \textsc{constants} \\
   	  
    	      
  &     x^{\mathfrak{B}}_{\tau} 
  & & \textit{variable} &
	&     0 ~| 1 ~| ... ~| n        
	& & \textit{integer} \\   	
  
  
  &    \gabst{x}{}{}{t}         
  & & \textit{abstraction} &   	
  &   \mathtt{true}~|\mathtt{false} 
  & & \textit{boolean} \\
        
  &    t~t    
  & & \textit{application} &
 	&  () 
 	& & \textit{unit} \\  	
   	
  & \mathtt{ghost} t  
  & & \textit{ghost~ term} &  
  & +|~-|~= 
  & & \textit{built-in operators}  \\ 
   	
  & \mathtt{let} ~ \gvar{x}{}{} = t~ \mathtt{in}~ t    
  & & \textit{local binding} 
  & & & & \\		
   								   
  & \mathtt{if}~t~\mathtt{then}~t~\mathtt{else}~t
  & & \textit{if-branching}    
  &  \tau & ::= 
  & & \textsc{types} \\ 
   		 								      								   
  &	\mathtt{rec} ~ \gvar{f}{}{} ~ \gvar{x}{}{}. t    
  & &  \textit{recursive function} &
  &  \mathtt{int}~ | ~\mathtt{bool}~ | ~\mathtt{unit}~  
  & & \textit{built-in types} \\ 
   	   		
   	   								      								   
  &  	\mathtt{!}\gvar{r}{}{} 
  & &  \textit{reference access} &
  &     \tau^{\mathfrak{B}}  \rightarrow \tau  
  & & \textit{function type} \\    					
 	   
	&  	\gvar{r}{}{} := t  
	& &  \textit{reference assignment} & \\
	
	& & & &		\mathtt{ref}~\tau  &
	& & \textsc{reference's type}  \\   
  
  & & & & \\  
   	
  & & & & \mathfrak{B} & ::= 
  & & \textsc{ghost indicator} \\
         
  & & & & &  \bot \mathtt{(* false *)} 
  & & \textit{raw code} \\ 
    
  & & & & &  \top \mathtt{(* true *)} 
  & & \textit{ghost code} \\ 
 
 \hline						   
  \end{array} 
\end{displaymath} 
\caption{ \textbf{\gml syntax} \hfill}
\end{adjustwidth}
\end{figure}
%
%Here, the letters \textit{p, t, v, c} are meta-variables for programs, terms, variables and constants respectively. 
%
%As we can see, g$\lambda$-calculus is just an extension of standard simply 
%typed $\lambda$-calculus with \texttt{arrow} and \texttt{unit} enriched
%with a construction \texttt{ghost t} for ghost terms. 
%Moreover, variables and formal arguments in abstraction are annotated with
%their type and \textit{ghost} mark $\mathfrak{B}$ which indicates whether 
%it is ghost ($\top$) or not ($\bot$). \\
%
%The evaluation of g$\lambda$-calculus corresponds as well to the standard SOS call-by-value semantics of $\lambda$-calculus, except for the rule  \makebox{(\textsc{E-DeGhost})}.

\newcommand{\mem}{_{|\mu}}\newcommand{\memp}{_{|\mu'}}
\begin{figure}[H]
	\hrulefill
	\begin{spacing}{1.1} 
	
	\infax[E-Const] 
	{\stepone
		{ c~v\mem }
		{ \delta(c,v)}\mem \qquad \text{if } \delta(c,v) \text{ is defined } } 
	
	\infax[E-AppFun] 
	{ \stepone
		{ (\lambda \gvar{x}{}{}.t)v\mem }
		{ \sbst{t}{\gvar{x}{}{}}{v}}\mem } 
	  

	\infax[E-AppRec]
	{\stepone
		{(\mathtt{rec}~ \gvar{f}{1}{1} ~ \gvar{x}{2}{2}. t) v\mem}
		{t[\gvar{x}{2}{2} \mapsfrom v, 
			 \gvar{f}{1}{1} \mapsfrom 
			 \mathtt{rec}~ \gvar{f}{1}{1}~\gvar{x}{2}{2}.t]}\mem}		
	
	\infax[E-LetV]
	{\stepone
	 {\mathtt{let} ~ \gvar{x}{}{} = v_{1} ~ \mathtt{in}~ {t_{2}}\mem}
	 {\sbst{t_{2}}{x^{b}_{\tau}}{v_{1}}}\mem} 
	
	\infax[E-If-true]		
	{\stepone
		{\mathtt{if}~\mathtt{true}~\mathtt{then}~t_{1}~\mathtt{else}~t_{2}{\mem}}
		{t_{1}{\mem}}}
	
	\infax[E-If-false]		
	{\stepone
		{\mathtt{if}~\mathtt{false}~\mathtt{then}~t_{1}~\mathtt{else}~t_{2}{\mem}}
		{t_{2}{\mem}}}	

	\infax[E-DeGhost]	 	
	{\stepone{\text{ghost } t\mem}{t \mem}}

	\infax[E-Deref]	 	
	{\stepone
	{\gref{r}{}{}{\mem}}
	{\mu_{\mathfrak{B}}(\gref{r}{}{})}}

	\infax[E-Assign]
	{\stepone
		{!\gref{r}{}{} := v\mem}
		{()_{|\mu_{\mathfrak{B}}[\gref{r}{}{} \mapsfrom v]}}}
	
	\infrule[E-Let]
	{{t_{1}}\mem \rightarrow {{t'}_{1}}\memp}
	{\mathtt{let} ~ \gvar{x}{}{} = t_{1} ~ \mathtt{in}~ {t_{2}}\mem
		\rightarrow 
		\mathtt{let} ~ \gvar{x}{}{} = t_{1}' ~ \mathtt{in}~ {t_{2}}\memp}
  
	\infrule[E-AppLeft]
	{{t_{1}}\mem \rightarrow {t'_{1}}\memp}
	{{t_{1} t_{2}}\mem
						\rightarrow {t'_{1}t_{2}}\memp}
  
	\infrule[E-AppRight]
	{{\texttt{t}_{2}}\mem \rightarrow {\texttt{t}'_{2}}\memp}
	{{\texttt{v}_{1} \texttt{t}_{2}}\mem 
	\rightarrow {\texttt{v}_{1}\texttt{t}'_{2}}\memp}  

	\infrule[E-If]
	{{t_{1}}\mem \rightarrow {t'_{1}}\memp}
	{\mathtt{if}~t_{1}~\mathtt{then}~t_{2}~\mathtt{else}~t_{3}{\mem}	
	\rightarrow \mathtt{if}~t'_{1}~\mathtt{then}~t_{2}~\mathtt{else}~t_{3}{\memp}}

	\infrule[E-Assign-Ctx]
	{{t_{1}}\mem \rightarrow {t'_{1}}\memp}
	{\stepone
		{\gref{r}{}{} := t_{1}{\mem}}
		{\gref{r}{}{} := t'_{1}{\memp}}}
\end{spacing}
\caption{ \textbf{\gml small step operation semantics} \hfill}
\hrulefill	\\
\end{figure}
\subsection{Typing Relation}
\begin{figure}[H]
\hrulefill
\begin{spacing}{1.11}

\begin{multicols}{2} 
	\infrule[T-Const]
		{\text{Type}(c) = \tau}
		{\typrule
			{c}
			{\tau}
			{\bot}
			{\bot}}
		
	\infrule[T-Var]
		{}
		{\typrule
			{\gvar{x}{}{}}
			{\tau}
			{\mathfrak{B}}
			{\bot}}
	\end{multicols}
	\infrule[T-Abst]
		{\typrule
			{t_{1}}
			{\tau_{1}}
			{\mathfrak{B_{1}}}
			{\Sigma_{1}}
			{\textcolor{BrickRed}
				{\overline{\mathfrak{B}_{1} \wedge \Sigma_{1}}}}}
		{\typrule
			{\lambda \gvar{x}{2}{2}.t_{1}}
			{\tau_{2}^{\mathfrak{B_{2}}} \rightarrow \tau_{1}}
			{\mathfrak{B_{1}}}
			{\Sigma_{1}}}
			
	\infrule[T-App]
		{ \qquad\qquad
		\typrule
		 {t_{1}}
		 {\tau_{2}^{\mathfrak{B}_{2}} 
		 \stackrel{\Sigma_{0}}{\longrightarrow} \tau_{1}}
		 {\mathfrak{B}_{1}}
		 {\Sigma_{1}}	
		{\typrule
			{t_{2}}
			{\tau_{2}}
			{\mathfrak{B'_{2}}}
			{\Sigma_{2}}\\
			{\textcolor{BrickRed}
			{\overline{(\mathfrak{B}_{1} \vee 
			(\overline{\mathfrak{B}_{2}}\wedge
			\mathfrak{B'}_{2}))	\wedge	
			(\Sigma_{0}\vee\Sigma_{1}\vee\Sigma_{2})}
			\quad
			\textcolor{BrickRed}
			{\overline{\mathfrak{B'_{2}} \wedge \Sigma_{2}}}
			\quad				
			\textcolor{BrickRed}
			{\mathfrak{B}_{2} \Rightarrow \mathfrak{B'}_{2}}}}
			}}
		{\typrule
			{t_{1}~t_{2}}
			{\tau_{1}}
			{\mathfrak{B}_{1} \vee 
			(\overline{\mathfrak{B}_{2}}\wedge
			\mathfrak{B'}_{2})}
			{\Sigma_{0}\vee\Sigma_{1}\vee\Sigma_{2}}}

	\infrule[T-If]
		{
		\typrule
		{t_{1}}
		{\text{bool}}
		{\mathfrak{B_{1}}}
		{\Sigma_{1}}
		\typrule
			{t_{2}}
			{\tau_{0}}
			{\mathfrak{B}_{2}}
			{\Sigma_{2}}
			\typrule
			{t_{3}}
			{\tau_{0}}
			{\mathfrak{B}_{3}}
			{\Sigma_{3}}\\
			{\textcolor{BrickRed}
			{\overline{
			(\mathfrak{B}_{1}\vee
			\mathfrak{B}_{2}\vee\mathfrak{B}_{3}) \wedge
			 (\Sigma_{1}\vee\Sigma_{2}\vee\Sigma_{3})}}}}
		{\typrule
			{\text{if } t_{1} \text{ then } t_{2} 
		 	\text{ else } t_{3}}
		 	{\tau_{0}}
		 	{\mathfrak{B_{1} \vee \mathfrak{B_{2} \vee
		 	\mathfrak{B_3}}}}
		 	{\Sigma_{1} \vee \Sigma_{2} \vee \Sigma_{3}}
		 }	
	
	\infrule[T-Rec]
		{\typrule
			{t}
			{\tau_{1}}
			{\mathfrak{B}_{1}}
			{\Sigma_{1}}
			{\textcolor{BrickRed}
			{\overline{\mathfrak{B}_{1} \wedge \Sigma_{1}}}}}
		{\typrule
			{\text{rec } 
			f^{\mathfrak{B_{1}}}_{\tau_{2}^{\mathfrak{B_{2}}}
			\rightarrow \tau_{1}}~
			\gvar{x}{2}{2}. t_{1}}
			{\tau_{2}^{\mathfrak{B_{2}}}
			\rightarrow \tau_{1}}
			{\mathfrak{B_{1}}}
			{\Sigma_{1}}}
	
	\infrule[T-Let]
		{
		\typrule
			{t_{1}}
			{\tau_{1}}
			{\mathfrak{B_{1}}}
			{\Sigma_{1}}
		{\typrule
			{t_{2}}
			{\tau_{2}}
			{\mathfrak{B'_{2}}}
			{\Sigma_{2}}
			{\textcolor{BrickRed}
			{\overline{\mathfrak{B'}_{2} \wedge \Sigma_{2}}
			}}}\\[0.2cm]
			{\textcolor{BrickRed}
			{\overline{(\mathfrak{B}_{1} \vee 
			(\overline{\mathfrak{B}_{2}}\wedge
			\mathfrak{B'}_{2}))	\wedge	
			(\Sigma_{1}\vee\Sigma_{2})}
			\quad	\mathfrak{B}_{2}
			\Rightarrow \mathfrak{B'}_{2}}}}
		{\typrule
			{\text{let }\gvar{x}{2}{2}=t_{2}\text{ in }t_{1}}
			{\tau_{1}}
			{\mathfrak{B}_{1} \vee 
			(\overline{\mathfrak{B}_{2}}\wedge
			\mathfrak{B'}_{2})}
			{\Sigma_{1}\vee\Sigma_{2}}}
		
	\begin{multicols}{2}
	\infrule[T-Ghost]
		{\typrule
			{t}
			{\tau}
			{\mathfrak{B}}
			{\bot}}
		{\typrule
			{\mathtt{ghost  }~t}
			{\tau}
			{\top}
			{\bot}}

	\infrule[T-Deref]
		{}
		{\typrule
			{!\rvar{r}{}{}}
			{\tau}
			{\mathfrak{B}}
			{\bot}}
	\end{multicols}
	
	\infrule[T-Assign]
		{\typrule
			{t_{2}}
			{\tau_{2}}
			{\mathfrak{B_{2}}}
			{\Sigma_{2}}
			{\textcolor{BrickRed}
			{\overline{\mathfrak{B}_{2} \wedge \Sigma_{2}}
			\quad 
			{\mathfrak{B_{2}} \Rightarrow \mathfrak{B_{1}}}}}}
		{\typrule
			{\rvar{r}{1}{2} := t_{2}}
			{\text{unit}}
			{\mathfrak{B_{1}}}
			{\overline{\mathfrak{B}}_{1} \vee \Sigma_{2}}}

\end{spacing}
\caption{ \textbf{\gml typing rules} \hfill}
\hrulefill
\end{figure}
%
%										
%The rules \textsc{(T-Unit)}, \textsc{(T-Var)} and \textsc{(T-Abs)} are straightforward.
%
%The rule \textsc{(T-Ghost)} states simply that a term of \texttt{ghost t}
%is a ghost code, whether \texttt{t} is ghost or not. That is, if \texttt{t} were a well-typed non-ghost term, once we put inside a ghost code, it becomes ghost itself. 
%
%The more complex rule \textsc{(T-App)} states that formal argument of $\texttt{t}_{1}$ and actual parameter $\texttt{t}_{2}$ must be
%of the same type. Moreover, it forbids the application of normal code
%to a function whose formal argument is ghost.  
%
%However, a function whose formal argument is not ghost is applied to a 
%ghost code, then the application itself becomes ghost, whether function's
%body is ghost or not. We summarized that is the following table:
%
%\begin{figure}[H]
%\begin{center}
%\begin{tabular}{|c|c|c|c|l|}
%\hline
%$\mathfrak{B_{1}}$ & $\mathfrak{B_{2}}$ & $\mathfrak{B'_{2}}$
% &  \multicolumn{2}{c|}{result} \\
%\hline
%$\bot$ & $\bot$ & $\bot$ & $\bot$ &  normal code application \\ \hline
%$\top$ & $\bot$ & $\top$ & $\bot$ &  ghost-code passing inside normal code
%\\ \hline
%$\top$ & $\top$ & $\top$ & $\top$ &  ghost code application \\ \hline
%$\bot$ & $\top$ & $\bot$ & $\top$ &  normal code passing inside ghost code
% \\ \hline
% \rowcolor{lightgray} 
% $\bot$ & $\top$ & $\top$ & $\top$ &formal parameter contamination \\ \hline
%  \rowcolor{lightgray} 
% $\bot$ & $\bot$ & $\top$ & $\top$ &function parameter and body contamination \\ \hline
%  $\top$ & $\bot$ & $\bot$ & $-$ & impossible \\ \hline
% $\top$ & $\top$ & $\bot$ & $-$ & impossible \\ \hline
% 
%\end{tabular}
%\end{center}
%\end{figure}  

 
%
%
%
%
%\subsubsection{Properties of typing}
%\begin{lemma}[Inversion of typing relation]~\\
%\vspace{-0.5cm}
%\begin{enumerate}
%	\item 
%	if $\tystepone{\texttt{ghost t}}{\tau_{1}}{\mathfrak{B_{1}}}$ 
%	then	$ \mathfrak{B_{1}} = \top $ 
%	and $\tystepone{\texttt{t}}{\tau_{1}}{\mathfrak{B_{2}}} $.
%%	with $\mathfrak{B_{2}} = \{\bot, \top\}$.
%	
% 	\item
% 	if $\tystepone
% 			{\lambda x_{\tau_{2}}^{\mathfrak{B}_{2}}.\texttt{t}}
% 			{\tau_{1}}
% 			{\mathfrak{B_{1}}}$
% 	then $\tau_{1} = \tau_{2}^{\mathfrak{B}_{2}} \rightarrow \tau_{11}$
% 	for some $\tau_{11}$ with \mbox{$\tystepone{\texttt{t}}{\tau_{11}}{\mathfrak{B_{1}}} $}.
%  
% \item 
% 	If ${\tystepone{\texttt{t}_{1}~ \texttt{t}_{2} }{\tau_{1}}{\mathfrak{B}_{1}}}$
%	then there exist $\tau_{11}$, $\tau_{2}$, $\mathfrak{B_{2}}$ and 	
%	$\mathfrak{B'_{2}}$
%	such that \\ 
%	$\tystepone
%		{\texttt{t}_{1}}
%		{\tau_{2}^{\mathfrak{B}_{2}} \rightarrow \tau_{11}}
%		{\mathfrak{B_{1}}}$
%		and 
%	${\tystepone{\texttt{t}_{2}}{\tau_{2}}{\mathfrak{B'}_{2}}}$ with \\
%	 \quad
%	\mbox{$\vDash \mathfrak{B}_{1} \vee (\neg \mathfrak{B}_{2} \wedge \mathfrak{B'}_{2} ) \wedge (\mathfrak{B}_{2} \Rightarrow \mathfrak{B'}_{2})$}. 
%	
%	In particular, if 
% 		${\tystepone{\texttt{t}_{1}~ \texttt{t}_{2} }{\tau_{1}}{\bot}}$
%then $\vDash \mathfrak{B_{2}} \Leftrightarrow \mathfrak{B'_{2}} $.
%\end{enumerate}
%\end{lemma} 
%\begin{proof}
%Straightforward from definition of the typing relation.
%\end{proof}
%
%\begin{lemma}[Progress] 
%If \texttt{t} is a closed, well-typed term 
%(that is ${\tystepone{\texttt{t}}{\tau}{\mathfrak{B}}}$ 
%for some $\tau$ and $\mathfrak{B}$, 
%then either \texttt{t} is a value or else there is some \texttt{t}' 
%such that \mbox{\texttt{t} $\rightarrow_{g\lambda}$ \texttt{t}'}.
%\end{lemma}
%\begin{proof}
%Admit.
%\end{proof}
%
%Note that the typing preservation works only for well-typed non-ghost terms. Indeed,as the following example shows, the contamination of non-ghost code makes the preservation fail for ghost code: if we evaluate
%ghost code \quad \texttt{ghost} $(\lambda \gvarF{x}{unit}. \gvarF{x}{unit}) 
%\rightarrow_{g\lambda} \lambda \gvarF{x}{unit}. \gvarF{x}{unit}$,  
%then \\
%${\tystepone
%{\texttt{ghost} (\lambda \gvarF{x}{unit}. \gvarF{x}{unit}) }{unit \rightarrow unit}{\top}}$ holds, while the typing relation  ${\tystepone{\gvarF{x}{unit}}{unit}{\top}}$ fails.
%
%
%
%\begin{lemma}[Non-ghost Code Typing Preservation] 
%If ${\tystepone{\texttt{t}}{\tau}{\mathfrak{\bot}}}$ 
%and \mbox{\texttt{t} $\rightarrow_{g\lambda}$ \texttt{t}'},
%then ${\tystepone{\texttt{t'}}{\tau}{\mathfrak{\bot}}}$ 
%\end{lemma}
%\begin{proof}
%Admit.
%\end{proof}
%
%\begin{theorem}[Soundness] 
%\end{theorem}
%\begin{proof}
%Admit.
%\end{proof}
%--------------------------------------------------------------------------%
\subsection{Ghost Code Erasure}
\newcommand{\e}{\mathcal{E}}
\theoremstyle{remark}
\newtheorem{dfn}[theorem]{Definition}

\qquad Once we formally defined the simply typed lambda-calculus enriched with ghost expressions, our goal is to show that the computational meaning
 of non ghost well typed non-ghost $g\lambda$-terms is preserved by their translation to simply typed lambda-calculus.

To achieve this, we start by defining the \textit{type-} and \textit{term-} \textit{erasure} functions from g$\lambda$-calculus to simply typed  $\lambda$-calculus. 


\subsubsection{Ghost Erasure}


\begin{dfn}[Type-Erasure] 
\label{type-erasure}
We define type-erasure by induction on the structure of types : \vspace{0.2cm} 

\noindent$ \e_{\top}(\tau) = \texttt{unit} $ 

\noindent$ \e_{\bot}(\texttt{unit}) = \texttt{unit}$ 

\noindent$ \e_{\bot}(\tau_{2}^{\mathfrak{B_{2}}} \rightarrow \tau_{1})  
= \e_{\mathfrak{B_{2}}}(\tau_{2}) \rightarrow \e_{\bot}(\tau_{1}) $. \\
\end{dfn} 

\begin{dfn}[Term-Erasure] 
\label{term-erasure}
Let \texttt{t} be a term such that 
~~ $\tystepone{\texttt{t}}{\tau}{\mathfrak{B}}$ ~ holds. 
We define term-erasure function by induction on 
the structure of \texttt{t}, parametrizing it by the value of 
$ \mathfrak{B}$. \\[0.1cm]
$ \e_{\top}(\texttt{t}_{1}) = \texttt{()}$ 
\quad where $ \tystepone{\texttt{t}_{1}}{\tau_{1}}{\top}$.\\[0.05cm]
$\e_{\bot}(\texttt{()}) = \texttt{()}$\\[0.05cm]
$\e_{\bot}(x_{\tau}^{\bot}) = x_{\e_{\bot}(\tau)}$\\[0.05cm]  
$\e_{\bot}(\lambda x_{\tau_{2}}^{\mathfrak{B_{2}}}.t_{1}) 
= \lambda x_{\e_{\mathfrak{B_{2}}}(\tau_{2})}. \e_{\bot}(t_{1})$
\quad where $ \tystepone{\texttt{t}_{1}}{\tau_{1}}{\bot}$.\\[0.05cm]
$\e_{\bot}(\texttt{t}_{1} ~ \texttt{t}_{2}) 
= \e_{\bot}(\texttt{t}_{1})  ~ \e_{\mathfrak{B_{2}}}(\texttt{t}_{2})$
\quad 
where $ \tystepone{\texttt{t}_{1}}{\tau_{1}}{\bot}$
and $ \tystepone{\texttt{t}_{2}}{\tau_{2}}{\mathfrak{B_{2}}}$.
\end{dfn}
As it can be seen, the erasure function is a morphism that preserve
the structure of operational (not ghost) terms and their types ($\sim\e_{\bot}(\star)$), and sends ghost expressions and types to \texttt{()} and \texttt{unit} respectively ($\sim\e_{\top}(\star)$).

Here is two examples of term erasure:

$ 
\e_{\top}[
(\lambda \gvarF{z}{\tau_{1}}.\gvarF{z}{\tau_{1}})(\texttt{ghost }())] 
= () $

$
\e{\bot}[
(\lambda \gvarT{x}{\tau_{1}}. 
\lambda \gvarF{y}{\tau_{1}}. \gvarF{y}{\tau_{1}}) 
(\texttt{ghost }())] = 
(\lambda x_{\texttt{unit}}. \lambda y_{\tau_{1}}.y_{\tau_{1}})() $ 




%--------------------------------------------------------------------------%
\subsection{Properties of ghost erasure}

\quad Now that we defined the erasure-translation of $g\lambda$-calculus to $\lambda$-calculus, our concern is to show that evaluation result of well-typed operational terms as well as their typing  are preserved under erasure. 
First off we need to state and prove a few basic lemmas. 





%cccccccccccccccccccccccccccccccccccccccccccccccccccccccccccccccccccccccccccccc
%%%%%%%%%%%%%%%%%%%%%%%%%%%%%%%%%%%%%%%%%%%%%%%%%%%%%%%%%%%%%%%%%%%%%%%%%%%%%%%
\subsubsection{Evaluation Preservation}
\begin{lemma}[Substitution under erasure] ~\\
If $\tystepone
		{\texttt{t}_{1}}
		{\tau_{1}}
		{\bot}$  
and 
	\mbox{
		$\tystepone
			{\texttt{v}_{2}}
			{\tau_{2}}
			{\mathfrak{B_{2}}}$}	
hold, \\ 
then
$\e_{\bot}
	(\sbst
		{\texttt{t}_{1}}
		{x^{\mathfrak{B_{2}}}_{\tau_{2}}}
		{\texttt{v}_{2}})
= \sbst
		{\e_{\bot}(\texttt{t}_{1})}
		{x_{\e_{\mathfrak{B_{2}}}(\tau_{2})}}
		{\e_{\mathfrak{B_{2}}}(\texttt{v}_{2})}$
\label{Substitution under erasure}
\end{lemma}

\begin{proof}
By induction on the structure of $\texttt{t}_{1}$.  \\

\noindent\textit{Case} 
$\texttt{t}_{1} = x^{\mathfrak{B_{2}}}_{\tau_{2}}$:\\
In that case,  we can deduce that $\mathfrak{B_{2}} = \bot$. Therefore:

\begin{center}
 $\e_{\bot}
	(\sbst
		{x^{\bot}_{\tau_{2}}}
		{x^{\bot}_{\tau_{2}}}
		{\texttt{v}_{2}}) 
	= \e_{\bot}(\texttt{v}_{2}) $ 
	$= \sbst
			{x_{\e_{\bot}(\tau_{2})}}
			{x_{\e_{\bot}(\tau_{2})}}
			{\e_{\bot}(\texttt{v}_{2})}$ \\[0.08cm]
	$= \sbst
			{\e_{\bot}(x^{\bot}_{\tau_{2}})}
			{x_{\e_{\bot}(\tau_{2})}}
			{\e_{\bot}(\texttt{v}_{2})}$
\end{center}

\noindent\textit{Case} $
\texttt{t}_{1} = y^{\bot}_{\tau'_{2}} \neq x^{\mathfrak{B_{2}}}_{\tau_{2}}$:
\begin{center}
 $\e_{\bot}
	(\sbst
		{y^{\bot}_{\tau'_{2}}}
		{x^{\mathfrak{B_{2}}}_{\tau_{2}}}
		{\texttt{v}_{2}}) 
	= \e_{\bot}(y^{\bot}_{\tau'_{2}}) 
	= y_{\e_{\bot}(\tau'_{2})} $ \\[0.08cm]
$ = \sbst
			{y_{\e_{\bot}(\tau'_{2})}}
			{x_{\e_{\mathfrak{B_{2}}}(\tau_{2})}}
			{\e_{\mathfrak{B_{2}}}(\texttt{v}_{2})}$ 
$ = \sbst
			{\e_{\bot}(y^{\bot}_{\tau'_{2}})}
			{x_{\e_{\mathfrak{B_{2}}(\tau_{2})}}}
			{\e_{\mathfrak{B_{2}}}(\texttt{v}_{2})}$
\end{center}

\noindent\textit{Case} 
$\texttt{t}_{1} 
= \lambda y^{\mathfrak{B'_{2}}}_{\tau'_{2}}. t_{11} \quad \text{ with }   
y^{\mathfrak{B'_{2}}}_{\tau'_{2}} \not\in \texttt{FV}(\texttt{v}_{2}) 
\text{ and } \neq x^{\mathfrak{B_{2}}}_{\tau_{2}} : $
\begin{center}
 $\e_{\bot}
	(\sbst
		{(\lambda y^{\mathfrak{B'_{2}}}_{\tau'_{2}}. t_{11})}
		{x^{\mathfrak{B_{2}}}_{\tau_{2}}}
		{\texttt{v}_{2}})
	= \e_{\bot}[\lambda y^{\mathfrak{B'_{2}}}_{\tau'_{2}}.
	(\sbst
		{ t_{11}}
		{x^{\mathfrak{B_{2}}}_{\tau_{2}}}
		{\texttt{v}_{2}})] $\\[0.08cm]
$	= \lambda y_{\e_{\mathfrak{B'_{2}}}(\tau'_{2})}.
		\e_{\bot}(
		\sbst
			{\texttt{t}_{11}}
			{x^{\mathfrak{B_{2}}}_{\tau_{2}}}
			{\texttt{v}_{2}})] $\\[0.08cm]
$ \stackrel{Ind.Hyp.}{=} 
	\lambda y_{\e_{\mathfrak{B'_{2}}}(\tau'_{2})}.		
	\sbst
			{\e_{\bot}(\texttt{t}_{11})}
			{x_{\e_{\mathfrak{B_{2}}}(\tau_{2})}}
			{\e_{\mathfrak{B_{2}}}(\texttt{v}_{2})}$ \\[0.08cm]
$	= \sbst
			{(\lambda y_{\e_{\mathfrak{B'_{2}}}(\tau'_{2})}.
				\e_{\bot}(\texttt{t}_{11}))}
			{x_{\e_{\mathfrak{B_{2}}}(\tau_{2})}}
			{\e_{\mathfrak{B_{2}}}(\texttt{v}_{2})}$ \\[0.08cm]
$ = 
	\sbst
			{\e_{\bot}(\lambda y^{\mathfrak{B'_{2}}}_{\tau'_{2}}. t_{11}) }
			{x_{\e_{\mathfrak{B_{2}}}(\tau_{2})}}
			{\e_{\mathfrak{B_{2}}}(\texttt{v}_{2})}$
\end{center}

\noindent\textit{Case} 
$\texttt{t}_{1} = \texttt{t}_{11}\texttt{t}_{12} $
\begin{center}
 $\e_{\bot}
	(\sbst
		{\texttt{t}_{11}\texttt{t}_{12}}
		{x^{\mathfrak{B_{2}}}_{\tau_{2}}}
		{\texttt{v}_{2}}) 	$ 
$	= \e_{\bot}
	(\sbst
		{\texttt{t}_{11}}
		{x^{\mathfrak{B_{2}}}_{\tau_{2}}}
		{\texttt{v}_{2}} ~
	\sbst
		{\texttt{t}_{12}}
		{x^{\mathfrak{B_{2}}}_{\tau_{2}}}
		{\texttt{v}_{2}}) 	  $ \\[0.08cm]
$ = \e_{\bot}
	(\sbst
		{\texttt{t}_{11}}
		{x^{\mathfrak{B_{2}}}_{\tau_{2}}}
		{\texttt{v}_{2}}) ~
	\e_{\bot}
	(\sbst
		{\texttt{t}_{12}}
		{x^{\mathfrak{B_{2}}}_{\tau_{2}}}
		{\texttt{v}_{2}}) 	  $ \\[0.08cm]
$ \stackrel{Ind.Hyp.}{=} 
		(\sbst
			{\e_{\bot}({\texttt{t}_{11}})}
			{x_{\e_{\mathfrak{B_{2}}}(\tau_{2})}}
			{\e_{\mathfrak{B_{2}}}(\texttt{v}_{2})})	
		(\sbst
			{\e_{\bot}({\texttt{t}_{12}})}
			{x_{\e_{\mathfrak{B_{2}}}(\tau_{2})}}
			{\e_{\mathfrak{B_{2}}}(\texttt{v}_{2})})	$ \\[0.08cm]
$ = 
	\sbst
		{\e_{\bot}(\texttt{t}_{11} \texttt{t}_{12})}
		{x_{\e_{\mathfrak{B_{2}}}(\tau_{2})}}
		{\e_{\mathfrak{B_{2}}}(\texttt{v}_{2})})	$\\[0.08cm]
\end{center}
 \end{proof}



\theoremstyle{remark}

\begin{lemma}[One-step evaluation under erasure]
For any closed g$\lambda$-term \texttt{t} such that
$\tystepone{\texttt{t}}{\tau}{\bot}$ holds, 
if	$\texttt{t} \rightarrow_{g\lambda} \texttt{t}'$ 
for some term \texttt{t}', 
then either 
\text{$\e_{\bot}(\texttt{t}) \rightarrow_{\lambda} \e_{\bot}(t')$} 
or 		$\e_{\bot}(\texttt{t}) = \e_{\bot}(t')$. 
\end{lemma}

\begin{proof} 
By induction on the evaluation relation of 
$\texttt{t} \rightarrow_{g\lambda} \texttt{t}'$. \\

\noindent\textit{Case} \textsc{E-AppAbs}: \qquad
$ ~\texttt{t} 
= (\lambda x^{\mathfrak{B_{2}}}_{\tau_{2}}.
		\texttt{t}_{1})\texttt{v}_{1}$ ~with~
$ (\lambda x^{\mathfrak{B_{2}}}_{\tau_{2}}.
		\texttt{t}_{1})\texttt{v}_{1} 
	\stackrel{\epsilon\quad}{\rightarrow_{g\lambda}}  
	\sbst
		{\texttt{t}_{1}}
		{x^{\mathfrak{B_{2}}}_{\tau_{2}}}
		{\texttt{v}_{1}}$\\
\phantom{\noindent\textit{Case} \textsc{E-AppAbs}: \qquad }		
$ \tystepone
		{(\lambda x^{\mathfrak{B_{2}}}_{\tau_{2}}.\texttt{t}_{1})\texttt{v}_{1}}	
		{\tau_{1}}
		{\mathfrak{B_{1}}}, \quad 
 	\tystepone
 		{\texttt{v}_{1}}
 		{\tau_{2}}
 		{\mathfrak{B'_{2}}}, $ \\
\phantom{\noindent\textit{Case} \textsc{E-AppAbs}: \qquad }		
$ \mathfrak{B_{1}} = \bot,  \quad 
	\vDash\mathfrak{B_{2}} \Leftrightarrow \mathfrak{B'_{2}} $ 

\begin{center}
	\begin{tabular}{lll}
   & $ \e_{\bot}
  [(\lambda x^{\mathfrak{B_{2}}}_{\tau_{2}}.
  		\texttt{t}_{1})\texttt{v}_{1}]$ &\\
   & $ = \lambda x_{\e_{\mathfrak{B_{2}}}(\tau_{2})}. 
   \e_{\bot}(\texttt{t}_{1}))\e_{\mathfrak{B'_{2}}}(\texttt{v}_{1})$ 
   & (as $ \mathfrak{B_{1}} = \bot $)  \\
&  $ \stackrel{\epsilon~~}{\rightarrow_{\lambda}} 
	   \sbst
   		{\e_{\bot}(\texttt{t}_{1})}
   		{x_{\e_{\mathfrak{B_{2}}}(\tau_{2})}}
   		{\e_{\mathfrak{B'_{2}}}(\texttt{v}_{1})}$ & (head red.) \\
& $ = \e_{\bot}(\sbst{\texttt{t}_{1}}{x^{\mathfrak{B_{2}}}_{\tau_{2}}}{\texttt{v}_{1}}) $ & (by Substitution under erasure lemma)
	\end{tabular}
\end{center}

\noindent\textit{Case} \textsc{E-DeGhost}:  \qquad \\ Trivially verified,
as for any instance of $\tystepone{(ghost ~ \texttt{t}_{1})}{\tau_{1}}{\mathfrak{B_{1}}}$, $\mathfrak{B_{1}} = \top $.\\

\noindent\textit{Case} \textsc{E-AppLeft}: \qquad 
 $\texttt{t} = \texttt{t}_{1} \texttt{t}_{2},\quad
 \texttt{t}^{'} = \texttt{t}^{'}_{1} \texttt{t}_{2}, ~ \text{ with }
 \texttt{t}_{1} \rightarrow_{g\lambda} \texttt{t}^{'}_{1}$ \\
\phantom{\noindent\textit{Case} \textsc{E-AppLeft}: \qquad}		
$\tystepone{\texttt{t}_{1}}{\tau_{2}^{\mathfrak{B_{2}}} \rightarrow \tau_{1}}{\mathfrak{B_{1}}}$,  ~
$\tystepone{\texttt{t}_{2}}{\tau_{2}}{\mathfrak{B'_{2}}}, $ \\ 
\phantom{\noindent\textit{Case} \textsc{E-AppLeft}: \qquad}		
$ \mathfrak{B_{1}} = \bot,  \quad 
\vDash\mathfrak{B_{2}} \Leftrightarrow \mathfrak{B'_{2}} $ \\

As $\mathfrak{B_{1}} = \bot$, we can apply induction hypothesis on
$\texttt{t}_{1}$ which gives  
$\e_{\bot}(\texttt{t}_{1}) \rightarrow_{\lambda} 
\e_{\bot}(\texttt{t}_{1}^{'})$.
Then, applying \textsc{E-AppRight} rule, we obtain:
$$
\e_{\bot}(\texttt{t}) 
= \e_{\bot}(\texttt{t}_{1})\e_{\bot}(\texttt{t}_{2})
\rightarrow_{\lambda} 
\e_{\bot}(\texttt{t}^{'}_{1})\e_{\bot}(\texttt{t}_{2})
= \e_{\bot}(\texttt{t}^{'}).$$




\noindent\textit{Case} \textsc{E-AppRight}: \qquad 
 $\texttt{t} = \texttt{v}_{1} \texttt{t}_{2},\quad
 \texttt{t}^{'} = \texttt{v}_{1} \texttt{t}^{'}_{2}, ~ \text{ with }
 \texttt{t}_{2} \rightarrow_{g\lambda} \texttt{t}^{'}_{2}$ \\
\phantom{\noindent\textit{Case} \textsc{E-AppRight}: \qquad}		
$\tystepone{\texttt{v}_{1}}{\tau_{2}^{\mathfrak{B_{2}}} \rightarrow \tau_{1}}{\mathfrak{B_{1}}}$,  ~
$\tystepone{\texttt{t}_{2}}{\tau_{2}}{\mathfrak{B'_{2}}}, $ \\ 
\phantom{\noindent\textit{Case} \textsc{E-AppRight}: \qquad}		
$ \mathfrak{B_{1}} = \bot,  \quad 
\vDash\mathfrak{B_{2}} \Leftrightarrow \mathfrak{B'_{2}} $ \\
If $\mathfrak{B_{2}} = \mathfrak{B'_{2}} = \top $, then
$$\e_{\bot}(\texttt{t}) 
= \e_{\bot}(\texttt{v}_{1})\e_{\top}(\texttt{t}_{2})
= (\e_{\bot}(\texttt{v}_{1}))\texttt{()}
= \e_{\bot}(\texttt{v}_{1})\e_{\top}(\texttt{t}^{'}_{2})
= \e_{\bot}(\texttt{t}^{'}). $$
Otherwise, $\mathfrak{B_{2}} = \mathfrak{B'_{2}} = \bot $.
By induction hypothesis, 
$\e_{\bot}(\texttt{t}_{2}) \rightarrow_{\lambda} 
\e_{\bot}(\texttt{t}_{2}^{'})$.
Then, applying \textsc{E-AppRight} rule, we obtain:
$$
\e_{\bot}(\texttt{t}) 
= \e_{\bot}(\texttt{v}_{1})\e_{\bot}(\texttt{t}_{2})
\rightarrow_{\lambda} 
\e_{\bot}(\texttt{v}_{1})\e_{\bot}(\texttt{t}^{'}_{2})
= \e_{\bot}(\texttt{t}^{'}).$$




%
%
%By canonical forms lemma, $\texttt{v}_{2} = 
%\lambda x^{\mathfrak{B_{2}}}_{\tau_{2}}. \texttt{t}_{2} $.    
 
% $\texttt{t} = \texttt{E t}_{1},\quad
% \texttt{t'} = \texttt{E t'}_{1}, \quad
% \texttt{t}_{1} \rightarrow_{g\lambda} \texttt{t'}_{1}$  

\end{proof}

Now we can prove the main theorem.
\begin{theorem}[Value preservation under erasure]
	For any closed g$\lambda$-term \texttt{t} such that
	$\tystepone{\texttt{t}}{\tau}{\bot}$ holds, 
	if	$\texttt{t} \rightarrow^{\ast}_{g\lambda} \texttt{v}$ for some value \texttt{v}, then
	 \text{$\e(\texttt{t}) \rightarrow^\ast_{\lambda} \e(v)$}. 
\end{theorem}
\begin{proof} By induction on the length of the evaluation of 
$\texttt{t} \rightarrow^{\ast}_{g\lambda} \texttt{v}$. 


We already have proved the base case : indeed, if $\texttt{t} \rightarrow_{g\lambda} \texttt{v}$ then by the one-step evaluation lemma, 
$\e_{\bot}(\texttt{t}) \rightarrow^{0 | 1}_{\lambda} 
\e_{\bot}(\texttt{v})$. 

Now, assume that $\texttt{t} \rightarrow^{1}_{g\lambda} 
\texttt{t}' \rightarrow^{n}_{g\lambda}  \texttt{v}$
for some arbitrary $n \in \mathbb{N}$.
By the progress of typing, 
$ \tystepone{\texttt{t}'}{\tau}{\bot} $, so we can apply induction hypothesis on $\texttt{t}'$ which gives~
$\e(\texttt{t}') \rightarrow^\ast_{\lambda} \e(\texttt{v})$.
By the one-step evaluation lemma again, we have 
$\e_{\bot}(\texttt{t}) \rightarrow^{0 | 1}_{\lambda} 
\e_{\bot}(\texttt{t}')$. 
That is, 
$\e_{\bot}(\texttt{t}) \rightarrow^{\ast}_{\lambda} 
\e_{\bot}(\texttt{v})$.
\end{proof}





\subsubsection{Typing Erasure}
\begin{lemma}[Typing relation under erasure]~ \\
	If  $\tystepone{\texttt{t}}{\tau}{\bot}$  
	then \text{$\vdash_{\lambda}  \e_{\bot}(\texttt{t}) 
	~:~\e_{\bot}({\tau})$}.
\end{lemma}
\begin{proof}
By induction on a derivation of the statement 
\text{$\vdash_{g\lambda}  \e_{\bot}(\texttt{t}) ~:~\e_{\bot}({\tau})$}. 
For a given derivation, we proceed by case analysis on the final typing rule
 used in the proof. 
 
 	\noindent\textit{Case} \textsc{T-Unit}:\quad 
 	$\tystepone{\texttt{()}}{\texttt{unit}}{\bot}$
 	
 	Immediately by definition of $\e_{\bot}$.\\[0.08cm] 
% 		As  $\e_{\bot}(\texttt{()}) = \texttt{()}$ and $ \e_{\bot}(\texttt{unit}) = \texttt{unit}$
% 		 we have immediately $\vdash_{\lambda} \texttt{()} : \texttt{unit}$.
 	\noindent\textit{Case} \textsc{T-Var}:\quad 
 	$\tystepone{x_{\tau}^{\bot}}{\tau}{\bot}$
 	 
 	 $\e_{\bot}(x_{\tau}^{\bot}) = x_{\e_{\bot}(\tau)}$ gives  immediately 
 	 $\vdash_{\lambda} x_{\e_{\bot}(\tau)} : \e(\tau)$.\\[0.08cm] 
 	\noindent\textit{Case} \textsc{T-Abs}:\quad
 		$\tystepone{\lambda x_{\tau_{2}}^{\mathfrak{B}_{2}}.\texttt{t}_{1}}
 		{\tau_{2}^{_{2}} \rightarrow 
 		\tau_{1}}{\bot}~ \text{with} \tystepone{\texttt{t}_{1}}{\tau_{1}}{\bot}	$
 		
 		By induction hypothesis $\vdash_{\lambda} \e_{\bot}(\texttt{t}_{1}) : \e_{\bot}(\tau_{1}) $.
 		There are two cases to consider, depending on whether the parameter 
 		of the abstraction is ghost or not. If $\mathfrak{B_{2} = \top}$ then 
 		$\e_{\bot}(\lambda x_{\tau_{2}}^{\top}. \texttt{t}_{1}) = \lambda x_{\texttt{unit}}. \e(\texttt{t}_{1})$
 		and therefore
 		\infrule[T-Abs]
 		{\vdash_{\lambda} \e_{\bot}(\texttt{t}_{1}) : \e_{\bot}(\tau_{2})}
 		{ \vdash_{\lambda} \lambda x_{\texttt{unit}}.\e_{\bot}(\texttt{t}_{1}) : 
 		\texttt{unit} \rightarrow 	\e_{\bot}(\tau_{1})} 
 	
 		Otherwise $\mathfrak{B_{2} = \bot}$ and again by the rule \textsc{T-Abs} 
 		we obtain :
 		\infrule[T-Abs]
 		{\vdash_{\lambda} \e_{\bot}(\texttt{t}_{1}) : \e_{\bot}(\tau_{1})}
 		{ \vdash_{\lambda} \lambda x_{\e_{\bot}(\tau_{2})}.\e_{\bot}(\texttt{t}_{1}) : 
 		\e_{\bot}(\tau_{2}) \rightarrow 	\e_{\bot}(\tau_{1})}
 		
 		 	
 	\noindent\textit{Case} \textsc{T-App}:\quad
 	$\tystepone{\texttt{t}_{1}~\texttt{t}_{2}}{\tau_{1}}{\bot} $ ~
 	with sub-derivations: \\
	\phantom{{Case} \textsc{T-App}: \qquad}
	$\tystepone{\texttt{t}_{1}}
	{\tau_{2}^{\mathfrak{B}_{2}} \rightarrow \tau_{1}}{\mathfrak{B}_{1}}$ \\
	\phantom{\textit{Case} \textsc{T-App}: \qquad}	
	$ \tystepone{\texttt{t}_{2}}{\tau_{2}}{\mathfrak{B'}_{2}} $, \quad 
%	and constraints: \\ 
%	\phantom{\textit{Case} \textsc{T-App}: \qquad}	
%	$ \mathfrak{B}_{1}\vee (\neg \mathfrak{B}_{2} \wedge \mathfrak{B'}_{2})=\bot,$
%	$ \mathfrak{B}_{2} \Rightarrow \mathfrak{B'}_{2} = \top $ \\
	
%	By lemma's statement, $\texttt{t}_{1}~\texttt{t}_{2}$ 
%	should not be a ghost term. Therefore
%	$\mathfrak{B}_{1}\vee (\neg \mathfrak{B}_{2} \wedge \mathfrak{B'}_{2})=\bot$.
%	From that and from the rule's premise condition 
%	$ \mathfrak{B}_{2} \Rightarrow \mathfrak{B'}_{2} = \top $ 
%	we deduce 

As $\tystepone{\texttt{t}_{1}~\texttt{t}_{2}}{\tau_{1}}{\bot} $, the inversion lemma gives as 
By inversion that $\vDash \mathfrak{B}_{2} \Leftrightarrow \mathfrak{B'}_{2}$. That is, we have two cases to consider.
	
	If \text{$ \mathfrak{B}_{2} = \mathfrak{B'}_{2} = \bot $} 
	then by induction hypotheses \\
	$\vdash_{\lambda} \e_{\bot}(\texttt{t}_{1}) : 
	\e_{\bot}(\tau_{2}) \rightarrow \e_{\bot}(\tau_{1}) $ and
	\text{$\vdash_{\lambda} \e_{\bot}(\texttt{t}_{2}) : \e_{\bot}(\tau_{2}) $}.
	By  \textsc{T-App} rule,
	\infrule[T-App]
	{\vdash_{\lambda} \e_{\bot}(\texttt{t}_{1}) : 
	\e_{\bot}(\tau_{2}) \rightarrow \e_{\bot}(\tau_{1}) 
	\qquad
	\vdash_{\lambda} \e_{\bot}(\texttt{t}_{2}) : \e_{\bot}(\tau_{2}) }
	{\vdash_{\lambda} \e_{\bot}(\texttt{t}_{1}~\texttt{t}_{2}): \e(\tau_{1})}  
	
	If \text{$ \mathfrak{B}_{2} = \mathfrak{B'}_{2} = \top $},
	then by definition of $\e$ we have $\e(\texttt{t}_{2}) = \texttt{()}$ 
	and $\e_{\mathfrak{B'_{2}}}(\tau_{2}) = \texttt{unit}$. 
	By induction hypothesis on $\texttt{t}_{1}$,
	\text{$\vdash_{\lambda} \e_{\bot}(\texttt{t}_{1}) : 
	\texttt{unit} \rightarrow \e_{\bot}(\tau_{1}) $}. 
	Applying \textsc{T-App} rule gives us \\

\hspace*{1.2in}
\infer[\mbox{\textsc{(T-App)}}]
 {\vdash_{\lambda} \e_{\bot}(\texttt{t}_{1}~\texttt{t}_{2}): \e_{\bot}(\tau_{1})}
 {\infer
   {\vdash_{\lambda} \e_{\bot}(\texttt{t}_{1}~\texttt{t}_{2}): \e(\tau_{1})}
   {}
   &
 \infer[\mbox{\textsc{(T-Unit)}}]
  {\vdash_{\lambda} \texttt{()} : \texttt{unit}}
  {}}	\\
  
  	The case of (\textsc{T-Ghost}) as well as any other valid derivation  
  	where a typed term is marked as ghost do not satisfy lemma's requirement, 
  	so these cases are trivially verified.									
\end{proof}

%TODO:
%Extensions:
%- If-then-else
%- rec
%- let in
%- match 
%- constructors
%- ref global mono
%- types polymorphes
%- operators
%
%Potential difficulties:
%- polymorphism
%- schemes
%- non interference
%
%-->(E,E')
%
%règles de typage
%règles d'effacement




%\begin{adjustwidth}{-22em}{-2em}
%\begin{spacing}{1.4}
%
%
%\infax[E-AppFun] {\stepone{(\lambda x^{b}_{\tau}.\texttt{t})v}{\sbst{\texttt{t}}{x^{b}_{\tau}}{v}}} 
%
%\infax[E-DeGhost]	 	{\stepone{\texttt{ghost t}}{\texttt{t}}}
%   \end{minipage}
%\hphantom{....}
%
%   \begin{minipage}{0.5\linewidth}
%  \infrule[T-AppLeft]{{\texttt{t}_{1}} \rightarrow {\texttt{t}'_{1}}}{{\texttt{t}_{1} \texttt{t}_{2} } \rightarrow {\texttt{t}'_{1} \texttt{t}_{2}}}
%  
%\infrule[T-AppRight]{{\texttt{t}_{2}} \rightarrow {\texttt{t}'_{2}}}{{\texttt{v}_{1} \texttt{t}_{2} } \rightarrow {\texttt{v}_{1} \texttt{t}'_{2}}}  
%
%    
%   \end{minipage}
%   
%\end{spacing}
%\end{adjustwidth}



%  	 \begin{flushright}  	 \textbf{Evaluation}  	 \end{flushright}
%  	\infax[E-AppFun]    	{\stepone{(\lambda x^{b}_{\tau}.\texttt{t})v}{\sbst{\texttt{t}}{x^{b}_{\tau}}{v}}}
%	\infax[E-DeGhost]	 	{\stepone{\texttt{ghost t}}{\texttt{t}}}
%%	\infrule[E-Context]{{\texttt{t}} \rightarrow {\texttt{t}'}}{{\texttt{E[t]}} \rightarrow {\texttt{E[t']}}}
%%	\begin{center}
%% 	 where ~ \texttt{E}   $::= \square$  | \texttt{E}~\texttt{t} |	 \texttt{v} \texttt{E}
%% 	 \end{center}	 
%	\infrule[T-AppLeft]{{\texttt{t}_{1}} \rightarrow {\texttt{t}'_{1}}}{{\texttt{t}_{1} \texttt{t}_{2} } \rightarrow {\texttt{t}'_{1} \texttt{t}_{2}}}
%	\infrule[T-AppLeft]{{\texttt{t}_{2}} \rightarrow {\texttt{t}'_{2}}}{{\texttt{v}_{1} \texttt{t}_{2} } \rightarrow {\texttt{v}_{1} \texttt{t}'_{2}}}	
%\caption{ \textbf{\glam syntax and semantics} \hfill}

%
%\subsubsection*{Free variables, scope and equivalence of terms} 
%We do not present here the formal definitions of free variables, 
%abstraction scope, $\alpha$-equivalence and substitution: they are exactly 
%the same that those of $\lambda$-calculus.
%
%However, it is important to notice that two variables $x^{\mathfrak{B_{1}}}_{\tau_{1}}$ and  
%$y^{\mathfrak{B_{2}}}_{\tau_{2}}$ are syntactically equal only if 
%$x = y$, $\mathfrak{B_{1}} = \mathfrak{B_{2}}$ and $ \tau_{1} = \tau_{2} $. That is  $\lambda \gvarT{x}{1}. \gvarT{x}{1}$ corresponds to $\tau_{1}$- identity, whilst  $\lambda \gvarT{x}{1}. \gvarF{x}{1}$ corresponds to the constant function that ignores its argument. 
%
%
%  	 \begin{flushright}  	 \textbf{Typing}  	 \end{flushright}
%	\infrule[T-Unit]	 	{}{\tystepone{()}{unit}{\bot}} 
%	
%	\infrule[T-Var]	 	{}{\tystepone{x^{\mathfrak{B}}_{\tau}}{\tau}{\mathfrak{B}}}
%	
%	\infrule[T-Ghost]	 {\tystepone{\texttt{t}}{\tau}{\mathfrak{B}}}{\tystepone{\texttt{ghost t}}{\tau}{\top}}
%	
%	\infrule[T-Abs]	 	 {\tystepone{\texttt{t}}{\tau_{2}}{\mathfrak{B}_{2}}}
%										{\tystepone{\lambda x_{\tau_{1}}^{\mathfrak{B}_{1}}.\texttt{t}}{\tau_{1}^{\mathfrak{B}_{1}} \rightarrow \tau_{2}}{\mathfrak{B}_{2}}}
%										
%	\infrule[T-App]	 
%	{\tystepone{\texttt{t}_{1}}{\tau_{2}^{\mathfrak{B}_{2}} \rightarrow \tau_{1}}{\mathfrak{B}_{1}} 
%	\qquad  {\tystepone{\texttt{t}_{2}}{\tau_{2}}{\mathfrak{B'}_{2}}}
%	\qquad \mathfrak{B}_{2} \Rightarrow \mathfrak{B'}_{2}}
%	{\tystepone{\texttt{t}_{1}~ \texttt{t}_{2} }{\tau_{1}}{\mathfrak{B}_{1} \vee (\neg \mathfrak{B}_{2} \wedge \mathfrak{B'}_{2} ) }}

%In the simply typed $\lambda$-calculus, typing rules out  statically some erroneous programs. For instance, it allows to check that when function is applied to some argument, the type of the formal parameter corresponds to that one of actual parameter.
%
%The typing of the programs containing some ghost code should of course 
%assure the same safety with respect of program evaluation. However, in the presence of ghost code, this should also include the \textit{non-interference} of ghost code with the rest of the program. Intuitively, the \textit{non-interference} means that computation of ghost code must be completely disjoint from the computation of the non-ghost code. 
%
%To assure that, we define the typing relation as three-argument predicate  
%\fbox{${}{\tystepone{t}{\tau}{\mathfrak{B}}} $} which both carries statically the information about type of term $t$, and indicates whether
%this term is ghost or not. Formally, we define the typing relation inductively over the structure of terms as follows:
\newpage
\section{logic} 


%%%%%%%%%%%%%%%%%%%%%%%%%%%%%%%%%%%%%%%%%%%%%%%%%%%%%%%%%%%%%%%%%%%%%%%%%%%%%%



 In previous section we described how to define formally ghost terms inside
 ML-like programs. 
 
 However, the ghost code becomes useful only within a full-featured 
 specification language.
 
 In this section we define a high-order logic with simple types such as units, 
 integers, boolean. To make our examples more interesting we also provide 
 built-in integer lists and trees.
 
 First off, we describe our logic's syntax, semantics and typing. Then
 we extend the ghost-ML language with specifications such as assertions
 and functional pre- and post-conditions.
 
	\begin{figure}[H]
		\begin{adjustwidth}{-7em}{-3em}
		\begin{displaymath}
		\hrulefill
		\begin{array}{ll@{\hspace*{-0.1cm}}
							 l@{\hspace*{1.5cm} }
							 l|@{\hspace*{0.5cm}}
							 llll}
	\hline		

  \tau & ::= 
  & & \textsc{types} &  
  
  c & ::= 
	& & \textsc{constants} \\



	& \text{unit }| \text{ int }| \text{ bool} 
	& & \textit{built-in simple types} &
	
	&     0 ~|~ 1 ~|~ ... ~|~ n        
	& & \textit{integer} \\    
	
	
	
	& \text{list int }	| \text{ tree int} 
	& & \textit{built-in recursive types} &
	
	&   \mathtt{true}~|~\mathtt{false} 
  & & \textit{boolean} \\ 
	
	
	
	& \tau \rightarrow \tau 
	& & \textit{function type} &

	&  () 
	& & \textit{unit} \\  	
	
	

  & \mathtt{prop} 
  & & \textit{proposition} & 

  & \mu \mathtt{l}. \texttt{ Nil | Cons n l }
  & &  \textit{integer list}  \\ 
  
  

	& \ 
	& & \ &
	
	& \mu \mathtt{t}. \texttt{ Empty | Node t n t }
  & &  \textit{integer binary tree}  \\ 
	
	

	t & ::= 
	& & \textsc{terms} & 
	
	& \mathtt{True} ~|~ \mathtt{False} 
	& & \textit{proposition value} \\



	& c
	& & \textit{constant} &
	
	&  + ~|~ - ~|~ = | ... 
	& & \textit{built-in operators} \\



	& \lgvar{x}{\tau}
	& & \textit{variable}
	
	& \ \\


	
	& \lgabs{x}{\tau}{t}
	& & \textit{abstraction}  &

  f & ::= 
	& & \textsc{formulas} \\

	
	
	& \lgapp{t}{t}
	& & \textit{application} &
	
	& True ~|~ False 
	& & \textit{logical truth values} \\
	
	
	
	& \lglet{x}{\tau}{t}{t}
	& & \textit{local binding} &
	
	& \lgand{t}{t} 
	& & \textit{conjunction}\\
	
	
	
	& \lgrec{g}{\tau}{x}{\tau}{t}
	& & \textit{recursive function} &
	
	& \lgor{t}{t} 
	& & \textit{disjunction} \\
	
		
		
	& f
	& & \textit{formula} &
	
	& \neg t
	& & \textit{negation} \\		



	& \ 
	& & \ &
	
	&  \lgexist{x}{\tau}{f} 
	& & \textit{existential quantification} \\ 



	& \ 
	& & \ &
	
	&  \lgforall{x}{\tau}{f} 
	& & \textit{universal quantification} \\ 
	
	
	
	& \ 
	& & \ &
	
	&  f = f
	& & \textit{equality} \\ 

	\hline			   
  	\end{array} 
		\end{displaymath} 
 		\caption{Logic Syntax}
 		\end{adjustwidth}
	\end{figure}
 
\bibliographystyle{plain}
\bibliography{abbrevs,demons,demons2,demons3,team,crossrefs,crossrefs2,./biblio}
\end{document}
