\newpage
\section{logic} 


%%%%%%%%%%%%%%%%%%%%%%%%%%%%%%%%%%%%%%%%%%%%%%%%%%%%%%%%%%%%%%%%%%%%%%%%%%%%%%



 In previous section we described how to define formally ghost terms inside
 ML-like programs. 
 
 However, the ghost code becomes useful only within a full-featured 
 specification language.
 
 In this section we define a high-order logic with simple types such as units, 
 integers, boolean. To make our examples more interesting we also provide 
 built-in integer lists and trees.
 
 First off, we describe our logic's syntax, semantics and typing. Then
 we extend the ghost-ML language with specifications such as assertions
 and functional pre- and post-conditions.
 
	\begin{figure}[H]
		\begin{adjustwidth}{-7em}{-3em}
		\begin{displaymath}
		\hrulefill
		\begin{array}{ll@{\hspace*{-0.1cm}}
							 l@{\hspace*{1.5cm} }
							 l|@{\hspace*{0.5cm}}
							 llll}
	\hline		

  \tau & ::= 
  & & \textsc{types} &  
  
  c & ::= 
	& & \textsc{constants} \\



	& \text{unit }| \text{ int }| \text{ bool} 
	& & \textit{built-in simple types} &
	
	&     0 ~|~ 1 ~|~ ... ~|~ n        
	& & \textit{integer} \\    
	
	
	
	& \text{list int }	| \text{ tree int} 
	& & \textit{built-in recursive types} &
	
	&   \mathtt{true}~|~\mathtt{false} 
  & & \textit{boolean} \\ 
	
	
	
	& \tau \rightarrow \tau 
	& & \textit{function type} &

	&  () 
	& & \textit{unit} \\  	
	
	

  & \mathtt{prop} 
  & & \textit{proposition} & 

  & \mu \mathtt{l}. \texttt{ Nil | Cons n l }
  & &  \textit{integer list}  \\ 
  
  

	& \ 
	& & \ &
	
	& \mu \mathtt{t}. \texttt{ Empty | Node t n t }
  & &  \textit{integer binary tree}  \\ 
	
	

	t & ::= 
	& & \textsc{terms} & 
	
	& \mathtt{True} ~|~ \mathtt{False} 
	& & \textit{proposition value} \\



	& c
	& & \textit{constant} &
	
	&  + ~|~ - ~|~ = | ... 
	& & \textit{built-in operators} \\



	& \lgvar{x}{\tau}
	& & \textit{variable}
	
	& \ \\


	
	& \lgabs{x}{\tau}{t}
	& & \textit{abstraction}  &

  f & ::= 
	& & \textsc{formulas} \\

	
	
	& \lgapp{t}{t}
	& & \textit{application} &
	
	& True ~|~ False 
	& & \textit{logical truth values} \\
	
	
	
	& \lglet{x}{\tau}{t}{t}
	& & \textit{local binding} &
	
	& \lgand{t}{t} 
	& & \textit{conjunction}\\
	
	
	
	& \lgrec{g}{\tau}{x}{\tau}{t}
	& & \textit{recursive function} &
	
	& \lgor{t}{t} 
	& & \textit{disjunction} \\
	
		
		
	& f
	& & \textit{formula} &
	
	& \neg t
	& & \textit{negation} \\		



	& \ 
	& & \ &
	
	&  \lgexist{x}{\tau}{f} 
	& & \textit{existential quantification} \\ 



	& \ 
	& & \ &
	
	&  \lgforall{x}{\tau}{f} 
	& & \textit{universal quantification} \\ 
	
	
	
	& \ 
	& & \ &
	
	&  f = f
	& & \textit{equality} \\ 

	\hline			   
  	\end{array} 
		\end{displaymath} 
 		\caption{Logic Syntax}
 		\end{adjustwidth}
	\end{figure}
	
\newpage
\newcommand{\paragraphred}[1]{\paragraph{\textcolor{red}{#1}}}
\section{Inlining}
	
\paragraphred{Motivation:}  
instantiate higher-order iterators with previously defined or anonymous functions, in order to obtain a first-order function, whose proof obligation is of the same complexity as p.o. of equivalent loop statement. 

\paragraphred{Goal:} 
\quad reduce every application of a higher-order function to a term that
is not a bound variable.

\paragraphred{Input:}  
a higher-order language in \textit{A-normal} form with functions whose codomain is of some base type, and whose formal parameters is partially ordered (the higher order parameters come before lesser order parameters)   
 
\paragraphred{Output:}  a language where the only high-order applications are those where
argument of application is a bound variable. That is, a language where
higher-order applications can occur only under \textbf{inside} high-order functions. \\

\noindent\textbf{Input Language Syntax:} \\

\begin{figure}[H]
\hrulefill
\begin{displaymath}
\begin{array}{lll}
 t & ::= & v ~ | ~ vv ~ | let 
\end{array}
\end{displaymath}
\caption{Source language in A-normal form}
\hrulefill
\end{figure}


 


 