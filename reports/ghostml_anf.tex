%%%%%%%%%%%%%%%%%%%%%%%%%%%%%%%%%%%%%%%%%%%%%%%%%%%%%%%%%%%%%%%%%%%%%%%%%%%%%%%%
\documentclass[a4paper,11pt,oneside]{article}
\label{packages}
\usepackage[sc]{mathpazo} \linespread{1.0}
\usepackage[scaled]{helvet} % ss
\usepackage[T1]{fontenc}
\usepackage[francais]{babel}
\usepackage[a4paper,pdftex,colorlinks=true,
	urlcolor=blue,pdfstartview=FitH]{hyperref}
\usepackage{graphicx}
\usepackage{amssymb,amsfonts,amsthm,amsmath}
\usepackage{titlesec}
\usepackage{listings}
\usepackage[usenames,dvipsnames]{xcolor}
\usepackage[dvipsnames]{xcolor}
\usepackage{caption} 
\protect\usepackage{semantic}
\usepackage{bcprules}
\usepackage{stmaryrd}
\titleformat{\section}[hang]% style du titre
  {\normalfont\Large\bfseries}% police du titre + numéro
  {\thesection}% numérotation
  {0.4in}% espacement numéro/titre <--- valeur à changer
  {}% police spécifique du titre
\titlespacing*{\section}
  {-0.5in}% espacement gauche
  {0.2in}% espacement avant
  {0.2in}% espacement après
  [0.1in]% espacement à droite

\titleformat{\subsection}[hang]% style du titre
  {\normalfont\large\bfseries}% police du titre + numéro
  {\thesubsection}% numérotation
  {0.2in}% espacement numéro/titre <--- valeur à changer
  {}% police spécifique du titre
\titlespacing*{\subsection}
  {-0.44in}% espacement gauche
  {0.2in}% espacement avant
  {0.2in}% espacement après
  [0.1in]% espacement à droite


%\titleformat{\subsubsection}[hang]% style du titre
%  {\normalfont\large\bfseries}% police du titre + numéro
%  {\thesubsubsection}% numérotation
%  {0.1in}% espacement numéro/titre <--- valeur à changer
%  {}% police spécifique du titre
%\titlespacing*{\subsubsection}
%  {-0.30in}% espacement gauche
%  {0.2in}% espacement avant
%  {0.2in}% espacement après
%  [0.1in]% espacement à droite

\RequirePackage{listings}
\RequirePackage{amssymb}

\lstset{basicstyle={\ttfamily}}

\lstdefinelanguage{why3}
{
morekeywords=[1]{},%
%keywordstyle=[1]{\color{blue}},%
string=[b]",%
showstringspaces=false,%
stringstyle=\itshape,%
commentstyle= \color{gray}\small\texttt, %
columns=[l]fullflexible,%
sensitive=true,%
morecomment=[s]{(*}{*)},%
escapeinside={*?}{?*},%
keepspaces=true,
literate=%
%{'a}{$\alpha$}{1}%
%{'b}{$\beta$}{1}%
%{<}{$<$}{1}%
%{>}{$>$}{1}%
%{<=}{$\le$}{1}%
%{>=}{$\ge$}{1}%
% {<>}{$\ne$}{1}%
% {/\\}{$\land$}{1}%
% {\\/}{ $\lor$ }{3}%
% {\ or(}{ $\lor$(}{3}%
% {not\ }{$\lnot$ }{1}%
% {not(}{$\lnot$(}{1}%
% {+->}{\texttt{+->}}{2}%
% % {+->}{$\mapsto$}{2}%
% {-->}{\texttt{-\relax->}}{2}%
% %{-->}{$\longrightarrow$}{2}%
% {->}{$\rightarrow$}{2}%
% {<->}{$\leftrightarrow$}{2}%
%
%
}

\colorlet{notgreen}{ForestGreen!15}

\definecolor{cblue}{RGB}{2,4,8}
\lstset{language=why3,
    %%% BOUCLE, TEST & Co.
      emph={add, double, edges, vertices, merge,
      len,  nand, even, ackermann},
      emphstyle=\color{
black}\bf,
    %%% IMPORT & Co.
      emph=[2]{namespace,predicate,function,inductive,type,use,clone,%
import,export,theory,end,in,match,with,%
goal,axiom,lemma,forall},
      emphstyle=[2]\color{
NavyBlue},
 		emph=[3]{nat, list_nat, tree_nat, int, bool},
      emphstyle=[3]\color{
ForestGreen},
	emph=[4]{S, Cons, O, True, False, true, false, Leaf, Node},
      emphstyle=[4]\color{
cblue}
%    %%% MODULES
%      emph={[3]mod_1, mod_2},
%      emphstyle=[3]\color{red!50!green},
%    %%% FONCTIONS NUMERIQUES
%      emph={[4]cos, sin, tan, acos, asin, atan},
%      emphstyle=[4]\color{gray},
%    %%% CONSTANTES
%      emph={[5]pi, True, False},
%      emphstyle=[5]\color{orange}
}


% default language is Why3
\lstset{language=why3}
\lstnewenvironment{why3}{\lstset{language=why3}}{}
\let\why\lstinline
\newcommand{\HRule}{\rule{\linewidth}{0.5mm}}



 \begin{document}

\vspace*{-5cm}
\begin{small}
\section*{{\normalsize \begin{center}GhostML : a mini-ML with global references and ghost terms\end{center}}}
%%%%%%%%%%%%%%%%%%%%%%%%%%%%%%%%%%%%%%%%%%%%%%%%%%%%%%%%%%%%%%%%%%%%%%%%%%%%%%%%



\label{GhostML syntax ----------------------------------------}
\label{GhostML  programs}
\begin{figure}[!ht]\vspace*{-1cm}
\begin{displaymath}
  \begin{array}{lll@{\hspace*{2.5cm} }l}
     prog       	& ::= &		typedecl^{*} \quad vardecl^{*}  \quad  t	 			&							\text{program} \\
     typedecl 	& ::= &	\texttt{type} ~ id ~ \alpha, ..., \alpha ~ = \tau 		&  						\text{type declaration} \\
     vardecl   	& ::= &	\texttt{val} ~ id : \texttt{ref} ~ \tau							 &  						\text{global reference declaration}\\										
  \end{array} 
\end{displaymath} \vspace*{-0.5cm}
\caption*{ \textbf{GhostML Programs} \hfill}
\end{figure}

\label{GhostML  types and schemes}
\begin{figure}[!h]
\begin{displaymath}
  \begin{array}{rrl@{\hspace*{4.5cm}}l}	
    \tau 		& ::= & \alpha 			&																	\text{type variable} \hspace*{5cm} \\
      				&   | & \varepsilon ~ (\tau,\dots, \tau)	 &							\text{datatype constructor} \\
       				&   | & \tau \rightarrow \tau &												\text{function type} \\
       				&   | & ~\texttt{int} ~|~ \texttt{bool} ~|~ \texttt{Prop} ~|~ ...	 &							\text{build-in types} \\
     \sigma & ::= & \forall \overline{\alpha} . \tau		&												\text{type scheme} \hspace*{5cm} \\   
  \end{array} 
\end{displaymath} \vspace*{-0.5cm}
\caption*{ \textbf{GhostML Types and Schemes} \hfill}
\end{figure}
\label{GhostML  values}
\begin{figure}[!h]\vspace*{-0.5cm}
\begin{displaymath}
  \begin{array}{rrl@{\hspace*{4.5cm} }l}
   v & ::= &		   x						&										\text{variable} \\
      &   | & 		op					& 									\text{build-in constants and operands \ ( \textit{1, true, +}, $\vee$, ...)} \\
      &   | & 		C(v,\dots, v) & 									\text{constructor application  }  \\
      &   | & 		(\texttt{fun} \ \varepsilon x \rightarrow t) & 	\text{function }  \\
  \end{array} 
\end{displaymath} \vspace*{-0.5cm}
\caption*{ \textbf{GhostML Values} \hfill}
\end{figure}
\label{GhostML  terms}
\begin{figure}[!h]\vspace*{-0.5cm}
\begin{displaymath}
  \begin{array}{rrl@{\quad\quad\quad}l}
   t & ::=  & v &																											\text{value} \\
      &   | & v (v) &			 																		                  \text{application} \\
      &   | & \texttt{let}~x = t \ \texttt{in} \ t	&													\text{local binding} \\
      &   | & \texttt{letrec}~ f~ x  = t & 																		\text{recursive function} \\
      &   | & !~x & 																										\text{global reference access} \\
      &   | & x := t & 																									\text{global reference assignment} \\
      &   | & \mathsf{if}~t~\mathsf{then}~t~\mathsf{else}~t & 						\text{conditional expression} \\
      &   | & \mathsf{match} ~ t ~ \mathsf{with} ~
         p \rightarrow t, ~\dots,~p \rightarrow t  ~ \mathsf{end}& 				    \text{pattern-matching} \\
  \end{array} 
\end{displaymath} \vspace*{-0.5cm}
\caption*{ \textbf{GhostML Terms} \hfill}
\end{figure}
\label{GhostML  patterns}
\begin{figure}[!h]
\begin{displaymath}
  \begin{array}{rrl@{\hspace*{4.5cm}}l}
    p & ::= & x 			&								\text{variable pattern} \hspace*{5cm} \\
      &   | & C(p,\dots,p)	 &						\text{constructor pattern} \\
  \end{array} 
\end{displaymath} \vspace*{-0.5cm}
\caption*{ \textbf{GhostML Patterns} \hfill}
\end{figure}
\end{small}


\label{GhostML operational semantics ------------------}
\newpage
\section*{{\normalsize \begin{center} Operational  Semantics \end{center}}}

\definecolor{gray}{gray}{0.27}
\newcommand{\stepone}[4]{#1{\color{gray}_{|#3}}   ~~\leadsto~~ #2{\color{gray}_{|#4}}}
\newcommand{\sbst}[3]{#1 [#2 \mapsfrom #3] }
\reservestyle{\command}{\textbf}
\command{let[\;let\;], fun[fun\;],  ghost[ghost\;],  in[\;in\;],  if[\;if\;],  then[\;then\;],  else[\;else\;]}
\typicallabel{E-IF-False-Redex}

 \infax[E-App-Redex]
	{\stepone{(\<fun> \varepsilon x_{\tau} \rightarrow t)v}{\sbst{t}{x}{v}}{\sigma}{\sigma}}
	
 \infax[E-Let-Redex]
	{\stepone{\<let> \varepsilon x = v_{1} \<in> t}{\sbst{t}{x}{v}}{\sigma}{\sigma}}	
	
  \infax[E-IF-True-Redex]
  {\stepone{\<if> \mathtt{true} \<then> t_{1} \<else> t_{2}} {t_{1}}{\sigma}{\sigma}}
	
 \infax[E-IF-False-Redex]
  {\stepone{\<if> \mathtt{false} \<then> t_{1} \<else> t_{2}} {t_{2}}{\sigma}{\sigma}}
	
 \infrule[E-IF-Context]
 {\stepone{t_{1}}{t_{1}'}{\sigma}{\sigma'}}
  {\stepone{\<if> t_{1}\<then> t_{2} \<else> t_{3}}{\<if> t_{1}'\<then> t_{2} \<else> t_{3}}{\sigma}{\sigma'}}	
	
  \infrule[E-Let-Context]
   	{\stepone{t_{1}}{t_{1}'}{\sigma}{\sigma'}}
    {\stepone{\<let> \varepsilon x = t_{1} \<in> t_{2}}{\<let> \varepsilon x = t_{1} \<in>  t_{2}'}{\sigma}{\sigma'}}
   	
 

\label{E-Local}
\label{E-Ref}
\label{E-Cases}
\bibliographystyle{plain}
\bibliography{abbrevs,demons,demons2,demons3,team,crossrefs,crossrefs2,./biblio}

\end{document}
