%%%%%%%%%%%%%%%%%%%%%%%%%%%%%%%%%%%%%%%%%%%%%%%%%%%%%%%%%%%%%%%%%%%%%%%%%%%%%%%%
\documentclass[a4paper,11pt,oneside]{article}
\pagestyle{headings}
\label{packages}
\usepackage[sc]{mathpazo}
\usepackage[scaled]{helvet} % ss
\usepackage[utf8]{inputenc}
\usepackage[T1]{fontenc}
\usepackage[english]{babel}
\usepackage[a4paper,pdftex,colorlinks=true,
	urlcolor=blue,pdfstartview=FitH]{hyperref}
\usepackage{graphicx}
\usepackage{amssymb,amsfonts,amsthm,amsmath}
\usepackage{titlesec}
\usepackage{listings}
\usepackage[usenames,dvipsnames]{xcolor}
\usepackage[dvipsnames]{xcolor,colortbl}
\usepackage{caption}
\protect\usepackage{semantic}
\usepackage{bcprules, proof}
\usepackage{stmaryrd}
 \usepackage{float}
\usepackage{changepage}
\usepackage{setspace}
\usepackage{pstricks,pstricks-add,pst-math,pst-xkey}
\usepackage{multicol}
\usepackage[small,nohug,heads=vee]{diagrams}
\usepackage{cleveref}
\usepackage{xargs}
\crefname{enumi}{position}{positions}
\diagramstyle[labelstyle=\scriptstyle]
\titleformat{\section}[hang]% style du titre
  {\normalfont\LARGE\bfseries}% police du titre + numéro
  {\thesection}% numérotation
  {0.4in}% espacement numéro/titre <--- valeur à changer
  {}% police spécifique du titre
\titlespacing*{\section}
  {-0.9in}% espacement gauche
  {0.2in}% espacement avant
  {0.2in}% espacement après
  [0.1in]% espacement à droite

\titleformat{\subsection}[hang]% style du titre
  {\normalfont\Large\bfseries}% police du titre + numéro
  {\thesubsection}% numérotation
  {0.2in}% espacement numéro/titre <--- valeur à changer
  {}% police spécifique du titre
\titlespacing*{\subsection}
  {-0.7in}% espacement gauche
  {0.2in}% espacement avant
  {0.2in}% espacement après
  [0.1in]% espacement à droite


\titleformat{\subsubsection}[hang]% style du titre
  {\normalfont\large\bfseries}% police du titre + numéro
  {\thesubsubsection}% numérotation
  {0.1in}% espacement numéro/titre <--- valeur à changer
  {}% police spécifique du titre
\titlespacing*{\subsubsection}
  {-0.6in}% espacement gauche
  {0.2in}% espacement avant
  {0.2in}% espacement après
  [0.1in]% espacement à droite


\RequirePackage{listings}
\RequirePackage{amssymb}

\lstset{
  basicstyle={\ttfamily},
  framesep=2pt,
  frame=single,
  keywordstyle={\color{blue}},
  stringstyle=\itshape,
  commentstyle=\itshape,
  columns=[l]fullflexible,
  showstringspaces=false,
}

\lstdefinelanguage{why3}
{
morekeywords={namespace,predicate,function,inductive,type,use,clone,%
import,export,theory,module,end,in,with,%
let,rec,for,to,do,done,match,if,then,else,while,try,invariant,variant,%
absurd,raise,assert,exception,private,abstract,mutable,ghost,%
downto,raises,writes,reads,requires,ensures,returns,val,model,%
goal,axiom,lemma,forall},%
string=[b]",%
sensitive=true,%
morecomment=[s]{(*}{*)},%
keepspaces=true,
}
%literate=%
%{'a}{$\alpha$}{1}%
%{'b}{$\beta$}{1}%
%{<}{$<$}{1}%
%{>}{$>$}{1}%
%{<=}{$\le$}{1}%
%{>=}{$\ge$}{1}%
% {<>}{$\ne$}{1}%
% {/\\}{$\land$}{1}%
% {\\/}{ $\lor$ }{3}%
% {\ or(}{ $\lor$(}{3}%
% {not\ }{$\lnot$ }{1}%
% {not(}{$\lnot$(}{1}%
% {+->}{\texttt{+->}}{2}%
% % {+->}{$\mapsto$}{2}%
% {-->}{\texttt{-\relax->}}{2}%
% %{-->}{$\longrightarrow$}{2}%
% {->}{$\rightarrow$}{2}%
% {<->}{$\leftrightarrow$}{2}%

\lstnewenvironment{whycode}{\lstset{language=why3}}{}
\lstnewenvironment{ocamlcode}{\lstset{language={[Objective]Caml}}}{}

%\lstset{basicstyle={\ttfamily}}
\let\why\lstinline
\newcommand{\HRule}{\rule{\linewidth}{0.5mm}}


\label{proclamations}
\newtheoremstyle{plain}
{\topsep}{\topsep}{\upshape}{}{}{:~}{ }
{\textsc{\hspace{-1.55cm} #2 \quad #1} \textsc{#3}}
\theoremstyle{plain}
\newtheorem{definition}{Definition}[subsection]
\newtheorem{lemma}[definition]{Lemma}
\newtheorem{theorem}[definition]{Theorem}
\newtheorem{corr}[definition]{Corrolary}


\label{ML-Terms marcos}
\newcommand{\mlt}[1]{#1}
\newcommand{\tmapp}[2]{(#1 ~ #2)}

\label{ML-Types marcros}
\newcommand{\ty}[1][]{\tau_{#1}}
\newcommandx{\tyarr}[4][1=1, 2=2, 3=\theta, 4=\rho]
	{\tau_{#1}\hspace*{-0.2cm}\stackrel{(#3,#4)}
	{\Rightarrow}\hspace*{-0.2cm}\tau_{#2}}
\newcommandx{\tyord}[2][2=]{\ty[#2]^{#1}}

%
%\newcommand{\tarrS}[4]
%	{\tau^{\mf{B}_{#3}}_{#1}
%	\stackrel{\Sigma_#4}{\longrightarrow} \tau_{#2}}


\newcommand{\bwedge}{\boldsymbol{\wedge}}
\newcommand{\bvee}{\boldsymbol{\vee}}
\label{ML-Typing marcos}
\newcommandx{\typerule}[5]{~\vdash  #1 : (#2, #3, #4) #5}

\label{Semantics marcos}
\newcommand{\evalstep}[4]{~#1_{|\mu_#2} \rightarrow #3_{|\mu_#4} ~}
\newcommand{\evalstar}[4]{~#1_{|\mu_#2} \rightarrow^{\star} #3_{|\mu_#4} ~}
\newcommand{\evalinfty}[2]{~#1_{|#2} \rightarrow \infty ~}
\newcommand{\eqv}[1]{#1 \thicksim #1'}
\newcommand{\eqvsbst}[2]{#1 \thicksim_{#2} #1'}

\let\mf\mathfrak


\addtolength{\textwidth}{1.5cm}

\def\nrepeat#1#2{\count0=#1 \loop \ifnum\count0>0 \advance\count0 by -1 #2\repeat}

%\swapnumbers




\newcommand{\mem}{_{|\mu}}\newcommand{\memp}{_{|\mu'}}

%%%%%%%%%%%%%%%%%%%%%%%%%%%%%%%%%%%%%%%%%%%%%%%%%%%%%%%%%%%%%%%%%%%%%
\newcommand{\glam}{\textit{ghost}-$\lambda$~}
\newcommand{\gml}{\textit{ghost}-ml~}
%%%%%%%%%%%%%%%%%%%%%%%%%%%%%%%%%%%%%%%%%%%%%%%%%%%%%%%%%%%%%%%%%%%%%

\newcommand{\var}[3]{#1^{#2}_{#3}}
\newcommand{\gvar}[3]{#1^{\mathfrak{B_{#2}}}_{\tau_{#3}}}
\newcommand{\gref}[3]
{#1^{\mathfrak{B_{#2}}}_{\mathtt{ref}~\tau_{#3}}}
\let\rvar\gref
\newcommand{\gvarT}[2]{#1^{\top}_{\tau_{#2}}}
\newcommand{\gvarF}[2]{#1^{\bot}_{\tau_{#2}}}
\newcommand{\gabst}[4]{\lambda \gvar{#1}{#2}{#3}. #4}
\newcommand{\gghost}[1]{\mathtt{ghost}~ #1}
\newcommand{\glet}[5]
{\mathtt{let}~\gvar{#1}{#2}{#3} = #4 ~ \mathtt{in}~ #5}
\newcommand{\gif}[3]{\mathtt{if}~#1~\mathtt{then}~#2~\mathtt{else}~#3}
\newcommand{\grech}[6]
	{\mathtt{rec}~\var{#1}{\mf{B_{#2}}}{}~\gvar{#3}{#4}{#5}~:~\tau_{#6}. t}
\newcommand{\grec}[4]
	{\mathtt{rec}~\var{f}{\mf{B_{#1}}}{}~\gvar{x}{#3}{#4}:\tau_{#2}.~t}
\newcommand{\gread}[3]{!\gref{#1}{#2}{#3}}
\newcommand{\gwrite}[4]{\gref{#1}{#2}{#3} := #4}
%%%%%%%%%%%%%%%%%%%%%%%%%%%%%%%%%%%%%%%%%%%%%%%%%%%%%%%%%%%%%%%%%%%%%

%SEMANTICS
\newcommand{\leval}[4]{~#1_{|#2} \rightarrow_{\lambda} #3_{|#4} ~}
\newcommand{\geval}[4]{~#1_{|#2} \rightarrow_{g\lambda} #3_{|#4} ~}

\newcommand{\levalh}[4]{~#1_{|#2} \stackrel{\epsilon}{\rightarrow}_{\lambda} #3_{|#4} ~}
\newcommand{\gevalh}[4]{~#1_{|#2} \stackrel{\epsilon}{\rightarrow}_{g\lambda} #3_{|#4} ~}

\newcommand{\levalstar}[4]{~#1_{|#2} \rightarrow_{\lambda}^{\star} #3_{|#4} ~}
\newcommand{\gevalstar}[4]{~#1_{|#2} \rightarrow_{g\lambda}^{\star} #3_{|#4} ~}
\newcommand{\gstep}[2]{~#1 ~ {\rightarrow}_{g\lambda} ~ #2~}
\newcommand{\ghead}[2]{~#1~\stackrel{\epsilon}{\rightarrow}~#2~}
\newcommand{\gstar}[2]{~#1 ~ {\rightarrow}^{\star}_{g\lambda} ~ #2~}
\newcommand{\glhead}[2]{#1~\stackrel{\epsilon\quad}
										{\rightarrow_{g\lambda}}~#2}
\newcommand{\stepone}[2]{#1 ~ {\rightarrow}~ #2}
\let\eval\stepone





%%%%%%%%%%%%%%%%%%%%%%%%%%%%%%%%%%%%%%%%%%%%%%%%%%%%%%%%%%%%%%%%%%%%%
\newcommand{\tarr}[3]{\tau^{\mf{B}_{#3}}_{#1} \rightarrow \tau_{#2}}
\newcommand{\tarrS}[4]
	{\tau^{\mf{B}_{#3}}_{#1}
	\stackrel{\Sigma_#4}{\longrightarrow} \tau_{#2}}

\newcommand{\sbst}[3]{#1 [#2 \mapsfrom #3] }
%%%%%%%%%%%%%%%%%%%%%%%%%%%%%%%%%%%%%%%%%%%%%%%%%%%%%%%%%%%%%%%%%%%%%
\newcommand{\grtitle}[2]{#1 & ::= &  & \textit{#2} \\}
\newcommand{\grhead}[3]{#1 & ::= & #2 & \textit{#3} \\}
\newcommand{\grcase}[2]{&  & #1 & \textit{#2} \\}
%%%%%%%%%%%%%%%%%%%%%%%%%%%%%%%%%%%%%%%%%%%%%%%%%%%%%%%%%%%%%%%%%%%%%

% TYPING
\newcommand{\tystepone}[3]{\vdash_{g\lambda}  #1 : (#2, #3) }
\newcommand{\typrule}[5]{~\vdash_{g\lambda}  #1 : (#2, #3, #4) #5}

% ERASURE
\newcommand{\e}{\mathcal{E}}
\newcommand{\ebot}[1]{\e_{\bot}(#1)}
\newcommand{\etop}[1]{\e_{\top}(#1)}
\newcommand{\evar}[2]{\e_{#1}(#2)}


% CONSTANTS
\newcommand{\glvar}{\gvar{x}{}{}}
\newcommand{\glref}{\gref{r}{}{}}
\newcommand{\glabst}{\gabst{x}{}{}{t}}
\newcommand{\glapp}{t_1 ~ t_2}
\newcommand{\gltyping}{~\typrule{t}{\tau}{\mf{B}}{\Sigma}{}~}
\newcommand{\vardecl}{\text{var }\glref = v}
\newcommand{\gllet}{\glet{x}{}{}{t}{t}}
\newcommand{\glif}{\gif{t}{t}{t}}
\newcommand{\glrec}{\grec{}{}{}{}}
\newcommand{\glread}{!\gref{r}{}{}}
\newcommand{\glwrite}{\gref{r}{}{} := t}
%%%%%%%%%%%%%%%%%%%%%%%%%%%%%%%%%%%%%%%%%%%%%%%%%%%%%%%%%%%%%%%%%%%%%
\newcommand{\Longdownarrow}{\rotatebox{90}{$\Longleftarrow$}}
%%%%%%%%%%%%%%%%%%%%%%%%%%%%%%%%%%%%%%%%%%%%%%%%%%%%%%%%%%%%%%%%%%%%%
\newcommand{\lgvar}[2]{#1_{#2}}
\newcommand{\lgvarc}{x_{\tau}}

\newcommand{\lgabs}[3]{\lambda \lgvar{#1}{#2}.{#3}}
\newcommand{\lgabsc}{\lgabs{x}{\tau_2}{t_1}}

\newcommand{\lgapp}[2]{#1~#2}
\newcommand{\lgappc}{t_1~t_2}

\newcommand{\lglet}[4]{\text{let } \lgvar{#1}{#2} = #3 \text{ in } #4}
\newcommand{\lgletc}{\lglet{x}{\tau_2}{t_2}{t_1}}

\newcommand{\lgif}[3]{\text{ if } #1 \text{ then } #2 \text{ else } #3}
\newcommand{\lgifc}{\lgif{t_1}{t_2}{t_3}}

\newcommand{\lgrec}[5]{\text{rec } #1~\lgvar{#3}{#4} : #2 = #5}
\newcommand{\lgrecc}{\lgrec{g}{\tau_1}{x}{\tau_2}{t_1}}

\newcommand{\lgand}[2]{#1 \wedge #2}
\newcommand{\lgandc}{\lgand{t_1}{t_2}}

\newcommand{\lgor}[2]{#1 \bvee #2}
\newcommand{\lgorc}{\lgor{t_1}{t_2}}

\newcommand{\lgneg}[1]{\neg #1}

\newcommand{\lgexist}[3]{\exists \glvar{#1}{#2}. #3}
\newcommand{\lgexistc}{\lgexist{x}{\tau_2}{f_1}}

\newcommand{\lgforall}[3]{\forall \glvar{#1}{#2}. #3}
\newcommand{\lgforallc}{\lgforall{x}{\tau_2}{f_1}}

%%%%%%%%%%%%%%%%%%%%%%%%%%%%%%%%%%%%%%%%%%%%%%%%%%%%%%%%%%%%%%%%%%%%%%%%%%%%%%%
\title{TITLE}
\author{Léon}
%\date{\today}

\begin{document}

\maketitle

\tableofcontents

\begin{abstract}
  This is the abstract...
\end{abstract}

\section{Introduction}

context: deductive program verification~\cite{filliatre11sttt}

main idea = if a program is using a HO function to write a loop, its
proof of correctness should not be more difficult than its imperative
counterpart using a for/while loop

motivating examples

related work 

\section{Programming Language}

mini-ML + global references + recursive functions

\subsection{Syntax}

\subsection{Semantics}

\subsection{Type System with Effects}

side effects + termination, independently
%%%%%%%%%%%%%%%%%%%%%%%%%%%%%%%%%%%%%%%%%%%%%%%%%%%%%%%%%%%%%%%%%%%%%%%%%%%%%%%
\newpage
\section{Inlining}

	present general idea of inlining in two phases and a brief description of 
	required tools.
	
%%%%%%%%%%%%%%%%%%%%%%%%%%%%%%%%%%%%%%%%%%%%%%%%%%%%%%%%%%%%%%%%%%%%%%%%%%%%%%%
\subsection{Semantic Equivalence Relation}

%%%%%%%%%%%%%%%%%%%%%%%%%%%%%%%%%%%%%%%%%%%%%%%%%%%%%%%%%%%%%%%%%%%%%%%%%%%%%%%
\subsubsection*{Semantic Equivalence Between Closed Terms}

%%%%%%%%%%%%%%%%%%%%%%%%%%%%%%%%%%%%%%%%%%%%%%%%%%%%%%%%%%%%%%%%%%%%%%%%%%%%%%%
\label{semantic equivalence}
\begin{definition}[(Semantic Equivalence)]
	Let $t$ and $t'$ be two well-typed closed terms of the source 
	language such that 
	
	$$
		\typerule{t}{\tau}{\theta}{\rho}{} \quad \typerule{t'}{\tau}{\theta}{\rho}{} 
	$$

	We define $\eqv{t}$, the \textit{semantic equivalence} between $t$ and $t'$, 
	by induction on the structure of type $\ty$ :
	
	\begin{itemize}
		\item[$(\alpha)$]

			if $\ty{} = \ty{}^0$ for some base type $\ty{}^0$, then $\eqv{t}$ iff
			$$	\forall \mu_0.(\evalinfty{t}{\mu_0} \bwedge \evalinfty{t'}{\mu_0})~
					\bvee ~ \exists \mu_1 \exists v: 
					\evalstar{t}{0}{v}{1} \bwedge \evalstar{t'}{0}{v}{1} $$
		
		\item[$(\beta)$]
			if $\ty = \tyarr$ for some types $\ty[1], \ty[2]$,
			then\\[0.2cm]
			$\hspace*{0em}\forall v_0, v'_0. ~ \eqv{v_0} ~ 
			\bwedge \typerule{v_0}{\ty[1]}{\bot_\theta}{\bot_\rho}{} ~~
			\bwedge \typerule{v'_0}{\ty[1]}{\bot_\theta}{\bot_\rho}{} $ \\[0.2cm]
			$\hspace*{1em}
			\Rightarrow \forall\mu_0.(\evalinfty{\tmapp{t}{v_0}}{\mu_0}
			\bwedge\evalinfty{\tmapp{t'}{v'_0}}{\mu_0})$\\[0.2cm]
			$\hspace*{2em}\bvee~(\exists \mu_1 \exists v_1, v'_1 : 
			 	\evalstar{\tmapp{t}{v_0}}{0}{{v_1}}{1} 
				\bwedge \evalstar{\tmapp{t'}{v'_0}}{0}{{v'_1}}{1}
				\bwedge~\eqv{v_1})$				
	\end{itemize}
	\end{definition}


%%%%%%%%%%%%%%%%%%%%%%%%%%%%%%%%%%%%%%%%%%%%%%%%%%%%%%%%%%%%%%%%%%%%%%%%%%%%%%%
	\begin{lemma} 
		The semantic equivalence 	terms, $\thicksim$, is reflexive, 	
		symmetric and transitive relation.
	\end{lemma}
	
	\begin{proof}
		By straightforward induction on the structure of type of the terms, using 
		the fact that $\rightarrow^\star$ is an evaluation strategy, thus 
		deterministic.
	\end{proof}

%%%%%%%%%%%%%%%%%%%%%%%%%%%%%%%%%%%%%%%%%%%%%%%%%%%%%%%%%%%%%%%%%%%%%%%%%%%%%%%
	\begin{lemma}
		$\forall t,t'.~ \eqv{t} \Leftrightarrow 
			(\forall v, v'.~\eqv{v}~\Rightarrow
			\tmapp{t}{v} \thicksim \tmapp{t'}{v'})$
	\label{equiv-def-l}
	\end{lemma}
	
	\begin{proof} By induction on the structure of type $\ty[t]$.
	For Detailed proof, see \ref{equiv-def-p}.
	\end{proof}

%%%%%%%%%%%%%%%%%%%%%%%%%%%%%%%%%%%%%%%%%%%%%%%%%%%%%%%%%%%%%%%%%%%%%%%%%%%%%%%
	
	\begin{lemma}
		For any pair of states $\mu_0$, $\mu_1$ such that
		$\evalstar{{t_0}}{0}{{t_1}}{1}$ and $\evalstar{{t'_0}}{0}{{t'_1}}{1}$
		if $\eqv{t_0}$ then $\eqv{t_1}$.
		\label{equiv-red2-l}
	\end{lemma}
	
	\begin{proof}
		By induction on the structure of type of $t_0$. 
		For Detailed proof, see \ref{equiv-red2-p}.
	\end{proof}		

	\begin{corr} 
		For any pair of states $\mu_0$, $\mu_1$ such that
		$\evalstar{{t}}{0}{{v}}{1}$ and \mbox{$\evalstar{{t'}}{0}{{v'}}{1}$},
		if $\eqv{t}$ then $\eqv{v_1}$.
		\label{equivalence parallel preservation corr}
	\end{corr}
	
	
	
	\begin{lemma}
		For any state $\mu_0$, if $\evalstep{{t_0}}{0}{{t_1}}{0}$, then
		$t_0 \thicksim t_1$.
	\end{lemma}	
	\begin{proof}
		By Straightforward observation that $\rightarrow$ is an evaluation
		strategy, thus deterministic.
	\end{proof}		

	\begin{corr} 
		For any state $\mu_0$, if $\evalstar{{t}}{0}{{v}}{0}$, then		
		$t \thicksim v$.
	\end{corr}
%%%%%%%%%%%%%%%%%%%%%%%%%%%%%%%%%%%%%%%%%%%%%%%%%%%%%%%%%%%%%%%%%%%%%%%%%%%%%%%
\subsubsection*{Semantics Equivalence Between Parallel Substitutions}

	\begin{definition}[(parallel substitutions)]
	
	\label{}
	\end{definition}

	\begin{definition}[(equivalent parallel substitutions)]

	\label{}
	\end{definition}

	\begin{lemma} 
		$\forall t. \forall (\sigma, \sigma'). ~\eqvsbst{\sigma}{t}
		\Rightarrow \tmapp{t}{\sigma} \thicksim \tmapp{t}{\sigma'}.$ 
	\label{equiv-subst-l}
	\end{lemma}

	\begin{proof}
		By induction on the structure of term $t$. For detailed proof, see
		\ref{equiv-subst-p}.
	\end{proof}

\subsection{Inlining As Rewriting Strategy}

Short description of two steps

\subsubsection*{Inlining Second-Order Local Bindings}

- output language subset
- sos definition
- theorem 1

\subsubsection*{Inlining Second-Order Applications}

- output language subset
- sos definition
- theorem 2

\subsubsection*{Output Language}
TODO: Nice figure 

%\subsection{Inlining Higher-Order Programs}

%%%%%%%%%%%%%%%%%%%%%%%%%%%%%%%%%%%%%%%%%%%%%%%%%%%%%%%%%%%%%%%%%%%%%%%%%%%%%%%
\newpage
\section{Specification Language}

\subsection{Ghost Code}

\begin{definition}[blah blah] blah
\end{definition}

\begin{theorem}[blah] 
blah
\end{theorem}

\begin{lemma}[blah blah blah]
blah
\end{lemma}

type system, erasure, proof of correctness of erasure

\subsection{Annotations}

requires, ensures, assert

\section{Putting Pieces Together}

\subsection{Orthogonality}

inlining adapted to specification language

\subsection{Experimental Evaluation}

\section{Conclusion and Perspectives}

blah blah

%%%%%%%%%%%%%%%%%%%%%%%%%%%%%%%%%%%%%%%%%%%%%%%%%%%%%%%%%%%%%%%%%%%%%%%%%%%%%%
\newpage
\section*{Appendices}
\newpage
%%%%%%%%%%%%%%%%%%%%%%%%%%%%%%%%%%%%%%%%%%%%%%%%%%%%%%%%%%%%%%%%%%%%%%%%%%%%%%
\appendix
\section{Detailed proofs} 
	\subsection{Semantic Equivalence Relation}
	\begin{lemma}
		$\forall t,t'.~ \eqv{t} \Leftrightarrow 
			(\forall v, v'.~\eqv{v}~\Rightarrow
			\tmapp{t}{v} \thicksim \tmapp{t'}{v'})$
	\label{equiv-def-p}
	\end{lemma}

	\begin{proof}
		Let $v, v'$ be two arbitrary values satisfying $\eqv{v}$.
		From the formulation of lemma, it follows that
		$$	\typerule{t}{\tyarr[1][2][\theta_0][\rho_0]}{\theta}{\rho}{} ~~ 
				\typerule{t'}{\tyarr[1][2][\theta_0][\rho_0]}{\theta}{\rho}{}.$$
		where $\ty[2]$ can be either the some first-order type $\tyord{0}[2]$ or
		some arrow type $\tyarr[21][22][\theta_1][\rho_1].$ 
		We prove the lemma by induction on the structure of type $\ty[t]$.
	\begin{itemize}		
	
		\item[$(\Rightarrow)$] Assume $\eqv{t}$. Thus 		
		
		\begin{itemize}
		
		\item[$(\alpha)$] if $\ty[\tmapp{t'}{v'}] = \tyord{0}[2]$, then
		by definition of $\eqv{t}$, for any initial state $\mu_0$ either 
		$\tmapp{t'}{v'}$ and $\tmapp{t'}{v'}$ diverge both, 
		or there is some pair of values $v_1, v'_1$ and a state $\mu_1$ such that 
		$$\evalstar{\tmapp{t}{v}}{0}{{v_1}}{1} 
			\bwedge \evalstar{\tmapp{t'}{v'}}{0}{{v'_1}}{1} \bwedge~\eqv{v_1}.$$
		
		By the preservation of typing, $v_1$ and $v'_1$ are of some base 
		type $\tyord{0}[2]$, and from the fact that they are values, we deduce
		that $v_1 = v'_1$. Therefore,  $\tmapp{t}{v} \thicksim \tmapp{t'}{v'})$.
			
		\item[$(\beta)$] Otherwise, 
		$\ty[\tmapp{t}{v}] = \tyarr[21][22][\theta_1][\rho_1]$.
		Let $v_0, v'_0$ be two arbitrary values satisfying $\eqv{v_0}$ and 
		$\ty[v_0] = \ty[21]$. We must show that for any initial state $\mu_0$,
		either both 
		$\tmapp{\tmapp{t}{v}}{v_0}$ and $\tmapp{\tmapp{t'}{v'}}{v'_0}$ diverge
		or there is a pair of values $v_1, v'_1$ and a state $\mu_1$ such that 
		$$\evalstar{\tmapp{\tmapp{t}{v}}{v_0}}{0}{{v_1}}{1} \bwedge
		\evalstar{\tmapp{\tmapp{t'}{v'}}{v'_0}}{0}{{v'_1}}{1} \bwedge~\eqv{v_1}.$$
		
		Obviously, if both $t$ and $t'$ both diverge,
		then so do $\tmapp{t}{v}$ and $\tmapp{t'}{v'}$, and consequently 
		$\tmapp{\tmapp{t'}{v'}}{v'_0}$ and $\tmapp{\tmapp{t'}{v'}}{v'_0}$.
		Otherwise, from the equivalence $\eqv{t}$ we can deduce that there are
		some values $v_2$, $v'_2$ and some intermediate state $\mu_2$ such that
		$$\evalstar{\tmapp{\tmapp{t}{v}}{v_0}}{0}{\tmapp{v_2}{v_0}}{2} \bwedge
		\evalstar{\tmapp{\tmapp{t'}{v'}}{v'_0}}{0}{\tmapp{v'_2}{v'_0}}{2} 
		\bwedge~\eqv{v_2}$$	
		By induction hypothesis on $\ty[\tmapp{t}{v}]=\ty[v]$, as $\eqv{v_2}$, we 
		get	$$ \tmapp{v_2}{v_0} \thicksim \tmapp{v'_2}{v'_0}.$$ Therefore, 
		$\tmapp{v_2}{v_0}$ diverges if and only if $\tmapp{v_2'}{v_0'}$ diverges 
		too, so again, $\tmapp{\tmapp{t'}{v'}}{v'_0}$ diverges if and only if 
		$\tmapp{\tmapp{t'}{v'}}{v'_0}$ diverges (by determinism of 
		$\rightarrow^\star$). Otherwise for some values $v_1, v'_1$ and a state
		$\mu_1$:
		$$\evalstar{\tmapp{v_2}{v_0}}{2}{{v_1}}{1} \bwedge
		\evalstar{\tmapp{v_2}{v'_0}}{2}{{v'_1}}{1} 
		\bwedge~\eqv{v_1}.$$ There, putting reduction chains together allow us to 
		conclude that $\tmapp{t}{v} \thicksim \tmapp{t'}{v'}.$ 
		\end{itemize}
		
	
		\item[$(\Leftarrow)$] Assume $(\forall v_0, v'.~\eqv{v}~\Rightarrow
		\tmapp{t}{v} \thicksim \tmapp{t'}{v'})$. From the definition of 
		\mbox{$\tmapp{t}{v}\thicksim\tmapp{t'}{v'}$} we almost immediate have 
		that $\eqv{t}$. Back to \ref{equiv-def-l}.
				
		\end{itemize}	
	\end{proof}

	\begin{lemma}
		For any pair of states $\mu_0$, $\mu_1$ such that
		$\evalstar{{t_0}}{0}{{t_1}}{1}$ and $\evalstar{{t'_0}}{0}{{t'_1}}{1}$
		if $\eqv{t_0}$ then $\eqv{t_1}$.
		\label{equiv-red2-p}
	\end{lemma}
	
	\begin{proof}
		By induction on the structure of type of $t_0$. 
		\begin{itemize}
		\item[$(\alpha)$] if $\ty = \ty^0$ for some base type $\ty^0$, then
		
			\begin{itemize}
			\item[$(\alpha_1)$] if $t_0$ diverges, then by definition of 
			$\eqv{t_0}$, $t'_0$ diverges too. Also, $\rightarrow^\star$ being 
			deterministic, both $t_1$ and $t'_1$ diverge. Symmetrically, if 
			$t'_0$ diverges, $t_1$ and $t'_1$ diverge both as well. Therefore,
			$$\eqv{t_1}.$$
						
			\item[$(\alpha_2)$] Otherwise, neither $t_0$ or $t_{0}'$ diverge. Then
				as $\eqv{t_0}$ by hypothesis, it follows that 
					$$ \exists \mu_2 \exists v: 
					\evalstar{{t_0}}{0}{v}{2} \bwedge \evalstar{{t'_0}}{0}{v}{2}.$$
				The evaluation $\rightarrow^\star$ being deterministic, we have 
				necessarily that
					$$\evalstar{{t_1}}{1}{v}{2} \bwedge \evalstar{{t'_1}}{1}{v}{2}.$$
				As the state $\mu_1$ depends uniquely on the arbitrarily chosen 
				$\mu_0$, we deduce again that $$\eqv{t_1}.$$					
			\end{itemize}			
			
		\item[$(\beta)$] if $\ty$ is an arrow type $\tyarr$ for some types 
		$\ty[1], \ty[2]$,	let $v_0, v'_0$ be two arbitrary values satisfying:
		$$~\eqv{v_0}~\bwedge \typerule{v_0}{\ty[1]}{\bot_\theta}{\bot_\rho}{}.$$
		By hypothesis, we have the reduction steps:
		$$\evalstar{{t_0}}{0}{{t_1}}{1}\quad\evalstar{{t'_0}}{0}{{t'_1}}{1}$$						which induce respectively: 
		$$\evalstar{\tmapp{t_0}{v_0}}{0}{\tmapp{t_1}{v_0}}{1} \quad
		\evalstar{\tmapp{t'_0}{v'_0}}{0}{\tmapp{t'_1}{v'_0}}{1}$$	
		From the definition of $\eqv{t_0}$, it follows that
		$${\tmapp{t_0}{v_0}} \thicksim {\tmapp{t'_0}{v'_0}}.$$
		Therefore, the induction hypothesis on $\ty[2]$ yields 
		$${\tmapp{t_1}{v_0}}\thicksim{\tmapp{t'_1}{v'_0}}.$$
		Finally, as $v_0, v'_0$ are arbitrary chosen values (satisfying the
		two conditions above), the result holds for any such values, 
		therefore by lemma \ref{equiv-def-l},
		$$\eqv{t_1}.$$ Back to \ref{equiv-red2-l}.	
		\end{itemize}	
	\end{proof}		

	\begin{lemma}

	\label{equiv-subst-p}
	\end{lemma}

	\begin{proof}
	Back to \ref{equiv-subst-l}.
	\end{proof}
	
	\subsection{Inlining Second-Order Local Bindings}
	\subsection{Inlining Second-Order Applications}

%%%%%%%%%%%%%%%%%%%%%%%%%%%%%%%%%%%%%%%%%%%%%%%%%%%%%%%%%%%%%%%%%%%%%%%%%%%%%%
\newpage
\section{References} 
%\addcontentsline{toc}{chapter}{Bibliography}
\bibliographystyle{plain}
\bibliography{abbrevs,demons,demons2,demons3,team,crossrefs}

\end{document}

(*
Local Variables:
compile-command: "rubber -d main"
End:
*)
