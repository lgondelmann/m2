%%%%%%%%%%%%%%%%%%%%%%%%%%%%%%%%%%%%%%%%%%%%%%%%%%%%%%%%%%%%%%%%%%%%%%%%%%%%%%%%
\documentclass[a4paper,11pt,oneside]{article}
\pagestyle{headings}
\label{packages}
\usepackage[sc]{mathpazo}
\usepackage[scaled]{helvet} % ss
\usepackage[utf8]{inputenc}
\usepackage[T1]{fontenc}
\usepackage[english]{babel}
\usepackage[a4paper,colorlinks=true,urlcolor=blue,pdfstartview=FitH]{hyperref}
\usepackage{graphicx}
\usepackage{amssymb,amsfonts,amsthm,amsmath}
\usepackage{titlesec}
\usepackage{listings}
\usepackage[usenames,dvipsnames]{xcolor}
\usepackage[dvipsnames]{xcolor,colortbl}
\usepackage{caption}
\protect\usepackage{semantic}
\usepackage{bcprules, proof}
\usepackage{stmaryrd}
\SetSymbolFont{stmry}{bold}{U}{stmry}{m}{n}
 \usepackage{float}
\usepackage{changepage}
\usepackage{setspace}
\usepackage{pstricks,pstricks-add,pst-math,pst-xkey}
\usepackage{multicol}
\usepackage[small,nohug,heads=vee]{diagrams}
\usepackage[capitalise,noabbrev,nameinlink]{cleveref}
\usepackage{xargs}
\usepackage{chngcntr}
\usepackage{ulem}
\usepackage{cancel}
\crefname{enumi}{position}{positions}
\diagramstyle[labelstyle=\scriptstyle]

\titleformat{\section}[hang]% style du titre
  {\normalfont\LARGE\bfseries}% police du titre + numéro
  {\thesection}% numérotation
  {0.4in}% espacement numéro/titre <--- valeur à changer
  {}% police spécifique du titre
\titlespacing*{\section}
  {-0.7in}% espacement gauche
  {0.2in}% espacement avant
  {0.2in}% espacement après
  [0.1in]% espacement à droite

\titleformat{\subsection}[hang]% style du titre
  {\normalfont\Large\bfseries}% police du titre + numéro
  {\thesubsection}% numérotation
  {0.2in}% espacement numéro/titre <--- valeur à changer
  {}% police spécifique du titre
\titlespacing*{\subsection}
  {-0.6in}% espacement gauche
  {0.2in}% espacement avant
  {0.2in}% espacement après
  [0.1in]% espacement à droite


\titleformat{\subsubsection}[hang]% style du titre
  {\normalfont\large\bfseries}% police du titre + numéro
  {\thesubsubsection}% numérotation
  {0.1in}% espacement numéro/titre <--- valeur à changer
  {}% police spécifique du titre
\titlespacing*{\subsubsection}
  {-0.4in}% espacement gauche
  {0.2in}% espacement avant
  {0.2in}% espacement après
  [0.1in]% espacement à droite


\titleformat{\paragraph}[hang]% style du titre
  {\normalfont\large\bfseries}% police du titre + numéro
  {\paragraph}% numérotation
  {0.1in}% espacement numéro/titre <--- valeur à changer
  {}% police spécifique du titre
\titlespacing*{\paragraph}
  {0.0in}% espacement gauche
  {0.2in}% espacement avant
  {0.2in}% espacement après
  [0.1in]% espacement à droite

\RequirePackage{listings}
\RequirePackage{amssymb}

\lstset{
	xleftmargin=\parindent,
  basicstyle={\ttfamily},
%  framesep=2pt,
%  frame=single,
  keywordstyle={\color{blue}},
  stringstyle=\itshape,
  commentstyle=\itshape,
  columns=[l]fullflexible,
  showstringspaces=false,
  mathescape=true
}

\lstdefinelanguage{why3}
{
morekeywords={begin,namespace,predicate,function,inductive,type,use,clone,%
import,export,theory,module,end,in,with,%
let,rec,for,to,do,done,match,if,then,else,while,try,invariant,variant,%
absurd,raise,assert,exception,private,abstract,mutable,ghost,%
downto,raises,writes,reads,requires,ensures,returns,val,model,%
goal,axiom,lemma,forall},%
string=[b]",%
sensitive=true,%
morecomment=[s]{(*}{*)},%
keepspaces=true,
}
%literate=%
%{'a}{$\alpha$}{1}%
%{'b}{$\beta$}{1}%
%{<}{$<$}{1}%
%{>}{$>$}{1}%
%{<=}{$\le$}{1}%
%{>=}{$\ge$}{1}%
% {<>}{$\ne$}{1}%
% {/\\}{$\land$}{1}%
% {\\/}{ $\lor$ }{3}%
% {\ or(}{ $\lor$(}{3}%
% {not\ }{$\lnot$ }{1}%
% {not(}{$\lnot$(}{1}%
% {+->}{\texttt{+->}}{2}%
% % {+->}{$\mapsto$}{2}%
% {-->}{\texttt{-\relax->}}{2}%
% %{-->}{$\longrightarrow$}{2}%
% {->}{$\rightarrow$}{2}%
% {<->}{$\leftrightarrow$}{2}%

\lstnewenvironment{whycode}
	{\lstset{language=why3}}
	{\vspace*{-0em}}
\lstnewenvironment{ocamlcode}{\lstset{language={[Objective]Caml}}}{}

%\newcommand{whymode}[1]{\begin{whycode}#1\end{whycode}}

%\lstset{basicstyle={\ttfamily}}
\let\why\lstinline
\newcommand{\HRule}{\rule{\linewidth}{0.5mm}}


\label{proclamations}
\newtheoremstyle{plain}
{\topsep}{\topsep}{\upshape}{}{}{:~}{ }
{\textsc{\hspace{-1.55cm} #2 \quad \textcolor{black}{#1}} \textsc{\textcolor{black}{#3}}}
\theoremstyle{plain}
\newtheorem{definition}{Definition}[subsection]
\newtheorem{lemma}[definition]{Lemma}
\newtheorem{theorem}[definition]{Theorem}
\newtheorem{corr}[definition]{Corrolary}

\renewenvironment{proof}{\noindent \begin{adjustwidth}{1.5em}{} \textcolor{blue}{\textit{Proof.}}}
{{\begin{tiny}\textcolor{blue}{$\blacksquare$}\end{tiny}}
\end{adjustwidth}~\\\noindent}



\label{ML-Terms marcos}
\newcommand{\mlt}[1]{#1}
\newcommand{\tmapp}[2]{(#1 ~ #2)}
\newcommand{\tmlet}[3]{let~#1=#2~in~#3}
\newcommandx{\tmrec}[5][5=]{rec~#1_{#5}~#2 : #3.#4}
\newcommand{\varslash}[2]{#1 / #2}
\newcommand{\tmsbst}[3]{#1 [#2 \mapsfrom #3] }
\label{ML-Types marcros}
\newcommand{\ty}[1][]{\tau_{#1}}
\newcommandx{\tyarr}[4][1=1, 2=2, 3=\theta, 4=\rho]
	{\tau_{#1}\hspace*{-0.1cm}\stackrel{(#3,#4)}
	{\Longrightarrow}\hspace*{-0.1cm}\tau_{#2}}
\newcommandx{\tyord}[2][2=]{\ty[#2]^{#1}}
\newcommand{\rouge}[1]{\textcolor{red}{#1}}
%
%\newcommand{\tarrS}[4]
%	{\tau^{\mf{B}_{#3}}_{#1}
%	\stackrel{\Sigma_#4}{\longrightarrow} \tau_{#2}}


\newcommand{\bwedge}{\boldsymbol{~\wedge~}}
\newcommand{\bvee}{\boldsymbol{~\vee~}}
\newcommand{\brarr}{\boldsymbol{~\Rightarrow~}}
\label{ML-Typing marcos}
\newcommandx{\typerule}[5]{~\vdash  #1 : (#2, #3, #4) #5}

\newcommand{\typing}[4]{\vdash~#1~:~#2,~#3,~#4}
\newcommand{\ghosttyping}[6]{\vdash_{gh}~#1~:~#2,~(#3,~#4,~\rouge{#5})~\textcolor{blue}{\boldsymbol{#6}}}
\newcommand{\bth}{\bot_\theta}
\newcommand{\brh}{\bot_\rho}  
\newcommand{\tth}{\top_\theta}
\newcommand{\trh}{\top_\rho}   
\label{Semantics marcos}
\newcommand{\evalstep}[4]{~#1_{\mu_#2} \rightarrow #3_{\mu_#4} ~}
\newcommand{\evalstar}[4]{~#1_{\mu_#2} \rightarrow^{\star} #3_{\mu_#4} ~}
\newcommand{\evalinfty}[2]{~#1_{#2} \rightarrow \infty ~}
\newcommand{\eqv}[1]{#1 \thicksim #1'}
\newcommand{\eqvsbst}[2]{#1 \thicksim_{#2} #1'}
\label{Inlining macros}

\newcommand{\inlS}{\mathcal{S}}
\newcommand{\inlU}{\mathcal{U}}
\newcommand{\inlsrc}{\textit{ML}^{^2}}
\newcommand{\inlT}{\inlsrc}

\newcommand{\hookdownarrow}{\mathrel{\rotatebox[origin=c]{180}{$\hookleftarrow$}}}
\newcommand{\inlletarr}{\hookdownarrow}
\newcommand{\inlletstar}{\hookdownarrow^{\star}}
\newcommand{\inlletplus}{\inlletstar}
\newcommand{\inlletNF}{\overset{\inlletplus}{NF}}
\newcommand{\inllet}[2]{#1 \hookdownarrow #2}
\newcommand{\inllett}[2]{#1 \inlletplus #2}


\newcommand{\ilarr}{\hookdownarrow}
\newcommand{\ilarrt}{\ilarr^{\star}}
\newcommand{\ilNF}{\overset{\ilarrt}{NF}}
\newcommand{\il}[2]{#1 \ilarr #2}
\newcommand{\ilt}[2]{#1 \ilarrt #2}

\newcommand{\icarr}{\hookrightarrow}
\newcommand{\icarrt}{\icarr^{\star}}
\newcommand{\icNF}{\overset{\icarrt}{NF}}
\newcommand{\ic}[2]{#1 \icarr #2}
\newcommand{\ict}[2]{#1 \icarrt #2}

%
\let\mf\mathfrak


\addtolength{\textwidth}{1.5cm}

\def\nrepeat#1#2{\count0=#1 \loop \ifnum\count0>0 \advance\count0 by -1 #2\repeat}

%\swapnumbers




\newcommand{\mem}{_{\mu}}\newcommand{\memp}{_{\mu'}}

%%%%%%%%%%%%%%%%%%%%%%%%%%%%%%%%%%%%%%%%%%%%%%%%%%%%%%%%%%%%%%%%%%%%%
\newcommand{\glam}{\textit{ghost}-$\lambda$~}
\newcommand{\gml}{\textit{ghost}-ml~}
%%%%%%%%%%%%%%%%%%%%%%%%%%%%%%%%%%%%%%%%%%%%%%%%%%%%%%%%%%%%%%%%%%%%%

\newcommand{\var}[3]{#1^{#2}_{#3}}
\newcommand{\gvar}[3]{#1^{\mathfrak{B_{#2}}}_{\tau_{#3}}}
\newcommand{\gref}[3]
{#1^{\mathfrak{B_{#2}}}_{\mathtt{ref}~\tau_{#3}}}
\let\rvar\gref
\newcommand{\gvarT}[2]{#1^{\top}_{\tau_{#2}}}
\newcommand{\gvarF}[2]{#1^{\bot}_{\tau_{#2}}}
\newcommand{\gabst}[4]{\lambda \gvar{#1}{#2}{#3}. #4}
\newcommand{\gghost}[1]{\mathtt{ghost}~ #1}
\newcommand{\glet}[5]
{\mathtt{let}~\gvar{#1}{#2}{#3} = #4 ~ \mathtt{in}~ #5}
\newcommand{\gif}[3]{\mathtt{if}~#1~\mathtt{then}~#2~\mathtt{else}~#3}
\newcommand{\grech}[6]
	{\mathtt{rec}~\var{#1}{\mf{B_{#2}}}{}~\gvar{#3}{#4}{#5}~:~\tau_{#6}. t}
\newcommand{\grec}[4]
	{\mathtt{rec}~\var{f}{\mf{B_{#1}}}{}~\gvar{x}{#3}{#4}:\tau_{#2}.~t}
\newcommand{\gread}[3]{!\gref{#1}{#2}{#3}}
\newcommand{\gwrite}[4]{\gref{#1}{#2}{#3} := #4}
%%%%%%%%%%%%%%%%%%%%%%%%%%%%%%%%%%%%%%%%%%%%%%%%%%%%%%%%%%%%%%%%%%%%%

%SEMANTICS
\newcommand{\leval}[4]{~#1_{|#2} \rightarrow_{\lambda} #3_{|#4} ~}
\newcommand{\geval}[4]{~#1_{|#2} \rightarrow_{g\lambda} #3_{|#4} ~}

\newcommand{\levalh}[4]{~#1_{|#2} \stackrel{\epsilon}{\rightarrow}_{\lambda} #3_{|#4} ~}
\newcommand{\gevalh}[4]{~#1_{|#2} \stackrel{\epsilon}{\rightarrow}_{g\lambda} #3_{|#4} ~}

\newcommand{\levalstar}[4]{~#1_{|#2} \rightarrow_{\lambda}^{\star} #3_{|#4} ~}
\newcommand{\gevalstar}[4]{~#1_{|#2} \rightarrow_{g\lambda}^{\star} #3_{|#4} ~}
\newcommand{\gstep}[2]{~#1 ~ {\rightarrow}_{g\lambda} ~ #2~}
\newcommand{\ghead}[2]{~#1~\stackrel{\epsilon}{\rightarrow}~#2~}
\newcommand{\gstar}[2]{~#1 ~ {\rightarrow}^{\star}_{g\lambda} ~ #2~}
\newcommand{\glhead}[2]{#1~\stackrel{\epsilon\quad}
										{\rightarrow_{g\lambda}}~#2}
\newcommand{\stepone}[2]{#1 ~ {\rightarrow}~ #2}
\let\eval\stepone





%%%%%%%%%%%%%%%%%%%%%%%%%%%%%%%%%%%%%%%%%%%%%%%%%%%%%%%%%%%%%%%%%%%%%
\newcommand{\tarr}[3]{\tau^{\mf{B}_{#3}}_{#1} \rightarrow \tau_{#2}}
\newcommand{\tarrS}[4]
	{\tau^{\mf{B}_{#3}}_{#1}
	\stackrel{\Sigma_#4}{\longrightarrow} \tau_{#2}}

\newcommand{\sbst}[3]{#1 [#2 \mapsfrom #3] }
%%%%%%%%%%%%%%%%%%%%%%%%%%%%%%%%%%%%%%%%%%%%%%%%%%%%%%%%%%%%%%%%%%%%%
\newcommand{\grtitle}[2]{#1 & ::= &  & \textit{#2} \\}
\newcommand{\grhead}[3]{#1 & ::= & #2 & \textit{#3} \\}
\newcommand{\grcase}[2]{&  & #1 & \textit{#2} \\}
%%%%%%%%%%%%%%%%%%%%%%%%%%%%%%%%%%%%%%%%%%%%%%%%%%%%%%%%%%%%%%%%%%%%%

% TYPING
\newcommand{\tystepone}[3]{\vdash_{g\lambda}  #1 : (#2, #3) }
\newcommand{\typrule}[5]{~\vdash_{g\lambda}  #1 : (#2, #3, #4) #5}

% ERASURE
\newcommand{\e}{\mathcal{E}}
\newcommand{\ebot}[1]{\e_{\bot}(#1)}
\newcommand{\etop}[1]{\e_{\top}(#1)}
\newcommand{\evar}[2]{\e_{#1}(#2)}


% CONSTANTS
\newcommand{\glvar}{\gvar{x}{}{}}
\newcommand{\glref}{\gref{r}{}{}}
\newcommand{\glabst}{\gabst{x}{}{}{t}}
\newcommand{\glapp}{t_1 ~ t_2}
\newcommand{\gltyping}{~\typrule{t}{\tau}{\mf{B}}{\Sigma}{}~}
\newcommand{\vardecl}{\text{var }\glref = v}
\newcommand{\gllet}{\glet{x}{}{}{t}{t}}
\newcommand{\glif}{\gif{t}{t}{t}}
\newcommand{\glrec}{\grec{}{}{}{}}
\newcommand{\glread}{!\gref{r}{}{}}
\newcommand{\glwrite}{\gref{r}{}{} := t}
%%%%%%%%%%%%%%%%%%%%%%%%%%%%%%%%%%%%%%%%%%%%%%%%%%%%%%%%%%%%%%%%%%%%%
\newcommand{\Longdownarrow}{\rotatebox{90}{$\Longleftarrow$}}
%%%%%%%%%%%%%%%%%%%%%%%%%%%%%%%%%%%%%%%%%%%%%%%%%%%%%%%%%%%%%%%%%%%%%
\newcommand{\lgvar}[2]{#1_{#2}}
\newcommand{\lgvarc}{x_{\tau}}

\newcommand{\lgabs}[3]{\lambda \lgvar{#1}{#2}.{#3}}
\newcommand{\lgabsc}{\lgabs{x}{\tau_2}{t_1}}

\newcommand{\lgapp}[2]{#1~#2}
\newcommand{\lgappc}{t_1~t_2}

\newcommand{\lglet}[4]{\text{let } \lgvar{#1}{#2} = #3 \text{ in } #4}
\newcommand{\lgletc}{\lglet{x}{\tau_2}{t_2}{t_1}}

\newcommand{\lgif}[3]{\text{ if } #1 \text{ then } #2 \text{ else } #3}
\newcommand{\lgifc}{\lgif{t_1}{t_2}{t_3}}

\newcommand{\lgrec}[5]{\text{rec } #1~\lgvar{#3}{#4} : #2 = #5}
\newcommand{\lgrecc}{\lgrec{g}{\tau_1}{x}{\tau_2}{t_1}}

\newcommand{\lgand}[2]{#1 \wedge #2}
\newcommand{\lgandc}{\lgand{t_1}{t_2}}

\newcommand{\lgor}[2]{#1 \bvee #2}
\newcommand{\lgorc}{\lgor{t_1}{t_2}}

\newcommand{\lgneg}[1]{\neg #1}

\newcommand{\lgexist}[3]{\exists \glvar{#1}{#2}. #3}
\newcommand{\lgexistc}{\lgexist{x}{\tau_2}{f_1}}

\newcommand{\lgforall}[3]{\forall \glvar{#1}{#2}. #3}
\newcommand{\lgforallc}{\lgforall{x}{\tau_2}{f_1}}

\newcommand{\gb}{\beta}
\newcommand{\gba}[1]{\beta_{#1}}
\newcommand{\gbb}{\bot_\beta}
\newcommand{\gbt}{\top_\beta}

\newcommand{\gbr}{\textcolor{red}{\gb}}
\newcommand{\gbra}[1]{\textcolor{red}{\gba{#1}}}
\newcommand{\gbbr}{\textcolor{red}{\gbb}}
\newcommand{\gbtr}{\textcolor{red}{\gbt}}
\newcommand{\gbran}[1]{\textcolor{red}{\neg\gba{#1}}}


\newcommand{\meganerd}	
	{\url{http://www.meganerd.com/erikd/Blog/CodeHacking/Ocaml/fold.html}}
\newcommand{\libsndfile}
{\url{
http://www.meganerd.com/erikd/Blog/CodeHacking/libsndfile/ten_years.html}}	
	
%%%%%%%%%%%%%%%%%%%%%%%%%%%%%%%%%%%%%%%%%%%%%%%%%%%%%%%%%%%%%%%%%%%%%%%%%%%%%%%
\title{TITLE}
\author{Léon}
%\date{\today}
\counterwithin{figure}{subsection}
%\sloppy 
\addtolength{\hoffset}{-0.5in}
\addtolength{\textwidth}{0.8in}
\hbadness=10000
\hfuzz=\maxdimen
%\tolerance=10000
\begin{document}

\maketitle

\tableofcontents

%\begin{abstract}
%  This is the abstract...
%\end{abstract}

\section{Introduction}

	The main goal of this memoir is to present the a technical solution of higher-order programs \textit{inlining} in the context of deductive program verification. 
	
	Deductive software verification is the process where the correctness
of programs is proved using computational logic and an axiomatic formal semantics approach$^{\cite{Hoare69anaxiomatic}}$.
	In this process, first the correctness of a program is expressed
as a set of axioms, hypotheses and assertions, called \textit{verification conditions}.
	Using axioms and hypotheses, assertions are then proved
by either interactive proof assistants or automated theorem provers.

	Until recently, interactive proof assistants were predominating
in verification practice.
	Indeed, problems involving linear arithmetic, floating-point number analysis,
or array bounds check, occur ubiquitously in any realistic software
 verification process, but were not supported by the earlier models of automated theorem provers, leaving no choice but to use proof assistants.
	
	The new generation of ATPs, the \textit{Satisfiability Modulo Theories} (SMT) provers, brings a hope of more automated verification.
	SMT solvers combine a classical ATP approach for solving first-order logic problems with a built-in support for linear arithmetic, nonlinear arithmetic, bitvectors, arrays, datatypes, etc.
	
	Though these theorem provers are rather powerful, generally they do not yet perform proofs for programs that make use of higher-order definitions. \\
	 
	More generaly, deductive software verification was historically and still is a framework focusing mostly on the verification of imperative programs written in languages like C or Java.
	
	Consequentially, software verification environments (such as \textit{Why3} platform, or Microsoft research \textit{Spec\#} TODO), that rely on ATPs to discharge verification conditions, do not yet provide a support for higher-order program specification. \\

%
%
%	Verification process begins with writing down specifications for a program we want to prove. 
%	Verification tools often dispose a special programming language, called \textit{Intermediate Specification Language (ISL)}, in which both programs and specifications can be written. 
%	The computational part of an \textit{ISL} usually corresponds to a fragment of some existing programming language. 
%	For instance, Why3 platform relies on a \textit{Whyml}, an ISL which contains a simplified subset of the Ocaml programming language. \\
%	
%	FramaC$^{\cite{Cuoq}}$'s \textit{isl} relies on \textit{CIL} (C Intermediate Language) which contains a simplified subset of the C programming language. \\
%	

  \subsection{Related Work}	
  
  
	The aim of this memoire is somehow different from the works above. 
	Our goal is not to describe the most general class of higher-order expressions that can be inlined, but to make rather a practical investigation about the potential of higer-order inlining in deductive verification discipline.  
	
	As we will show later in evaluation section, many of higher-order definitions of real projects does not require all the power of ML.
	On the contrary, most of higher-order expressions encountered in practice are expressions such as \textit{iterators} \textit{mappings}, and \textit{folders}.
	The goal of this work is thus to show that the verification of programs using these expressions can be done in a similar way to the verification of a programs using  simple imperative loop. 

	\subsection{Outline}	
	
  We start our discussion by giving some of the key-ideas of inlining informally.
  Then, in Section 2 we describe the a simple functional programming language in which we can write interesting programs we want to inline. 
  
   

%		The rest of our work is dedicated to a more formal description of inlining. 
%	In the section 2, we present Mini-ML, a tiny, but pertinent fragment of Ocaml, in which we can write programs like our introductory examples. 
%	To eliminate all possible programs on which inlining would fail, we restrain Mini-ML to a subset of second-order Mini-ML programs of a particular form, $\inlsrc$, that contains only the programs on which inlining procedure can operate safely. 

	Section 3 give then a formal definition of inlining of procedure itself and prove 
	prove some results, like termination and soundness of the procedure. 


	Section 4 is somehow orthogonal to the rest of work. It presents briefly the logic in which we can write annotations for our examples. It also introduces the notion ghost-code, which will be useful to annotate programs more easily. 
	
	 Finally, Section 5 illustrates how the inlining procedure works on the examples involving the most common higher-order definitions such as \textit{List.map}, \textit{List.fold}, etc. 
	
 
  
	
%%%%%%%%%%%%%%%%%%%%%%%%%%%%%%%%%%%%%%%%%%%%%%%%%%%%%%%%%%%%%%%%%%%%%%%%%%%%%%%	
\subsection{Key-ideas}		

\subsubsection*{An introductory example}			
	
	In one of his blog posts\footnote{\meganerd}, Eric di Castro Lopo, author of a widely-used C-library \textit{libsndfile}\footnote{\libsndfile}, gives the following example of two equivalent programs, one with an iterative loop, the other with a \textit{Array.iter}, that both compute the sum of the elements of a given \textit{array}:\vspace{-0.5cm}\footnote{we slightly adapt his examples to the syntax of WhyML} 

\begin{small}
	\begin{minipage}[t]{0.4\linewidth}
		\begin{whycode}  
 let sum_loop (a: array int) =		
  let s = ref 0 in
  let i = ref 0 in
  while !i < a.length do
    s := !s + a[!i];
    i := !i + 1
  done; !s 
 (*Array sum with while loop*)
		\end{whycode}
	\end{minipage}\hfill 
	\begin{minipage}[t]{0.51\linewidth}
		\begin{whycode} 
  let sum_iter (a: array int) =		 
    let s = ref 0 in
    Array.iter (fun x -> s := !s + x) a;



 
    (*Array sum with iterator*)
		\end{whycode}	
	\end{minipage}
\end{small}

	As states di Castro Lopo, functions like \texttt{iter}, \texttt{map}, or \texttt{fold}: \textit{``can reduce the number of points of possible error in a program. 
	More importantly, for the code reader who understands and is comfortable with these techniques, reading and understanding code using these functions is quicker than reading and understanding the equivalent for loop.''}
			
	Now imagine that one wants not only to understand the programs above, but also to verify their correctness formally. 
	The key for a solution of the imperative program would be a well-known \textit{Hoare logic} rule for \texttt{while}:
\begin{small}
$$\frac { \{P \land B \}\ S\ \{P\} }
{ \{P \}\ \textbf{while}\ B\ \textbf{do}\ S\ \textbf{done}\ \{\neg B \land P\}}$$
\end{small} 
where \texttt{P} stands the loop invariant and B is the loop condition.
% that once becoming false, must have caused the loop to stop. 
Here is how \texttt{sum\_loop} can be specified and proved in Why3. 
A natural way to specify and annotate \texttt{sum\_loop} is:
%	
% $\sum_{~0\leq i < n} a[i]$,
\begin{small}
	\begin{whycode}  
let sum_loop (a: array int) 	
  $\textcolor{OliveGreen}{\text{ensures}\{~\sum_{~0 \leq j < n} a[j]~\}}$ = 
  let s = ref 0 in
  let i = ref 0 in
  while !i < a.length do
    $\textcolor{OliveGreen}{\textbf{invariant }\{~P \}}$
    s := !s + a[!i];
    i := !i + 1
  done; !s 
 	\end{whycode}
 \end{small}
where the  P is a shortcut for $ !s = \sum_{~0\leq j < i} a[j] \bwedge 0 \leq !i \leq n~$ and $n$ for the length of array \texttt{a}. 
The verification conditions computed automatically by Why3 from these annotations are:
\begin{footnotesize}
\begin{displaymath}
\begin{array}{ll@{\hspace*{2em}}r}


(vc1)
	& \vdash (0 \leq 0 \bwedge 0 \leq n) 
\bwedge 0 = \sum_{~0\leq j < 0} a[j]	
	& \textsc{(loop initialisation)} \\


(vc2)
	& P \bwedge i < n \vdash
		(0 \leq (i+1) \leq n) 
		\bwedge s + a[i] = \sum_{~0 \leq j < i+1} a[j]
	& \textsc{(loop preservation)} \\

(vc3)
	& P \bwedge i < n \vdash 0 \leq i \leq n 
	& \textsc{(array bound check)} \\

(vc4)
	& P \bwedge i \geq n \vdash s = \sum_{~0 \leq j < n} a[j] 
	& \textsc{(postcondition)} 
\end{array}
\end{displaymath}
\end{footnotesize}

	All these verification conditions are first-order formulas with linear arithmetic. 
	They are proved by Alt-Ergo SMT solver instantly. \\
	
	So, is there anything similar to the \textit{Hoare logic} rule for \texttt{while} loop?
	Can we find a way to annotate and specify a functional program as easily as for the imperative one? 
	Can we get from \texttt{sum\_iter} annotations the verifications conditions that would be roughly the same as vc1-vc4 ?
	
\subsubsection*{A Technical Solution: Inlining}
	\qquad One possible solution is higher-order program \textit{inlining}.
	Inlining of higher-order programs is a syntactic transformation that turns
each higher-order expression into an expression of first-order degree. 
	Higher-order local variables of a higher-order type are replaced by functional expression they introduce; 
	formal parameters of functional type are substituted by actual arguments inside applications; 
	more generally, every higher-order expression inside a program is progressively simplified until no higher-order expression is left. 
	The rest of the source program, that is, any expression of first-order
type, remains syntactically unchanged. 

	It is obviously not possible to inline any higher-order program. 
	In the next subsection we show two examples where inlining fails. 
	However, in the case of \texttt{sum\_iter} program, inlining does produce a correct result of a first-order program equivalent to the \texttt{sum\_iter} above.
	Here is how. 
	First we inline the definition of \texttt{array\_iter}, then substitute its formal parameter \texttt{f} by the function \texttt{fun x -> s:= !s + x}, so that the source code becomes:


\begin{adjustwidth}{-1em}{-2em}
\begin{footnotesize}
\begin{minipage}[t]{0.3\linewidth}
	\begin{whycode}  
   let array_iter ($\textcolor{red}{\text{f:int -> unit}}$) (b: int array)   
   = let rec loop (i: int) =  
       if i < b.length 
       then ($\textcolor{red}{\text{f}}$ b[i] ; loop (i + 1)) 
     in loop 0
   
   let sum_iter (a: array int) =		 
     let s = ref 0 in
     array_iter $\textcolor{red}{\text{(fun x -> s := !s + x)}}$ a  
     
  (*(1.1) source code *)   
 	\end{whycode}
 	\end{minipage}\hfill\vline
 \begin{minipage}[t]{0.45\linewidth} 

	\begin{whycode}  
   let sum_iter (a: array int) =		
     let s = ref 0 in
     let array_iter $\textcolor{Sepia}{\text{\textvisiblespace}}$ (b: int array)   
     = let rec loop (i: int) =  
         if i < b.length then
         ($\textcolor{Sepia}{\text{\underline{(fun x -> s := !s + x)}}}$ b[i] ; 
         loop (i + 1)) 
       in loop 0   
     in Array.iter $\textcolor{Sepia}{\text{\textvisiblespace}}$ a   
     
    (*(1.2) after inlining *)
 	\end{whycode}
 	\end{minipage}
 \end{footnotesize}
\end{adjustwidth}

It remains to annotate (1.2) so that inlining would transform it into an annotated code that should suffice to prove the inlined program above. 
This can be achieved by extending the definition of \texttt{array\_iter} with an additional parameter \textcolor{blue}{\texttt{\textbf{inv:int-> int array -> prop}}} which stands for the loop invariant $P$. 
\begin{small}\hypertarget{sum-iter}{example}
\begin{whycode}  
   let array_iter ($\textcolor{red}{\text{f:int -> unit}}$) (b: int array) ($\textcolor{red}{\text{inv:int -> int array -> prop}}$)    
     $\textcolor{OliveGreen}{\text{requires \{ \textcolor{red}{\text{inv }} 0~b \}}}$
     $\textcolor{OliveGreen}{\text{ensures~~}\{ \text{ \textcolor{red}{inv} b.length~b }\}}$  
   = let rec loop (i: int)
       $\textcolor{OliveGreen}{ \text{requires \{~\textcolor{red}{inv} i b} \bwedge \text{0 <= i <= b.length}~\}}$
       $\textcolor{OliveGreen}{\text{ensures~~}\{ \text{ \textcolor{red}{inv} b.length~b}\}}$ =    
       = if i < b.length then ($\textcolor{red}{\text{f}}$ b[i]; loop (i+1)) 
     in loop 0       
                                          (*array_iter partial specification*)
\end{whycode}
\end{small}

It is important to see that the specification above is not only higher-order, but also \textit{partial}: it provides no information about behaviour of its functional parameter \texttt{f}. 
	For instance, it keeps no track of potential side-effects produced by \texttt{f}. 
%	However, in the example we discuss here, as in many other realistic examples, the auxiliary function \texttt{loop} of \texttt{array\_iter} simulates to imperative loop reiteration by recursion, the body of \texttt{f} corresponds exactly to the body of the loop. 
	However, we can provide the missing part of specification by instantiating  \textcolor{blue}{\texttt{\textbf{inv}}} with the actual loop invariant \textcolor{blue}{$\lambda i_{int}.\lambda c_{\texttt{int array}}. \sum_{~0\leq j < i} a[j] $} inside the call of \texttt{array\_iter}:
\begin{small}
\begin{whycode} 
 let sum_iter (a: array int) =		 
   let s = ref 0 in 
   array_iter $\textcolor{red}{\text{(fun x -> s := !s + x)}}~$ a $\textcolor{red}{(\lambda i_{int}.\lambda c_{\texttt{int array}}. \sum_{~0\leq j < i} a[j])}$                    
                                   (*invariant instantiation in the source code *)
\end{whycode}
\end{small}

After inlining processing, we get:

\begin{footnotesize}
\begin{whycode}  
   let sum_iter (a: array int)		
     let s = ref 0 in
     let array_iter $\textcolor{Sepia}{\text{\textvisiblespace}}$ (b: int array) $\textcolor{Sepia}{\text{\textvisiblespace}}$
     $\textcolor{OliveGreen}{\text{requires} \{~\underline{!s = \sum_{~0\leq j < 0} b[j]}}~\}$      
     $\textcolor{OliveGreen}{\text{ensures~~}\{~\underline{!s = \sum_{~0\leq j < b.length} b[j]}\}}$    
     = let rec loop (i: int) = 
         $\textcolor{OliveGreen}{ \text{requires} \{~\underline{!s = \sum_{~0\leq j < i} b[j] } \bwedge \text{0 <= i <= b.length}~\}}$
         $\textcolor{OliveGreen}{\text{ensures~~}\{~\underline{!s = \sum_{~0\leq j < b.length} b[j]}\}}$   
         = if i < b.length then  
         ($\textcolor{Sepia}{\text{\underline{s := !s + b[i]}}}$ ; 
          loop (i + 1)) 
       in loop 0   
     in Array.iter $\textcolor{Sepia}{\text{\textvisiblespace}}~a~\textcolor{Sepia}{\text{\textvisiblespace}}$   
               (*specified code after inlining*)
 	\end{whycode}
 \end{footnotesize}


As we can see, the key-idea is to represent the loop invariant of while construction, splitting it between the specification of iterator's definition and the annotation-like formula of iterator's call.
The advantage of this trick is that we need to write the specification of iterator only once, and to pass each call's invariant formula independently from the other calls.

	So, the specification of \texttt{sum\_iter} after inlining is now expressed by within first-order logic.
	Furthermore, it results in verification conditions similar to VCs \texttt{vc1-vc4}. 
	In particular, these VCs are all proved by Alt-Ergo as instantly as \texttt{vc1-vc4}, so that the inlined program using \texttt{array\_iter} is proved with respect to the initial specification.
 
 \subsubsection*{Incompleteness and Correctness of Inlining}
 
The most important thing to understand about inlining is that, by its
nature, inlining is a \textit{syntactic} transformation. 
Our main concern will be the \textbf{correctness} of inlining: it should be \textit{deterministic} and always \textit{terminating} procedure, whose output has exactly the same \textit{meaning} that the corresponding source program. 
 
%Obviously, we need to design the inlining procedure in such way that, when applied to a program that only \textit{makse use} of high-order expressions, but returns some
%first-order data like \texttt{int} or \textit{list bool}, it transforms the initial program in a purely first-order program like in the introductory example above. 
%must be \textit{total}. That is, for any program that only
%\textit{makes use} of high-order expressions, but should return some
%first-order data like \texttt{int} or \textit{list bool}, the inlining must
%\textit{always} result in a purely first-order program. (Indeed, programs
%being \textit{closed}, \textit{well-typed} terms, the only higher-order
%expressions that might appear inside, are either local definitions or
%applications of a functional to functions, and these are precisely what
%inlining aims to simplify). 

It might seem at first look that we could inline any relatively simple higher-order ML program alike in the case of \texttt{sum\_iter} above, because the only higher-order expressions that might appear inside are either local definitions or applications of a functional to functions, which are precisely what inlining aims to simplify.	 

Here are two examples of simple higher-order programs for which inlining fails.
%
%%So can we take the entire set of ML programs as the domain of inlining
%%function ?  Would it always be possible to transform a higher-order program
%%into a first-order program that contains no higher-order content ?
%
% Unfortunately, if we do not impose any restriction on the set of ML programs contains even very basic small programs, for which inlining would fail, producing an incorrect result. 
% Let us illustrate this with the following two examples. \\

One interesting feature of ML is its possibility to write programs with side
effects. For instance, we can assign a reference. 
Suppose now that we want to inline
	$$ t \triangleq 
	\tmlet 
            {\textcolor{blue}
            {F_{(int \rightarrow int) \rightarrow int}}}
	{(r_{int}:= 42; \lambda f_{int \rightarrow int}. t_0)}{t_1} $$
Intuitively, inlining operation should not modify the bounded expression
$(r_{int}:= 42; \lambda f_{int \rightarrow int}. t_0)$ in order to prevent an
assignment of variable $r$, so the result of inlining should be directly a
substitution
	$$ I(t) \approx 
	\tmsbst{t_1}
		{\textcolor{blue}{F}}
		{(r_{int}:= 42; \lambda f_{int \rightarrow int}. t_0)} $$ 
	Unfortunately, the semantics of $t_1$ is not preserved by this inlining: if
$\textcolor{blue}{F}$ has multiple occurrences in $t_1$, the variable
assignment $r:=42$ will be duplicated, and if $F$ does not occur in $t_1$,
then it will be lost. 
	Even if $\textcolor{blue}{F}$ occurs just once in
$t_1$, we cannot be sure that the initial order of assignments is
preserved.
 	And if we replace $r:=42$ in the example above by some recursive
call that loops infinitely, from a non-terminating program $t$ we get an possibly
terminating program $I(t)$, so again the semantics is not preserved. \\

Recursive functions with  higer-order parameters is another (de)motivating
example: suppose we want to inline two different programs $t$ and $t'$ :\vspace{-0.4cm}
	\begin{figure}[H]
	\label{fig:rec-bad-ex} \hypertarget{rec-bad-ex}{}
		\begin{footnotesize}
	$$ t \triangleq 
		\texttt{ let } \textcolor{blue}{F} =
  		\tmrec{f}{\textcolor{blue}{g_{int \rightarrow int}}~x_{int}}{int} 
  		{\texttt{ if } x = 0 
  		\texttt{ then } (\textcolor{blue}{g}~x) 
  		\texttt{ else }((f~\textcolor{Sepia}{\boldsymbol{g}})~(x - 1))} 
  		\texttt{ in }
  		(\textcolor{blue}{F}~\textcolor{red}{\lambda y_{int}. t_0})$$ 
  $$t' \triangleq 
  \texttt{ let } \textcolor{blue}{F} =
  \tmrec{f}{\textcolor{blue}{g_{int \rightarrow int}}~x_{int}}{int} 
  {\texttt{ if } x = 0 
  \texttt{ then } (\textcolor{blue}{g}~x) 
  \texttt{ else } ((f~\textcolor{Sepia}{\boldsymbol{\lambda y. 0}})~(x-1))} 
  \texttt{ in }(\textcolor{blue}{F}~\textcolor{red}{\lambda y_{int}.t_0}) $$
  \end{footnotesize}
   
  \end{figure}  \vspace{-0.4cm}
\indent As we suggested above, inlining consists in replacing,
inside application $(\textcolor{blue}{F}~\textcolor{red}{\lambda y_{int}. t_0})$, 
the local variable $\textcolor{blue}{F}$ by its definition and
then substituting inside the body of recursive function $f$ the formal parameter
$\textcolor{blue}{g}$ by the actual argument 
$\textcolor{red}{\lambda y_{int}. t_0}$. 
However substituting $\textcolor{blue}{g}$ inside recursive call 
$((f~\textcolor{Sepia}{\boldsymbol{g}})~(x - 1))$ does not make any sense,
because the output would still contain a higher-order expression that cannot be
simplified, that is, inlining would fail. The only solution would therefore to
simply erase inside every recursive call any formal parameter of functional
type:
	\begin{footnotesize}
	$$ I(t) \approx
  \tmrec{f}{\textcolor{blue}{\text{\textvisiblespace}}~x_{int}}{int}
  	{\texttt{ if } x = 0\texttt{ then }(\textcolor{red}{\lambda y_{int}.t_0}~x)
    \texttt{ else } (f~\textcolor{blue}{\text{\textvisiblespace}}~(x-1))} $$
	$$ I(t') \approx
  \tmrec{f}{\textcolor{blue}{\text{\textvisiblespace}}~x_{int}}{int} 
  	{\texttt{ if }x = 0\texttt{ then }(\textcolor{red}{\lambda y_{int}.t_0}~x)
    \texttt{ else }(f~\textcolor{blue}{\text{\textvisiblespace}}~(x-1))} $$
	\end{footnotesize}
\indent By chance, the formal parameter $g$ remains unchanged inside recursive call
$((f~g)~x)$, so $t$ and $I(t)$ do have the same semantics. Unfortunately, in the
case of program $t'$, instead of $g$ we have a constant function
$\lambda y. 0$, so that the soundness of inlining is broken: for instance, if for some
non-null integer x the call $(\lambda y. t_0~x)$ loops, then while $(t'~x)$ will
return zero, $(I(t')~x)$ will loop too. \\

There may be several solutions to this problem. For example, we could define inlining, without guaranteeing that semantics of the source program is preserved. 
However, our solution is to characterize a reasonably large subset of ML programs, such that every program can be inlined correctly. 

\newpage
\section{Higher-Order Programming Language }

	This section describes a little functional programming language \textit{Mini-ML}, in which we can write programs like previous examples. 
	It also introduces ${\inlsrc}$, a restriction of Mini-ML in which we can only write programs we can inline correctly, like in the example of \texttt{sum\_iter}. 
	
\subsection{Mini-ML Programming Language}

	Roughly speaking, Mini-ML is equivalent to \textbf{PCF + state}, a standard experimental subject in computer science{\footnotesize$^{ \cite[p.~143]{Pierce:2002:TPL:509043}}$}. 
	That is, Mini-ML consists of a simply-typed lambda-calculus with built-in primitive data types and recursive functions extended with global store of modifiable references.
	
	The formal description of the Mini-ML is given below: 
\begin{figure}[H]
\hrule
\begin{adjustwidth}{0em}{-3em}
\begin{footnotesize}
\begin{minipage}[t]{0.49\linewidth}
\begin{displaymath}
\begin{array}{llr}
\tau & ::= & \textsc{types} \\
   	& int ~|~ bool ~|~ unit ~|~ \dots & \textit{built-in simple types} \\
    & list_{int}											& \textit{built-in recursive types} \\
    & \tau \stackrel{\theta, \rho}{\Rightarrow} \tau           
    																	& \textit{function type} \\
  \\
\theta, \rho & ::= 						& \textsc{effects indicators} \\
& \bth ~|~ \tth & \textit{reference assignment} \\
& \brh ~|~ \trh			& \textit{recursive function use} \\
\\
p & ::=										 					& \textsc{programs} \\
	&(ref~r_{\tau} := v) \quad p &\textit{global reference declaration}\\
	& t & \textit{term} \\
	\\
& \hspace*{-0.8cm} t_{\mu} \rightarrow t_{\mu}
	& \textsc{evaluation rules} \\
& \qquad ...
	& \textit{(see \cref{mini-ml-def-sem})} \\
\\
& \hspace*{-0.8cm} \vdash t : \tau, \theta, \rho
	& \textsc{typing rules} \\ 						
& \qquad ...
	& \textit{(see \cref{mini-ml-def-typ})} \\[1cm]
	
& \hspace*{-0.8cm}\fbox{\textsc{Mini-ML}} 
	
\end{array}
\end{displaymath}
\end{minipage} 
\hspace*{0.7	em} \vrule \hfill 
\begin{minipage}[t]{0.49\linewidth}
\begin{displaymath}
\begin{array}{llr}
	t & ::=										& \textsc{terms}   \\
  	& v 										& \textit {value} \\
  	& (t~v) 								& \textit {application} \\
  	& let~x_{\tau} = t~in~t & \textit {local binding} \\
  	& if~v~then~t~else~t 		& \textit{if-then-else} \\
  	& match~v~with \\
  	& ~|~Nil~->~t~|~Cons~x_{\tau}~x_{\tau}~->t~ 
  													& \textit{match-with}\\
  	& r_{\tau} := v 				& \textit{assignment} \\
  	& !r_{\tau} 						& \textit {dereference} \\
	\\
	v & ::= 											& \textsc{values} \\
  	& x_{\tau}						  		& \textit {variable} \\
		& \lambda x_{\tau} . t 		  & \textit{function} \\
		& \mu f:~(\tau, \theta).~
			\lambda x_{\tau}. t				& \textit{recursive function} \\
		& c													& \textit{constant}  \\
	\\
  c & ::= 								& \textsc{constants} \\
  	& \mathbb{N}~|~\mathbb{B}~|~() ~|~ ... 
  												& \textit{base type constants} \\
  	& Nil ~|~ Cons   
  												& \textit {list constructors}  \\ 
%  	& \mu t. (Empty ~|~ Node~t~c_{int}~t )
%  												&  \textit{integer binary tree} \\
  	& + ~|~ - ~|~ ... 		
  												& \textit{arithmetic operators} \\
  	& \neg ~|~ \bwedge ~|~ ... 
  												& \textit{Boolean operators} \\
 		& = 
 													& \textit{monomorphic equality} \\
\end{array}
\end{displaymath}
\end{minipage} 	 		 
\end{footnotesize}
\end{adjustwidth}
\label{mini-ml-def-syn}
\hrule
\end{figure} 
	In the figure above all compound terms are put in variant of \textbf{A-normal} form$^{\cite{Flanagan}}$. 
	By that we mean that some part of compound terms is already a value. 
	For instance, we do not authorize applications $(t~t')$ of a some compound term $t$ to another compound terms $t'$.
	Instead, only applications $(t~v)$ of terms to values are allowed.
	 
	 This does not reduces thought the expressiveness of the language. 
	 Indeed, putting terms into A-normal form is just a compilation trick to make our presentation shorter and easier to read: 
	 we could pass from any term in A-normal form to general form and back via \textit{let} expression like in $$(t_1~t_2) \simeq~let~x~=~t_2~in~(t_1~x)$$.
	
	
\subsubsection*{The Semantics of Mini-ML }
	The semantics of Mini-ML is given by the standard call-by value operational semantics{\footnotesize$^{ \cite[p.~54]{Pierce:2002:TPL:509043}}$} in a small-step manner.  
	Each transition rule is of the form \fbox{$t_\mu \rightarrow t'_{\mu'}$} where $\mu$ and $\mu'$ are global reference store transition states. 
	Reduction step can take place directly \textit{on the top} of term $t$. All possible cases for such step are defined by set of transition rules below:

	\begin{figure}[H]
	  \begin{small}
	\begin{spacing}{2} 
	
	\infax[E-Op-$\delta$] 
	{\ghead
		{ c~v_1 \dots {v_{k}} {\mem}}
		{ \delta(c,v_1 \dots v_k) \mem} \quad 
		\text{if } k = Arity(c) \text{ and }\delta(c,v)\text{ is defined }} 
	
  \infax[E-Op-$\lambda$]
	{\ghead
		{{c~v_1 \dots v_k}{\mem}}
		{\lambda x_{\tau}. c~v_1 \dots v_k ~ x_{\tau}}\quad 
		\text{if } 1 \leq k < arity(c)}		
	
	\infax[E-AppFun] 
	{\ghead
		{ (\lambda x_\tau~v)\mem }
		{ \sbst{t}{x_{\tau}}{v}\mem }} 
	  
	
	\infax[E-AppRec]
	{\ghead
		{
			(\mu f:(\tau, \theta).~
				\lambda x_{\tau'}. t~v)\mem}
		{t[
			 x_{\tau_2} \mapsfrom v, 
			 f \mapsfrom  \mu f:(\tau, \theta) .~\lambda x_{\tau'}. t]\mem}}
			 
	
	\infax[E-Let-V]
	{\ghead
	 {let~x_{\tau} ~=~v_1~in~{t_{2}}\mem}
	 {\sbst{t_{2}}{x_\tau}{v_{1}}\mem}} 
	
	\infax[E-If-true]		
	{\ghead
		{\mathtt{if}~\mathtt{true}~\mathtt{then}~t_{1}~\mathtt{else}~t_{2}{\mem}}
		{t_{1}{\mem}}}
	
	\infax[E-If-false]		
	{\ghead
		{\mathtt{if}~\mathtt{false}~\mathtt{then}~t_{1}~\mathtt{else}~t_{2}{\mem}}
		{t_{2}{\mem}}}	

	\infax[E-Match-Nil]
	{\ghead
		{match~\textbf{Nil}~with 
  	 ~|~Nil~->~t_1~|~{Cons~x_1}_{\tau}~x_{\tau}~->{t_2}{\mem}~~}
		{~~ {t_1} \mem}
	}
	
	\infax[E-Match-Cons]
	{\ghead
		{match~\mathbf{Cons~n~l} ~with 
  	 ~|~Nil~->~t_1~|~{Cons~x_1}_{\tau}~{x_2}_{\tau'}~->{t_2}{\mem}\\}
		{{t_2}[{x_1}_{\tau} \mapsfrom n, {x_2}_{\tau'} \mapsfrom l] \mem} 
		\text{   where n} \in \mathbb{N} \text{ and l is some list value} 
	}	
	
	\infax[E-Deref]	 	
	{\ghead
	{{!r_\tau}_{\mem}}
	{\mu_{}(r_\tau)}}

	\infax[E-Assign]
	{\ghead
		{r_\tau := v\mem}
		{()_{|\mu[r_{\tau} \mapsfrom ~ v]}}}
\end{spacing}
\caption{ \textbf{Mini-ML Semantics (Head Reduction Rules)} \hfill}
\end{small}
\end{figure}
\vspace*{-0.4cm}
	Here, total application of built-in operators and list constructors is defined by the set of \textit{delta} rules. 
	These rules are quite intuitive, so we only give few examples.
	For instance, the rule for equality is 
   $$\delta(t \mathtt{=} u))~\triangleq~t =_{\tau} u $$	
	where $=_\tau$ is equality modulo $\alpha$-equivalence defined for each type $\tau$ by structural induction on $t$.
	Similarly, the rules for list constructors \textit{Cons, Nil} are defined by induction on the structure of lists: 
	\begin{displaymath}
	\begin{array}{lll}

	\delta(Nil) &=& Nil \\
	\delta(Cons, (n, l)) &=& Cons~n~\delta(l)~~~ \text{if } n \in \mathbb{N} \text{ and } \delta(l) \text{ is defined } 
	\end{array}
	\end{displaymath}

	Reduction step can also occur not on the top of $t$, but on the top of some subterm of $t$. 
	Because our language in A-normal form, there are only two possible cases for such reduction step: 
	\begin{figure}[H]
	\begin{small}
	\begin{spacing}{1.2} 	
	\begin{adjustwidth}{-6em}{-6em}
	\begin{minipage}[t]{0.41\linewidth}
	\infrule[E-App-T]	
	{{t_{1}}\mem \rightarrow {{t'}_{1}}\memp}
	{{(t_{1}~v_2)}\mem \rightarrow {({t'}_{1}~v_2)}\memp}
	\end{minipage}
	\begin{minipage}[t]{0.45\linewidth}
	\infrule[E-Let-T]
	{{t_{1}}\mem \rightarrow {{t'}_{1}}\memp}
	{\mathtt{let} ~ x_\tau = t_{1} ~ \mathtt{in}~ {t_{2}}\mem
		\rightarrow 
		\mathtt{let} ~ x_\tau = t_{1}' ~ \mathtt{in}~ {t_{2}}\memp}
 \end{minipage}
 \end{adjustwidth}
\end{spacing}
\end{small}
\caption{ \textbf{Mini-ML Semantics (Context Reduction Rules)} \hfill}
\label{mini-ml-def-sem}
\end{figure}

\subsubsection*{The Typing System of Mini-ML } 
	
	The typing system is more interesting, because it is actually a typing system \textit{with effects}. 
	
	Typing systems with effects guarantee not only that "well-typed programs do not go wrong", but also play a role of an "intelligence agency" that gathers information about potential side-effects of each term. 
	
	In our case, the typing system keeps track of whether a reference assignment or a recursive call appears inside each well-typed term.		
	
	Each typing rule is of the form $\vdash t : \tau, \theta, \rho$,  where $\tau$ stands for term's type, $\theta$ and $\rho$ indicates each respectively the presence/absence of reference assignments, and recursive calls inside $t$.
	 
	The set of typing rules of ML-typing is given in the figure below:	
\begin{figure}[H]
%\hrule
\begin{adjustwidth}{-0em}{-1em}
\begin{small}
\begin{minipage}[t]{0.49\linewidth}
\begin{displaymath}
\begin{array}{l}

\dfrac
	{}
	{\typing{x_{\tau}}{\tau}{\bot_{\theta}}{\bot_{\rho}}}
	{\textsc{  (T-var)}} \\[1cm]

\dfrac
	{\textsc{Typeof}(c) = \tau}
	{\typing{c}{\tau}{\bot_{\theta}}{\bot_{\rho}}}
	{\textsc{  (T-const)}} \\[1cm]

	
\dfrac
	{\typing{t_1}{\tau_1}{\theta_1}{\rho_1}}
	{\typing{\lambda x_{\tau_2} . t_1}
		{\tyarr[2][1][\theta_1][\rho_1]}{\bth}{\brh}}
	{\textsc{  (T-lam)}} \\[1cm]		
	
\dfrac
	{\typing{\lambda x_{\tau_2}. t}{\tyarr[2][1][\theta_1][\rho_1]}{\bth}{\brh}}
	{\typing{\mu f:~(\tau_1, \theta_1) .~
		\lambda x_{\tau_2}. t}
		{\tyarr[2][1][\theta_1][\textcolor{red}{\trh}]}
			{\bth}{\brh}}
	{\textsc{  (T-rec)}}\\[1cm]		
	
	\dfrac
	{
		\typing{v}{bool}{\bth}{\brh}
		\typing{t_1}{\tau_1}{\theta_1}{\rho_1} \qquad
		\typing{t_2}{\tau_1}{\theta_2}{\rho_2}
	}
	{	\typing{if~v~then~t_1~else~t_2}
		{\tau_1}{\theta_1 \bvee \theta_2}{\rho_1 \bvee \rho_2}}
	{\textsc{  (T-if)}}	\\[1cm]	
		
\dfrac
	{\typing{v}{int~list}{\bth}{\brh} \quad
	 \typing{t_1}{\tau}{\theta_1}{\rho_2}  \quad
	 \typing{t_2}{\tau}{\theta_2}{\rho_2}  }
	{\typing{match~v~with ~|~Nil~->~t_1~|~Cons~x_{int}~x_{int~list}~->t_2}{\tau}{\theta_1 \bvee \theta_2}
		{\rho_1 \bvee \rho_2}} 
{\textsc{  (T-match)}}
	
\end{array}
\end{displaymath}
\end{minipage} 
 \hfill 
\begin{minipage}[t]{0.49\linewidth}
\begin{displaymath}
\begin{array}{rrr}

\hspace*{-1.5cm}
\dfrac
	{
		\typing{t}{\tyarr[2][1][\theta_1][\rho_1]}{\theta_2}{\rho_2} \qquad
		\typing{v}{\tau_2}{\bth}{\brh}}
	{
		\typing{t~v}{\tau_1}{\theta_1 \bvee \theta_2}{\rho_1 \bvee \rho_2}}
	{
		\textsc{  (T-app)}} \\[1cm]	
\hspace*{-1cm}				
\dfrac
	{
		\typing{t_1}{\tau_1}{\theta_1}{\rho_1} \qquad
		\typing{t_2}{\tau_2}{\theta_2}{\rho_2}
	}
	{\typing
		{let~x_{\tau_1} = t_1~in~t_2}
		{\tau_2}
		{\theta_1 \bvee \theta_2}
		{\rho_1 \bvee \rho_2}}
	{\textsc{  (T-let)}} \\[1cm]	
					


\hspace*{1cm}		
\dfrac
	{\typing{v}{\tau}{\bth}{\brh}}
	{\typing{r_{\tau}~:=~v}
		{unit}
		{\textcolor{red}{\tth}}
		{\brh}} 			
 	{\textsc{  (T-assign)}}	\\[1cm]		
\hspace*{2.5cm}		   			
\dfrac
	{}
	{\typing{!r_{\tau}}{\tau}{\bth}{\brh}} 
{\textsc{  (T-deref)}}

\end{array}
\end{displaymath}
\end{minipage} 	 		 
\end{small}
\end{adjustwidth}
%\hrule
\caption{\textbf{Mini-ML Typing System With Effects}}
\label{mini-ml-def-typ}	
\end{figure}    
%	As we can see, the typing of Mini-ML is a monomorphic variant of ML-typing system$^{\cite{damas82popl}}$ enriched with a \textit{effect system} that keeps track of reference assignment and recursive function use inside each term.		

 
  Some of the cases above demands particular attention.
  Indeed, up to the moment when some application $t~v$ is simplified up to $(\lambda x. t_1 ~ v)$, the only observable side-effects are only those of $t$. However, after the next reduction step, $(\lambda x. t_1 ~ v)$ transforms into $t_1[x \mapsfrom v]$, so the side-effects of $t_1$ that were not visible before, become visible now.  
  
    Such side-effects produced by function's body are called \textit{latent effects}. 
    We represent by $\theta_1$ and $\rho_1$ above the function's arrow.
   	
   	At the moment we type a function itself, we mark these effects by $\theta_1$ and $\rho_1$ above the function's arrow:
  $$
  \dfrac
	{\typing{t_1}{\tau_1}{\theta_1}{\rho_1}}
	{\typing{\lambda x_{\tau_2} . t_1}
		{\tyarr[2][1][\theta_1][\rho_1]}{\bth}{\brh}}
	{\textsc{  (T-lam)}} $$	 
 
  It is when function definition is activated by some call, that we have to take latent effects into account: 
  $$\dfrac
	{
		\typing{t}{\tyarr[2][1][\textcolor{red}{\theta_1}][\textcolor{red}{\rho_1}]}{\theta_2}{\rho_2} \qquad
		\typing{v}{\tau_2}{\bth}{\brh}}
	{
		\typing{t~v}{\tau_1}{\textcolor{red}{\theta_1} \bvee \theta_2}{\textcolor{red}{\rho_1} \bvee \rho_2}}
	{
		\textsc{  (T-app)}} $$

However, the typing system of Mini-ML does not collect a detailed information about how many side-effects appear in terms, where they appear exactly, etc. It simply gives a rough side-effects' \textit{over-approximation} just to indicate the absence/presence of side effects. 
	For instance, in the rule \textsc{T-match}, 
$$\dfrac
	{\typing{v}{int~list}{\bth}{\brh} \quad
	 \typing{t_1}{\tau}{\theta_1}{\rho_2}  \quad
	 \typing{t_2}{\tau}{\theta_2}{\rho_2}  }
	{\typing{match~v~with ~|~Nil~->~t_1~|~Cons~x_{int}~x_{int~list}~->t_2}{\tau}{\theta_1 \bvee \theta_2}
		{\rho_1 \bvee \rho_2}} 
{\textsc{  (T-match)}}$$
	
	if suffices that only one of the terms $t_1$ and $t_2$ produces some side-effects, so that the if-expression itself is tracked as impure, even thought, the eventual evaluation of if-expression can go through the pure branch.
	However, despite the apparent simplicity of our system, it still suffices to establish the soundness of inlining, as we show in the work later. \\

	Another interesting detail of typing rules above is the absence of the environment $\Gamma$ that usually appears on left of $\vdash$ in the presentation of typing judgements. 
	We do not use $\Gamma$ because we assume that all scope issues are resolved before typing, so that typing relation is defined only for well-formed terms. 
	 Consequently, we can always check the validity of typing judgement, without using $\Gamma$, because all variables are explicitly typed.
	 
	 Finally, note that we can slightly simplify the explicit type of $\mu~f$ in the typing of recursive definitions:
$$\dfrac 
	{\typing{\lambda x_{\tau_2}. t}{\tyarr[2][1][\theta_1][\rho_1]}{\bth}{\brh}}
	{\typing{\mu f:~(\tau_1, \theta_1) .~
		\lambda x_{\tau_2}. t}
		{\tyarr[2][1][\theta_1][\textcolor{red}{\trh}]}
			{\bth}{\brh}}
	{\textsc{  (T-rec)}}$$
because, the argument's type $\tau_2$ and $\top_{\rho}$ can be deduced by typing system itself. \\

		
\subsection{Properties of Mini-ML Typing System with Effects}	

	The most basic property	one expect any type system to posses is \textit{type soundness}: "well-typed programs do not go wrong". 
	Type soundness guarantees that a well-typed term is either a value or it can take a step according to the evaluation rules (\textit{progress}); moreover, if it actually takes a step of evaluation, then the resulting term is still well typed (\textit{preservation}):

\begin{theorem}[(Progress of Mini-ML)] 
if $\vdash t : \tau, \theta, \rho$ holds, then either $t$ is a value, or
there is some term $t'$ and store $\mu_2$ such that 
$\evalstep{t}{1}{t'}{2}$
\end{theorem}

\begin{theorem}[(Preservation of Mini-ML)] 
if $\vdash t : \tau, \theta, \rho$ holds and $\evalstep{t}{1}{t'}{2}$, then 
$\vdash t' : \tau, \theta', \rho'$ holds for some $\theta'$, $\rho'$.
\end{theorem}
	
 Clearly, if we ignore $\theta$ and $\rho$ from typing rules above, we obtain a standard \textit{simple type system}, the soundness of which is well-established result{\footnotesize$^{ \cite[p.~190]{Pierce:2002:TPL:509043}}$}, so we do not have to check the theorems above for Mini-ML. 
 
 However, our typing system also keeps track of side-effects, so we can actually strengthen the \textit{preservation} theorem, by making explicit the relation between the effect indicator $\theta$  and store $\mu$.
 
 But we also need the strengthened version of \textit{substitution} lemma:
 \begin{lemma}[(Substitution Lemma, Strengthened)] ~ 
 \hypertarget{subst-lemma}{}
\label{subst-lemma}

%
%\item 
%If $\typing{t_1}{\tau_1}{\theta_1}{\rho_1}$ and $\typing{v}{\tyarr[2][3][\textcolor{red}{\bth}][\textcolor{black}{\rho_2}]}{\bth}{\brh}$, ~~then $\typing{t_1[x_{\tau_2 \Rightarrow \tau_3} \mapsfrom v]}{\tau_1}{\theta_1}{\rho_3}$~~for some $\rho_2$ and $\rho_3$. 

If $\typing{t_1}{\tau_1}{\theta_1}{\rho_1}$ and $\typing{v}{\tau_2}{\bth}{\brh}$, ~~then $\typing{t_1[x_{\tau_2} \mapsfrom v]}{\tau_1}{\theta_1}{\rho_2}$~~for some $\rho_1$ and $\rho_2$. 

\end{lemma}
\begin{proof}
  By straightforward induction on the typing derivation of $t_1$.
\end{proof}
 
\begin{theorem}[(Preservation of Mini-ML, Strengthened)] 
if $\vdash t : \tau, \textcolor{red}{\bth}, \rho$ holds and $\evalstep{t}{1}{t'}{2},$ then $\mu_1 = \mu_2$ and $\vdash t' : \tau, \textcolor{red}{\bth}, \rho'$ for some $\rho'$.
\label{preserv-prop-d}
\end{theorem}
\begin{proof}
  By induction on the derivation of $\vdash t : \tau, \textcolor{red}{\bth}, \rho$.
  For detailed proof, see the proof \hyperlink{proof:preserv-prop-p}{below}. 
\end{proof}

\paragraph{Normalisation}

Actually, we can strengthen the preservation theorem by imposing the same conditions on $\rho$:
\begin{theorem}[(Preservation of Mini-ML, Strengthened 2)] 
if $\vdash t : \tau, \textcolor{red}{\bth}, \textcolor{red}{\brh}$ holds and $\evalstep{t}{1}{t'}{2}$ then $\mu_1 = \mu_2$ and $\vdash t' : \tau, \textcolor{red}{\bth}, \textcolor{red}{\brh}$.
\label{preserv-prop2-d}
\end{theorem}
\begin{proof}
  Similar to the proof of the theorem above.
\end{proof}

An interesting consequence of this theorem is that the term $t$ is a \textbf{strongly normalising} expression.

This must be true because intuitively a non-terminating Mini-ML program is either a program with some malicious recursive call like 
$$ (\mu f. \lambda x. (f x)~0) \rightarrow (f x) [ f \mapsfrom \mu f. \lambda x. (f x), x \mapsfrom 0] = \mu f. \lambda x. (f x)~0 \rightarrow ... $$ 
or a program that simulates the recursion by reference assignment, using a trick known as \textit{Landin's Knot}{\footnotesize$^{\cite{Pierce:2002:TPL:509043}}$}:
\begin{whycode} 
 $\text{r}_{int}$ := $\lambda x_{int}. $ (x 0) 
 begin
   let foo = $\lambda y_{int}$. (!r y)  in 
   $\text{r}_{int}$ := foo ; !r 42 
 end \end{whycode}
In the program above, at the moment of the evaluation of (!r$_{int}$ 42); !r$_{int}$ is already equal to $\lambda y_{int}.$ (!r y), so the evaluation of (!r$_{int}$ 42) never stops. 

Formally, we have:
\begin{theorem}
 If $\vdash t : \tau, \textcolor{red}{\bth}, \textcolor{red}{\brh}$ then  $\evalstar{t{_1}}{1}{v{_2}}{1}$ for some value $v_2$.
\end{theorem}
\begin{proof}
  From \cref{preserv-prop2-d} it follows that during every step of evaluation, the effect indicators $\theta_{(t)}$ and $\rho_{(t)}$ are always equal to $\bth$ and $\brh$. 
  That means that no recursive call or reference assignment can ever appear during evaluation.
  In other words, $t$ is a  well-typed \textit{pure} term of simple-typed lambda-calculus. 
  But the normalisation of the calculus is a well-established 
  result{\footnotesize$^{\cite{Tait67}}$}. 
\end{proof}

\subsection{From Mini-ML to \texorpdfstring{$ML^2$}{$ML^2$}}

%\paragraph{\texorpdfstring{$ML^{^{n}}$}{} Programming Language}

	In this subsection, we impose some restrictions  on Mini-ML, so that the only programs we can write are those we can inline correctly.  


\paragraph{Ordering Mini-ML types}

	We have already seen in the introductory example \hyperlink{rec-bad-ex}{above} that recursive function modifying its functional arguments inside recursive calls cannot be inlined. 
%	However, if in each recursive call functional arguments coincide with the correspondent formal parameters, such recursive definition can be inlined correctly. 
	In other words, the only recursive functions we can inline directly, use their functional arguments without modifying them through the recursion.  
	For these functions, we push these arguments simply outside recursive definition.
	For instance,  we can transform 
	$$(1)~ \texttt{ rec } \text{apply } f_{int \rightarrow int}~x_{int} .
  \texttt{ if } x = 0 \texttt{ then } 42 \texttt{ else} \text{ apply }~ f~ (f~x) ,$$ into 
  $$(2)~ \lambda f_{int \rightarrow int}. \texttt{ rec } \text{apply } x_{int} .
  \texttt{ if } x = 0 \texttt{ then } 42 \texttt{ else} \text{ apply } (f~x)$$
  
 	Thus, a possible solution to eliminate undesirable recursive definitions, is to limit their use to first-order parameters only. \\
 
 	More generally, if we order the formal parameters of each higher-order definition, according to the degree of their type, we can make inlining of a program using these  easier to describe. 
  To do so, we order ML types as follows:
\begin{definition}[($ML^{^{n}}$)]
\begin{displaymath}
	\begin{array}{lll@{\hspace*{2cm}}l}
	\tau^0 & ::= & int~|~bool~|~unit~|~list~int~|~\dots~ & \textsc{base
          type} \\ \tau^1 & ::= & \tau^0 \stackrel{\theta, \rho}{\rightarrow}
        \tau^0 ~|~ \tau^0 \stackrel{\theta, \rho}{\rightarrow} \tau^1 &
        \textsc{first-order functions}\\ \tau^2 & ::= & \tau^1 \stackrel{\theta,
          \rho}{\rightarrow} \tau^0 ~|~ \tau^1 \stackrel{\theta,
          \rho}{\rightarrow} \tau^1 ~|~ \tau^1 \stackrel{\theta,
          \rho}{\rightarrow} \tau^2
%	\tau^1 \stackrel{\bot_{\theta}, \bot_{\rho}}{\rightarrow} \tau^2 
	& \textsc{second-order functions} \\ \dots & & \dots & \dots
        \\ \tau^{n+1} & ::= & \tau^{n} \stackrel{\theta, \rho}{\rightarrow}
        \tau^0 ~|~ \tau^{n} \stackrel{\theta, \rho}{\rightarrow} \tau^1 ~|~
        \dots ~|~ \tau^{n} \stackrel{\theta, \rho}{\rightarrow} \tau^{n} ~|~
        \tau^{n} \stackrel{\theta, \rho}{\rightarrow}\tau^{n+1}
%	\tau^1 \stackrel{\bot_{\theta}, \bot_{\rho}}{\rightarrow} \tau^2 
	& \textsc{higher-order functions} \\
	
	\end{array}
\end{displaymath}
\label{MLn-ty-d}

The ordering above is defined inductively, and for every $n \in \mathbb{N}$, we write $ML^{^{n}}$ to designate the set of well-typed programs, where each type belongs to $\bigcup_{~0 \leq n} \tau^i$.
% (by strong induction on type's degree
%$i$): if $i=0$, then the base case consist of all base types such as $int$,
%$bool$, etc; otherwise, for $i > 0$, the set of possible types of degree $i+1$
%is an enumeration $\{\tau^{i} \rightarrow \tau^{k} ~|~ k \in [ 0 \dots (i+1) ]
%\}$ of all possible codomain inferior degrees including $(i+1)$. 
\end{definition}

	We also admit that $ML^{^{n}}$ is isomorphic the set of Mini-ML types.
Intuitively, this is true, because in our presentation all types are simple, so that we can always order types of any well-typed program in Mini-ML to obtain a well-typed program in $ML^{^{n}}$. 
	Consequently, we admit that restraining the source language typing to $ML^{^{n}}$ does not reduce the expressiveness of Mini-ML.

	

%However, it is still can be proved that all definitions, theorems and proofs in the rest of this work can be extended without limiting $\mu$.
%
%Note that the set of $ML^{^{n}}$ types is totally ordered: indeed, for every
%couple of types $\tau^i \rightarrow \tau^k$ $\tau^j \rightarrow \tau^l$ we can
%compare them lexicographically starting by comparing their domain degrees $i$
%and $j$, and if those are equal, then by comparing their codomain degrees $k$
%and $l$.



So far, we defined $ML^{^{n}}$ types for an arbitrary degree $n$. 
In practice, however, the use of the most common high-order expressions such as $folders$, $iterators$, $mappings$ limits  to the first-order functions.
That is, most of the functional programs are essentially second-order ones. 

For that reason, and for the sake of simplicity, from now on we limit our presentation to $ML^{^{2}}$ where all programs are of second-order degree at most.

Still for the sake of simplicity, (especially for the proof inlining soundness), we assume that all reference assignment are done with to the values of degree zero, i.e. that the global store $\mu$ contains uniquely values of some zero-degree data like integers, lists, etc.

\paragraph{Restrictions on The Typing of \texorpdfstring{$ML^{^{2}}$}{}}

	To eliminate programs that we cannot inline correctly, we impose the restrictions on the typing system of $\inlsrc$.
 
The typing of variables, constants and mutable variable
manipulation remains the same as for $ML$:
\begin{footnotesize}
	\begin{multicols}{2}	
		\infrule[T$_{ML^{^2}}$-Var]
			{}
			{\vdash x_{\tau^{i}} : \tau^{i}, 
			\bot_{\theta},
  		\bot_{\rho} } 
  	\infrule[T$_{ML^{^2}}$-Const]
  	{\text{Typeof}(c) = \tau^i}
  	{\vdash c_{\tau^{i}} : \tau^{i}, \bot_{\theta}, \bot_{\rho}}
	\end{multicols}

	\begin{multicols}{2}	
		\infrule[T$_{ML^{^2}}$-Assign] 
		{\vdash v:\tau^{0}, \bot_{\theta}, \bot_{\rho}}
    {\vdash r_{\tau^{0}} := v : unit, \top_{\theta}, \bot_{\rho}}
	
		\infrule[T$_{ML^{^2}}$-Deref] 
			{} 
			{\vdash !r_{\tau^{0}} : \tau^{0}, \bot_{\theta},\bot_{\rho}}
	\end{multicols}
\end{footnotesize}

	Typing of function definitions and applications can remain unchanged too.
	Indeed, while inlining expressions such as
$$ (\lambda f_{int \rightarrow int}.t~ (r_{int}:=42; \lambda x_{int}. x))$$
would be incorrect, we took care to rule out these expressions by imposing $A-normal form$ constraint on the syntax. 
So, the typing of abstractions and applications remains: 

	\begin{footnotesize}
		\begin{multicols}{2}	
	\infrule[T$_{ML^{^2}}$-Fun] 
		{\vdash t : \tau^{j}, \theta, \rho} 
		{\vdash \lambda x_{t^i}. t : 
			\tau^{i} \stackrel{\theta, \rho}{\rightarrow} \tau^{j}, 
			\bot_{\theta},
 			\bot_{\rho}}

	\infrule[T$_{ML^{^2}}$-App] 
		{\vdash t: \tau^{i} \stackrel{\theta, \rho}{\rightarrow}
  	\tau^{j}, \theta', \rho' 
  	\qquad \vdash v:\tau^{i}, \bot_{\theta}, \bot_{\rho}}
    {\vdash (t~v) : \tau^{j}, (\theta \bvee \theta'), (\rho \bvee \rho')}
		\end{multicols}
	\end{footnotesize}

The interesting part is the typing of recursive functions and local
variables.  

The domain of the former's is limited to zero-order arguments only:
	\begin{footnotesize}
		\infrule[T$_{ML^{^2}}$-Rec] 
		{\vdash t : \tau^{\textcolor{red}{1}}, \theta, \rho} 
		{\vdash \mu f: (\tau\textcolor{red}{_1}, \theta) ~ \lambda x_{\tau\textcolor{red}{^0}}~. t : 
			\tau^{\textcolor{red}{0}} \stackrel{\theta, \top_{\rho}}{\Rightarrow}\tau^{\textcolor{red}{1}}, 
			\bot_{\theta}, 
			\bot_{\rho}}
	\end{footnotesize}
	
	
As to let expressions, obviously, if second-order term $t_2$ produces some side-effects inside "$\texttt{let } F_{\boldsymbol{i^{2}}} = t_2 \texttt{ in } t_1 $", we have no guarantee that these effects will be preserved correctly by inlining. 
So, we have to restrain term $t_2$ to be free from any possible side-effects:
\hypertarget{restr-let-f}{above}
	\begin{footnotesize}
		\infrule[T$_{ML^{^2}}$-Let] 
			{\vdash t : \tau^i, \theta, \rho 
			\qquad \vdash t' : \tau^j, \theta', \rho' 
			\qquad \textcolor{red}{(i = 2) \Rightarrow 
			\boldsymbol{(\theta = \bot_{\theta}\bwedge \rho = \bot_{\rho})}}} 
			{\vdash \text{ let } x_{\tau^i} = t \text{ in } t' 
				: \tau^{j}, 
				(\theta \bvee \theta'), 
				(\rho \bvee \rho')}
	\end{footnotesize}
\paragraph{Restricting Codomain Type}
	Finally, if $t$ is a $\inlsrc$ program of some second-order type $\tau^2$, we cannot  transform $t$ into some first-order program. 
	So it is naturally to assume whenever we talk about inlining a higher-order program $t$, $t$ itself is a program of some first-order type.

\paragraph{Notations}

	Finally, to make the rest of presentation more readable, we introduce the following notations:\\
 \hspace*{0.5cm} - $x,y,...$ denote variables of some zero-degree type: \texttt{int, bool, ...};\\
 \hspace*{0.5cm} - $f,g,...$ denote variables of some first-order type: \texttt{int -> int, int -> bool -> int, ...};\\ 
 \hspace*{0.5cm} - $F,G,...$ denote variables of some second-order type: \texttt{(int -> bool) -> int}, ....

\noindent	Also, we will write programs of $\inlsrc$ in the syntax of WhyML, like we did in the introduction. 
	However, the correspondence between $\inlsrc$ and a fragment of WhyML we use will be always straightforward and intuitive.
	
\newpage
\section{Higher Order Logic}

\subsection{\texorpdfstring{$WhyML^2$}{WhyML2}: Specification Language for \texorpdfstring{$ML^{^{2}}$}{ML2}}
%\subsubsection*{Why3ML axiomatic first-order logic}
%\subsubsection*{Higher-order logic}

\subsubsection*{Detailed Example}

In the introductory example of \texttt{sum\_iter}, we extend the definition of \texttt{array\_iter} with an additional parameter  \textcolor{blue}{\texttt{{inv:int -> int array -> prop}}} represents for loop invariant:
 
\begin{small}
\begin{whycode}  
   let array_iter ($\textcolor{black}{\text{f:int -> unit}}$) (b: int array) ($\textcolor{red}{\text{inv:int -> int array -> prop}}$)    
     $\textcolor{OliveGreen}{\text{requires \{ \textcolor{red}{\text{inv }} 0~b \}}}$
     $\textcolor{OliveGreen}{\text{ensures~~}\{ \text{ \textcolor{red}{inv} b.length~b }\}}$  
   = let rec loop (i: int)
       $\textcolor{OliveGreen}{ \text{requires \{~\textcolor{red}{inv} i b} \bwedge \text{0 <= i <= b.length}~\}}$
       $\textcolor{OliveGreen}{\text{ensures~~}\{ \text{ \textcolor{red}{inv} b.length~b}\}}$ =    
       = if i < b.length then ($\textcolor{black}{\text{f}}$ b[i]; loop (i+1)) 
     in loop 0 
  \end{whycode}
\end{small}



Let us illustrate this, by rewriting the example of \texttt{sum\_iter} with lists.
The source code of iterator itself is:
\begin{small}	
\begin{whycode}  
let list_iter ($\textcolor{black}{\text{f:int -> unit}}$) (l: int list)
 = let rec loop (ai: int list)   
    = match ai with 
       | Nil       -> ()
       | Cons x xs -> ($\textcolor{black}{\text{f}}$ x; loop ai); 
   in loop l  
 	\end{whycode}
 \end{small}	 
How do we represent the invariant of \texttt{list\_iter} ? As we see, the iteration on list \texttt{l0} relies on the structure of lists itself, using recursion and pattern-matching: at each step, we call \texttt{list\_array} recursively on the sub-list \texttt{xs} of \texttt{ai}.

Consequently, to prove the preservation of loop invariant at some step $i$, we need to show that the invariant property holds for all elements of \texttt{a\_i} iterated \textbf{before} the step $i$.

That means that we need to extend the definition of \texttt{list\_iter} by an addition parameter \texttt{bi: int list} to keep track of the list \texttt{ai} elements accessed before each recursive call: 

\begin{small}
	\begin{minipage}[t]{0.4\linewidth}	
	\begin{whycode}  
let list_iter  ($\textcolor{black}{\text{f:int -> unit}}$) (l: int list)
 = let rec loop (ai: int list) $(\textcolor{red}{\text{bi: int list}})$ = 
      match ai with 
       | Nil       -> ()
       | Cons x xs -> ($\textcolor{black}{\text{f}}$ x; loop xs $(\textcolor{red}{\text{Cons x bi}})$); 
   in loop 0 l Nil
 	\end{whycode}
 	\end{minipage}
 \end{small}	 
 
 First, let us check that when all elements of \texttt{l} have been accessed, the elements of \texttt{bi} and  \texttt{l} are exactly the same. It is true because:
 
	- when the auxiliary function \texttt{loop} is called first time, none of elements of \texttt{ai} is accessed yet, so \texttt{bi} is equal \texttt{Nil}; 
	
	- before each recursive call of \texttt{loop}, the element \texttt{x}  must be added to \texttt{bi} as already seen, so we apply \texttt{Cons} to \texttt{x} and \texttt{bi}; 
	
	- after the last recursive call, the elements of \texttt{bi} are  exactly those of the initial list \texttt{l}. \\

However, \texttt{l} and \texttt{bi} are not \textit{structurally} the same: the order of elements is \textbf{reversed}. We must take that into account, when we apply the invariant \texttt{inv: int list -> int list -> prop} on $bi$ and $l$. Consequently, the specification of \texttt{list\_iter} is: 
\hypertarget{list-iter}{example above}

\begin{small}
	\begin{minipage}[t]{0.4\linewidth}	
	\begin{whycode}  
let list_iter ($\textcolor{red}{\text{inv:int list -> int list -> unit}}$) ($\textcolor{red}{\text{f:int -> unit}}$) (l: int list)
     $\textcolor{OliveGreen}{\text{requires \{ \textcolor{red}{\text{inv }} Nil~l \}}}$
     $\textcolor{OliveGreen}{\text{ensures~~}\{ \text{ \textcolor{red}{inv} (reverse l)~Nil
     	 }\}}$    
 = let rec loop (a: int list) $(\textcolor{red}{\text{b: int list}})$ = 
 ($\textcolor{red}{\text{f:int -> unit}}$) (l: int list)
     $\textcolor{OliveGreen}{\text{requires \{ \textcolor{red}{\text{inv }} Nil~l \}}}$
     $\textcolor{OliveGreen}{\text{ensures~~}\{ \text{ \textcolor{red}{inv} (rev\_append l b)~Nil
     	 }\}}$  
      match a with 
       | Nil       -> ()
       | Cons x xs -> ($\textcolor{red}{\text{f}}$ x; loop xs $(\textcolor{red}{\text{Cons x b}})$); 
   in loop 0 l Nil
 	\end{whycode}
 	\end{minipage}
 \end{small}	 

As we can see, the original order of elements is restored in specification of \texttt{list\_iter}  by reversing the entire list \texttt{l}; in specification of \texttt{loop} by concatenating the accessed elements \texttt{bi} in the reversed order to list \texttt{li} of the remaining elements of \texttt{l}.

%
%\begin{small}
%	\begin{minipage}[t]{0.4\linewidth}	
%	\begin{whycode}  
%let list_iter ($\textcolor{red}{\text{inv:int list -> int list -> unit}}$) ($\textcolor{red}{\text{f:int -> unit}}$) (l: int list)
%     $\textcolor{OliveGreen}{\text{requires \{ \textcolor{red}{\text{inv }} Nil~l \}}}$
%     $\textcolor{OliveGreen}{\text{ensures~~}\{ \text{ \textcolor{red}{inv} (reverse l)~Nil
%     	 }\}}$  
%   
% = let rec loop (a: int list) $(\textcolor{red}{\text{b: int list}})$ = 
% ($\textcolor{red}{\text{f:int -> unit}}$) (l: int list)
%     $\textcolor{OliveGreen}{\text{requires \{ \textcolor{red}{\text{inv }} Nil~l \}}}$
%     $\textcolor{OliveGreen}{\text{ensures~~}\{ \text{ \textcolor{red}{inv} (rev\_append l b)~Nil
%     	 }\}}$  
%      match a with 
%       | Nil       -> ()
%       | Cons x xs -> ($\textcolor{red}{\text{f}}$ x; loop xs $(\textcolor{red}{\text{Cons x b}})$); 
%   in loop 0 l Nil
%   
%   let sum_list_iter (a: array int) =		 
%     let s = ref 0 in
%     list_iter $\textcolor{red}{\text{(fun x -> s := !s + x)}}$ a     
% 	\end{whycode}
% 	\end{minipage}
% \end{small}	 

%
% let sum_iter (a: array int) =		 
%   let s = ref 0 in 
%   array_iter $\textcolor{red}{\text{(fun x -> s := !s + x)}}~$ a $\textcolor{red}{(\lambda i_{int}.\lambda c_{\texttt{int array}}. \sum_{~0\leq j < i} a[j])}$                    
% 	
	
	
%%	Let \texttt{P} be a program of $\inlsrc$ and \texttt{L} the logical expressions of $WhyML^2$ used to specify and annotate \texttt{P}. 
%	Imagine that instead of writing a second-order program \texttt{M} in $\inlsrc$, we write  directly it in $WhyML^2$ as a specified and annotated program \texttt{W}. Suppose that we inlined \texttt{W} into a program $I_2(I_1((W))$,  and proved all generated verification conditions in Why3. Now,
		
	
		
	



\subsection{Ghost Code}

In the example above, we extended the definition of \texttt{list\_iter} with \texttt{bi}, the list of elements of \texttt{l} visited before each recursive call.

Fundamentally, the list \texttt{bi} is a part of specification.
However, contrary to the invariant \texttt{inv}, we used \texttt{bi} inside the \textit{source code} of \texttt{list\_iter} itself, mixing in some way specification  the source code.  

What if we want to restore the initial program back from the specifications: 
how do we recognize which term is specification, and which one is the executable code?
Clearly, in the example above, nothing indicates automatically that list \texttt{bi} is a part of specification, and \texttt{li} is a part of the source code.

On the other hand, the verifications conditions generated from the specification above, take into account list parameter \texttt{bi}:
How can we be sure that when we modify the body of \texttt{list\_iter} by adding  \texttt{bi}, we do not modify the semantics of the function itself ? 
Again, at the moment, nothing, a part of the simplicity of the example, guarantees that \texttt{list\_iter} with \texttt{bi} is equivalent to the initial definition of \texttt{list\_iter}. \\ 

In some modern verification systems, notably in Why3, a way to solve these problems is to use \textit{ghost-code}.
 Intuitively, \textit{ghost code} is a part of a program that is never \textit{executed}, never \textit{interferes} with the executable code, but helps to bring the source code closer to the intended specification.

%Also, the manual annotation of \textit{ghost code} should be traceable by type system: tools in which \textit{ghost code} is used, can verify the \textit{non-interference} of \textit{ghost code} with the rest of the program. \\

%Finally, there should be an way to extract the source executable code, erasing every ghost-expression inside it. For instance, such erasure may be defined by simply replacing every ghost term by unit value (). \\

In this subsection we show how to use ghost-code in MiniML. Concretely, we describe Ghost-ML, a version of MiniML in which one can write ghost code inside programs of Mini-ML, using explicit ghost-code annotation.

We explain in details how to extend Mini-ML typing system to guarantees ghost-code \textit{non-interference} and how to extract executable-code soundly from a program containing ghost-code.


\subsubsection*{Syntax of Ghost-ML}
To annotate term as ghost,  we need to modify of Mini-ML syntax by\\

(1) marking explicitly with the ghost indicator $\beta$ every variable, reference,  formal parameter, and domain type: 
$$x_\tau^{\rouge{\beta}},~ \quad !r^{\rouge{\beta}}_{\tau},~ \quad \lambda x_\tau^{\rouge{\beta}}. t,~ \quad {\tau^{\rouge{\beta}} \Rightarrow \tau}...;~$$

(2) adding a key-word \texttt{\rouge{ghost}} to annotate the ghost-code itself: 
$$\rouge{ghost}~x^{\rouge{\beta}},~~~\rouge{ghost}~(if~v~ then~ t_1~ else~ t_2),~ ...;~$$

The indicator $\gbr$ above is either true (variable or term is ghost) or false (variable or term belongs to the source code).

\subsubsection*{Semantics of Ghost-ML}

The semantics of Ghost-ML remains essentially the same as for Mini-ML, except the new reduction rule for ghost expression:
$$\dfrac{}{\rouge{ghost}~ t_\mu~ \rightarrow~ t_\mu}{\textsc{T-Ghost}}$$
That is, the semantics of ghost-ML simply \textit{ignores} all ghost annotations (as it ignores type annotations for Mini-ML).

\subsubsection*{Ghost-ML Type System}

The type system for Ghost-ML guarantees not-only that "well-typed terms do not go wrong", but also give us control on ghost code structural and applicative \textbf{\textit{propagation}} inside the program, and allows us to check  \textbf{\textit{non-interference}} of ghost code with the rest of the program.   

	Each typing rule is now of the form $\vdash_gh t : \tau, \theta, \rho, \rouge{\beta}$, where $\rouge{\beta}$ is ghost-indicator, $\tau$ stands for term's type, but where $\theta$ and $\rho$ indicate now the presence/absence of \textbf{non-ghost} reference assignments, and \textbf{non-ghost} recursive calls inside $t$.

That is, we should type a term like $r_{int}^{\gbtr} := 42 $ as 
$\ghosttyping{r_{int}^{\gbtr} := 42}{()}{\bth}{\brh}{\gbtr}{}$.

\paragraph*{The Invariant of Typing Relation}	
	
If a judgement $\vdash_{gh} t : \tau, (\theta, \rho, \gbtr)$, with $\theta = \tth$ or $\rho = \trh $, were valid, it would mean that $t$ is simultaneously a ghost term, and that is may produce some visible side-effects.

To take into account the side-effects of \textit{non-ghost} terms only, we impose the following invariant:

\begin{definition}[(Ghost-ML Typing Invariant)]
 $$\vdash_{gh} t : \tau, (\theta, \rho, \gbr) 
 ~~ \iff ~~
\models (\gbr = \gbtr) \Rightarrow (\theta = \bth \bwedge \rho = \brh) $$
\end{definition}   

From now on, we assume that this invariant above is verified for every typing rule conclusion and premises.  

\paragraph*{Typing Rules}	

The whole set typing rules of ML-typing is given in the \ref{ghost-mini-ml-def-typ}.
Below we explain each rule separately, to show what properties of ghost-code it reflects. \\

The rules for variables, constants, reference access and ghost annotation itself are straightforward:  

\begin{footnotesize}
\begin{minipage}[t]{0.5\linewidth}
\begin{displaymath}
\begin{array}{l}
\dfrac
	{}
	{\ghosttyping{x^{\gbr}_{\tau}}{\tau}{\bot_{\theta}}{\bot_{\rho}}{\gbr}{}}
	{\textsc{  (T-var)}} \\[1cm]	
	
\dfrac
	{}
	{\ghosttyping{!r^{\gbra{1}}_{\tau}}{\tau}{\bth}{\brh}{\gbra{1}}{}} 
{\textsc{  (T-deref)}}	
\end{array}
\end{displaymath}
\end{minipage}
\begin{minipage}[t]{0.5\linewidth}
\begin{displaymath}
\begin{array}{l}
\dfrac
	{\ghosttyping
		{t}
		{\tau}
		{\theta}
		{\rho}
		{\gbr}
		{}}
	{\ghosttyping
		{ghost~t}
		{\tau}
		{\bth}
		{\brh}
		{\gbtr}
		{}} 
{\textsc{  (T-ghost)}}\\[1cm]		

\dfrac
	{\textsc{Typeof}(c) = \tau}
	{\ghosttyping{c}{\tau}{\bot_{\theta}}{\bot_{\rho}}{\gbbr}{}}
	{\textsc{  (T-const)}} 	

\end{array}	
\end{displaymath}
\end{minipage}
\end{footnotesize}	
~\\
%\begin{footnotesize}	
%\begin{displaymath}
%\begin{array}{l}
%\end{array}	
%\end{displaymath}
%\end{footnotesize}	

The typing of \texttt{if~then~else} and \texttt{match~with} constructions shows
that in compound expressions, if only one branch is ghost, then the entire expression becomes ghost itself:
\begin{footnotesize}
\begin{displaymath}
\begin{array}{l}
	\dfrac
	{
		\ghosttyping{v}{bool}{\bth}{\brh}{\gbra{0}}{} ~~
		\ghosttyping{t_1}{\tau_1}{\theta_1}{\rho_1}{\gbra{1}}{} \qquad
		\ghosttyping{t_2}{\tau_1}{\theta_2}{\rho_2}{\gbra{2}}{}
	}
	{	\ghosttyping{if~v~then~t_1~else~t_2}
		{\tau_1}{\theta_1 \bvee \theta_2}{\rho_1 \bvee \rho_2}
		{\gbra{0} \bvee \gbra{1} \bvee \gbra{2}}{}}
	{\textsc{  (T-if)}}	\\[1.3cm]	
		
\dfrac
	{\ghosttyping{v}{int~list}{\bth}{\brh}{\gbra{0}}{} ~~
	 \ghosttyping{t_1}{\tau}{\theta_1}{\rho_2}{\gbra{1}}{}  ~~
	 \ghosttyping{t_2}{\tau}{\theta_2}{\rho_2}{\gbra{2}}{}  }
	{\ghosttyping{match~v~with ~|~Nil~->~t_1~|~Cons~x_{int}~x_{int~list}~->t_2}{\tau}{\theta_1 \bvee \theta_2}
		{\rho_1 \bvee \rho_2}
		{\gbra{0} \bvee \gbra{1} \bvee \gbra{2}}{}} 
{\textsc{  (T-match)}}
\end{array}	
\end{displaymath}
\end{footnotesize}

~\\
The typing of \texttt{let} expression in more intricate.
Suppose we want to type some term of the form $let~x^{\gbra{1}}_{\tau_1} = t_1~in~t_2$, where we know for $t_2$ that $t_2:\gbbr$.
In that case
\begin{itemize}
\item[(1)] we cannot bind $(t_1: \gbbr)$ by a variable $x_{\tau}^{\gbtr}$, because otherwise we must mark $t_1$ inside $t_2$ as ghost expression; 
\item[(2)] we can bind $(t_1: \gbtr)$ to $x_{\tau}^{\gbbr}$, but in that case, we have  consider the case when $x_{\tau}^{\gbbr}$ occurs in $t_2$. 
For instance, if $t_2 = x_{\tau}^{\gbbr}$ itself, we have to mark the entire expression as ghost, even thought $x_{\tau}^{\gbbr}$ and $ t_2$ were marked as non-ghost.   
\end{itemize}
A simple way to express conditions (1) (2) inside the typing of \texttt{let} is then: 
\begin{footnotesize}	
\begin{displaymath}
\begin{array}{l}
\dfrac
	{
		\ghosttyping
			{t_1}
			{\tau_1}
			{\theta_1}
			{\rho_1}
			{\gbra{1}^{\rouge{'}}}
			{} \qquad
		\ghosttyping
			{t_2}
			{\tau_2}
			{\theta_2}
			{\rho_2}
			{\gbra{2}}
			{  \quad \models \gba{1} \Rightarrow \gba{1}^{'}}
	}
	{\ghosttyping
		{let~x^{\gbra{1}}_{\tau_1} = t_1~in~t_2}
		{\tau_2}
		{\theta_1 \bvee \theta_2}
		{\rho_1 \bvee \rho_2}
		{\gbra{2} \bvee (\gbran{1} \bwedge \gbra{1}^{\rouge{'}})}
		{}}
	{\textsc{  (T-let)}}
	\end{array}	
\end{displaymath}
\end{footnotesize}	
~\\
The typing of lambda-abstractions shows that the ghost status of abstraction depends on the ghost status of its body:
\begin{footnotesize}	
\begin{displaymath}
\begin{array}{l}
\dfrac
	{\ghosttyping
		{t}
		{\tau_1}
		{\theta_1}
		{\rho_1}
		{\gbra{1}}
		{}}
	{\ghosttyping
		{\lambda x^{\gbra{2}}_{\tau_2} . t_1}
		{\tau_{2}^{\gbra{2}} \stackrel{\theta_1, \rho_1}{\Longrightarrow} \tau_1}
		{\bth}
		{\brh}
		{\gbra{1}}
		{}}
	{\textsc{  (T-lam)}}
\end{array}	
\end{displaymath}
\end{footnotesize}	
This rule gives us all four possibilities to :
\begin{itemize}
\item[(5)] write a Mini-ML function as \textbf{nont-ghost} function with \textbf{non-ghost} formal parameter; 

\item[(6,7)] modularize ghost code using \textbf{ghost} functions with \textbf{ghost} or \textbf{non-ghost} formal parameter; 

\item[(8)] extend, as in the \hyperlink{list-iter}{example above}, the definition of \textbf{non-ghost} function with \textbf{ghost} formal parameter.
\end{itemize}
~\\

Suppose now that we want to type the application of a term 
$\ghosttyping
			{t}
			{{\tau_{2}^{\gbra{2}} \stackrel{\theta_1, \rho_1}{\Longrightarrow} \tau_1}}
			{\theta_1}
			{\rho_1}
			{\gbra{1}}
			{}$ to a  value 
$\ghosttyping
			{v}
			{\tau_2}
			{\theta_2}
			{\rho_2}
			{\gbra{2}^{\rouge{'}}}
			{}$
To understand how to type $(t~v)$, let us first enumerate all possible cases for 
$(\gbra{1}, \gbra{2}, \gbra{2}^{\rouge{'}})$:


\begin{figure}[H]
\begin{small}
\begin{center}
\begin{tabular}{|c|c|c|c|l|}
\hline
\multicolumn{5}{|c|} 
 {$\ghosttyping
			{t}
			{{\tau_{2}^{\gbra{2}} \stackrel{\theta_1, \rho_1}{\Longrightarrow} \tau_1}}
			{\theta_1}
			{\rho_1}
			{\gbra{1}}
			{}$ \quad
		$\ghosttyping
			{v}
			{\tau_2}
			{\theta_2}
			{\rho_2}
			{\gbra{2}^{\rouge{'}}}
			{}$ 
			}
 			
			\\
\hline
$\gbra{2} \text{(parameter)}$ & $\gbra{1} \text{(body)}$ & $\gbra{2}^{\rouge{'}} \text{(argument)}$
 &  \multicolumn{2}{c|}{resulting ghost status} \\
\hline\hline
$\bot$ & $\bot$ & $\bot$ & $\bot$ &  pure source code application \\ \hline
$\top$ & $\bot$ & $\top$ & $\bot$ &  ghost-code passing inside source code
\\ \hline
$\top$ & $\top$ & $\top$ & $\top$ &  pure ghost code application \\ \hline
$\bot$ & $\top$ & $\bot$ & $\top$ &  source code passing inside ghost code
 \\ \hline

 $\bot$ & $\top$ & $\top$ & $\top$ &source code passing inside ghost code \\ \hline
 
 $\bot$ & $\bot$ & $\top$ & $\top$ &function parameter and body contamination \\ \hline
  $\top$ & $\bot$ & $\bot$ & $-$ & impossible \\ \hline
 $\top$ & $\top$ & $\bot$ & $-$ & impossible \\ \hline
\end{tabular}
\end{center}
\caption{Ghost-Code Propagation}
\end{small}
\end{figure}  



\begin{figure}[H]
%\hrule
\begin{adjustwidth}{-5em}{5em}
\begin{small}
\begin{minipage}[t]{0.45\linewidth}
\begin{displaymath}
\begin{array}{l}

\dfrac
	{}
	{\ghosttyping{x^{\gbr}_{\tau}}{\tau}{\bot_{\theta}}{\bot_{\rho}}{\gbr}{}}
	{\textsc{  (T-var)}} \\[1.3cm]	

\dfrac
	{\textsc{Typeof}(c) = \tau}
	{\ghosttyping{c}{\tau}{\bot_{\theta}}{\bot_{\rho}}{\gbbr}{}}
	{\textsc{  (T-const)}} \\[1.3cm]	


\dfrac
	{\ghosttyping
		{t}
		{\tau}
		{\theta}
		{\rho}
		{\gbr}
		{}}
	{\ghosttyping
		{ghost~t}
		{\tau}
		{\bth}
		{\brh}
		{\gbtr}
		{}} 
{\textsc{  (T-ghost)}}	\\[1.3cm]		
	
\dfrac
	{\ghosttyping
		{t}
		{\tau_1}
		{\theta_1}
		{\rho_1}
		{\gbra{1}}
		{}}
	{\ghosttyping
		{\lambda x^{\gbra{2}}_{\tau_2} . t_1}
		{\tau_{2}^{\gbra{2}} \stackrel{\theta_1, \rho_1}{\Longrightarrow} \tau_1}
		{\bth}
		{\brh}
		{\gbra{1}}
		{}}
	{\textsc{  (T-lam)}}\\[1.3cm]			
	
\dfrac
	{\ghosttyping
		{t_1}
		{\tau_1}
		{\theta_1}
		{\rho_1}
		{\gbra{1}}
		{\quad \gba{1} \Rightarrow \beta_1'}}
	{\ghosttyping
		{\mu f^{\gbra{1}^{\rouge{'}}}:~(\tau_1, \theta_1).~\lambda x^{\gbra{2}}_{\tau_2}. t} 
		{\tau_{2}^{\gbra{2}} \stackrel{\theta_1, \gbran{1}^{\rouge{'}}}{\Longrightarrow} \tau_1}
		{\bth}
		{\brh}
		{\gbra{1}^{\rouge{'}}}
		{}}
	{\textsc{  (T-rec)}}\\[1.3cm]			
	
	\dfrac
	{
		\ghosttyping{v}{bool}{\bth}{\brh}{\gbra{0}}{} ~~
		\ghosttyping{t_1}{\tau_1}{\theta_1}{\rho_1}{\gbra{1}}{} \qquad
		\ghosttyping{t_2}{\tau_1}{\theta_2}{\rho_2}{\gbra{2}}{}
	}
	{	\ghosttyping{if~v~then~t_1~else~t_2}
		{\tau_1}{\theta_1 \bvee \theta_2}{\rho_1 \bvee \rho_2}
		{\gbra{0} \bvee \gbra{1} \bvee \gbra{2}}{}}
	{\textsc{  (T-if)}}	\\[1.3cm]	
		
\dfrac
	{\ghosttyping{v}{int~list}{\bth}{\brh}{\gbra{0}}{} ~~
	 \ghosttyping{t_1}{\tau}{\theta_1}{\rho_2}{\gbra{1}}{}  ~~
	 \ghosttyping{t_2}{\tau}{\theta_2}{\rho_2}{\gbra{2}}{}  }
	{\ghosttyping{match~v~with ~|~Nil~->~t_1~|~Cons~x_{int}~x_{int~list}~->t_2}{\tau}{\theta_1 \bvee \theta_2}
		{\rho_1 \bvee \rho_2}
		{\gbra{0} \bvee \gbra{1} \bvee \gbra{2}}{}} 
{\textsc{  (T-match)}}
	
\end{array}
\end{displaymath}
\end{minipage} 
 \hfill 
\begin{minipage}[t]{0.55\linewidth}
\begin{displaymath}
\begin{array}{rrr}
\hspace*{-3cm}
\dfrac
	{
		\ghosttyping
			{t}
			{{\tau_{2}^{\gbra{2}} \stackrel{\theta_1, \rho_1}{\Longrightarrow} \tau_1}}
			{\theta_2}
			{\rho_2}
			{\gbra{1}}
			{} \qquad
		\ghosttyping
		{v}
		{\tau_2}
		{\bth}
		{\brh}
		{\gbra{2}^{\rouge{'}}}
		{\quad \gba{2} \Rightarrow \gba{2}^{'}}}
	{
		\ghosttyping
			{t~v}
			{\tau_1}
			{\theta_1 \bvee \theta_2}
			{\rho_1 \bvee \rho_2}
			{\gbra{1} \bvee (\gbran{2} \bwedge \gbra{2}^{\rouge{'}})}
			{}}
	{
		\textsc{  (T-app)}} \\[1.3cm]		
		
\dfrac
	{
		\ghosttyping
			{t_1}
			{\tau_1}
			{\theta_1}
			{\rho_1}
			{\gbra{1}^{\rouge{'}}}
			{} \qquad
		\ghosttyping
			{t_2}
			{\tau_2}
			{\theta_2}
			{\rho_2}
			{\gbra{2}}
			{\quad \gba{1} \Rightarrow \gba{1}^{'}}
	}
	{\ghosttyping
		{let~x^{\gbra{1}}_{\tau_1} = t_1~in~t_2}
		{\tau_2}
		{\theta_1 \bvee \theta_2}
		{\rho_1 \bvee \rho_2}
		{\gbra{2} \bvee (\gbran{1} \bwedge \gbra{1}^{\rouge{'}})}
		{}}
	{\textsc{  (T-let)}}\\[1.3cm]



\dfrac
	{\ghosttyping
		{v}
		{\tau}
		{\bth}
		{\brh}
		{\gbra{1}}
		{\quad \gba{1} \Rightarrow \gba{1}^{{'}}}}
	{\ghosttyping
		{r^{\gbra{1}^{\rouge{'}}}_{\tau}~:=~v}
		{unit}
		{\gbran{1}^{\rouge{'}}}	
		{\brh}
		{\gbra{1}^{\rouge{'}} }
		{}} 			
 	{\textsc{  (T-assign)}}	\\[1.5cm]		
   			
\dfrac
	{}
	{\ghosttyping{!r^{\gbra{1}}_{\tau}}{\tau}{\bth}{\brh}{\gbra{1}}{}} 
{\textsc{  (T-deref)}}


\end{array}
\end{displaymath}
\end{minipage} 	 		 
\end{small}
\end{adjustwidth}
%\hrule
\caption{\textbf{Mini-ML Typing System With Effects}}
\label{ghost-mini-ml-def-typ}	
\end{figure} 
\subsection*{Ghost-Code Erasure}


\subsection*{Properties of Erasure}


\subsection{Inlining of \texorpdfstring{$WhyML^{^{2}}$}{$WhyML^2$} programs}

Orthogonality


%%%%%%%%%%%%%%%%%%%%%%%%%%%%%%%%%%%%%%%%%%%%%%%%%%%%%%%%%%%%%%%%%%%%%%%%%%%%%%%
\newpage
\section{Inlining}

 Previously we defined $\inlsrc$ as the domain language of inlining procedure. 
 In this section we give a formal definition for the inlining procedure itself.
 We present inlining as a \textit{small-step} rewriting strategy described by a set of inference rules.
 
 We split our presentation in two consecutive steps: first, we define the inlining of second-order variables; then we define the inlining of second-order applications.
 For each of these steps we state and prove that it corresponds to a \textit{deterministic} procedure, always \textit{terminates} and results in an entirely first-order program of $\inlsrc$.
 
	Then, we conclude the formal presentation of inlining procedure by showing its \textit{total correctness}.\\

	
%
%With these notations, the syntax of $\inlsrc$ can be represented as follows:
%\begin{spacing}{1.3}
%\begin{displaymath}
%	\begin{array}{lll@{\hspace*{3cm}}l}
%			L &::=& F ~|~ l \qquad \text{with }  l~::=~ x ~|~ f &
%        \textsc{binders} \\ 
%      t &::=&  v~|~(t~v) ~|~\text{let } L = t 
%        \text{ in } t ~|~ r := v ~|~ !r & 
%        \textsc{terms}\\ 
%      v &::= & L ~|~ \lambda l. t ~|~
%        \text{rec } f~x :(\tau_1, \theta). t ~|~ c  & 
%        \textsc{values} \\ 
%      c &::=&
%        \text{()} ~|~ \mathbb{N}~|~\mathbb{B}~|~+~|~-~|~\bwedge~| ~\dots &
%        \textsc{constants} \\
%	\end{array}
%\end{displaymath}
%\end{spacing}


\subsection{Inlining Second-Order Local Variables}
%%%%%%%%%%%%%%%%%%%%%%%%%%%%%%%%%%%%%%%%%%%%%%%%%%%%%%%%%%%%%%%%%%%%%%%%%%%%%%%
	Our first goal is to inline second order binders $F,G,...$ inside any $\inlsrc$ program \texttt{P}. 
	Intuitively, this is possible, because by definition \texttt{P} is a well-typed closed term of some first-order type.
	Si, variables $F,G,...$ cannot be neither bound by some function, neither free in \texttt{P}. 
	Consequently, the only higher-order variables that can possibly occur in \texttt{P} are introduced by local binding expression $\tmlet{F}{t_1}{t_2}$.
	That is, our goal is to define inlining of second order variables as a function  $\inlletstar : \inlsrc \longrightarrow S$, where $S$ is a subset of $\inlsrc$ such that 
	 for $s \in S$ no second-order variables occur any more inside $s$. \\
	 
	We define $\inlletstar$ as a reflexive, transitive closure of the \textit{small-step rewriting strategy} $\inlletarr$ below: 
\begin{definition}[(One Step local bindings Inlining, $\hookdownarrow$)]
	Let $t$ be a term of $\inlsrc$ such that $t \not\in \inlS$. 
  Then, $t$ contains at least one occurence of some variable $F$ introduced by $\tmlet{F}{t_0}{t_1}$ construct.

	In that case, a step of inlining, $\inllet{t}{t'}$ is defined by the set of inference rules below:
%\cref{fig:inl-let-d}.

\label{inlletrule-macro}
\newcommandx{\inlletrule}[5]
{\infrule[$I_1$-#1]
	{\inllet{\boldsymbol{#2}}{\boldsymbol{#3}}}
	{\inllet{#4}{#5}}}
	
\label{inllettrulet-macro}	
	\newcommandx{\inllettrule}[5]
{\infrule[$I_1$-#1]
	{\inllett{\boldsymbol{#2}}{\boldsymbol{#3}}}
	{\inllett{#4}{#5}}}


 	\begin{figure}[H]
  \begin{footnotesize}
	\begin{spacing}{1.01}
	\hrulefill
	\begin{adjustwidth}{10em}{-17em}
		\begin{multicols}{2}
		\infrule[$I_1$-Let$_0$]
		{ s_1, s_2  \in S}
			{\boldsymbol{\tmlet{F}{s_1}{s_2} \hookdownarrow 				
			\tmsbst{s_2}{F}{s_1}}}
		\end{multicols}
		\vspace*{-2em}
	\end{adjustwidth}
	\begin{adjustwidth}{-5em}{-2em}
		\begin{multicols}{2}	
		\inlletrule{Fun}
			{t_1}{t'_1}
 			{\lambda l. \boldsymbol{t_1}}{\lambda l.\boldsymbol{t'_1}}
		\inlletrule{App$_1$}
			{t_1}{t'_1}
			{(\boldsymbol{t_1}v_2)}{(\boldsymbol{t'_1}v_2)}
		\inlletrule{Let$_1$}
			{t_1}{t'_1}
			{\tmlet{\varslash{l}{F}}{\boldsymbol{t_1}}{t_2}}
			{\tmlet{\varslash{l}{F}}{\boldsymbol{t'_1}}{t_2}}
			
		\infrule[$I_1$-If$_1$]
		{ \boldsymbol{t_1 \hookdownarrow t_1'} }
			{ if~v~then~\boldsymbol{t_1}~else~t_2 \hookdownarrow
				if~v~then~\boldsymbol{t_1'}~else~t_2 }					
			
	\infrule[$I_1$-Match$_1$]
		{ \boldsymbol{t_1 \hookdownarrow t_1'} }
			{~~~~~ match~v~with~Nil~->~ \boldsymbol{t_1}~|~Cons~x~y -> t_2 \hookdownarrow \\ 
				match~v~with~Nil~->~ \boldsymbol{t_1'}~|~Cons~x~y -> t_2}					
			
		\inlletrule{Rec}{t_1}{t'_1}
 			{\tmrec{f}{x}{(\tau,\theta)}{\boldsymbol{t_1}}}
 			{\tmrec{f}{x}{(\tau,\theta)}{\boldsymbol{t'_1}}}
 			
		\infrule[$I_1$-App$_2$]
		{ \boldsymbol{v_2 \hookdownarrow v_2'} \quad s_1 \in S }
			{(s_1\boldsymbol{v_2}) \hookdownarrow (s_1\boldsymbol{v_2'})}		

		\infrule[$I_1$-Let$_2$]
		{ \boldsymbol{t_2 \hookdownarrow t_2'} \quad s_1 \in S }
			{\tmlet{F}{s_1}{\boldsymbol{t_2}} \hookdownarrow \tmlet{F}{s_1}{\boldsymbol{t_2'}} }		

		\infrule[$I_1$-If$_2$]
		{ \boldsymbol{t_2 \hookdownarrow t_2'}  \quad s_1 \in S }
			{ if~v~then~s_1~else~\boldsymbol{t_2} \hookdownarrow
				if~v~then~s_1~else~\boldsymbol{t_2'} }	
				
		\infrule[$I_1$-Match$_2$]
		{ \boldsymbol{t_2 \hookdownarrow t_2'}  \quad s_1 \in S }
			{~~~~~match~v~with~Nil~->~s_1~|~Cons~x~y -> \boldsymbol{t_2} \hookdownarrow \\ 
				match~v~with~Nil~->~s_1~|~Cons~x~y -> \boldsymbol{t_2'}}					

		\end{multicols}
	\end{adjustwidth}	
	\hrulefill
	\end{spacing}
	\caption{ \textbf{Inlining of Second-Order Local Bindings}\hfill}
 	\label{fig:inl-let-d}
 	\end{footnotesize}
	\end{figure}
\end{definition}

	As we see, a step of $\ilarr$ is either a head reduction of let expression given by
\textsc{I$_1$-Let$_0$}, or one of contextual reductions above.\\

	We need to check that $\ilarr$ verifies the properties below:
	
\begin{lemma}[(Determinacy of $\inlletarr$)] 
	$\forall t, t_1, t_2.
		(t \inlletarr {t_1} \bwedge t \inlletarr {t_2}) \brarr
			t_1 = t_2$.	\label{inllet-determ-l}	
\end{lemma}
\begin{proof} Straightforward induction on the derivation of $t \inlletarr {t_1}$, observing that in each case only one rule applies. \end{proof} 
\begin{lemma}[($\inlletarr$ Preservation Properties)]
For any well-typed term $t$ such that\\ $\typerule{t}{\tau}{\theta}{\rho}$ and $\inllet{t}{t'}$, the following properties hold:
\begin{displaymath}
\begin{array}{l@{\hspace*{1cm}}l@{\hspace*{1cm}}l}
	(1)& FV(t') \subseteq FV(t) & \textsc{(free variables inclusion)} \\
	(2)& t' \in \inlT & \textsc{(A-normal form preservation)}\\ 
	(3)& \typerule{t'}{\tau}{\theta}{\rho} &\textsc{(typing and effects preservation)} \vspace*{-0.2cm}
\end{array}
\end{displaymath}
 \label{inllet-prop-l}
\end{lemma}
\begin{proof}
	By straightforward induction on the derivation of $\inllet{t}{t'}$. 
	The only interesting case is the head reduction \textsc{I$_1$-Let$_0$}:
	\infrule[$I_1$-Let$_0$]
		{ s_1, s_2  \in S}
			{\boldsymbol{\tmlet{F}{s_1}{s_2} \hookdownarrow 				
			\tmsbst{s_2}{F}{s_1}}}
	Here, one can check that (1) holds by simply computing  FV(t) and FV(t').
	(2) holds, because none of the following constructions can be a term of $\inlsrc$ : 
$$\textit{(t~F) \qquad r~:=~F \qquad if~F~then~... else ... \qquad match~F~with~...}$$
	Finally, the typing of $s_1$ is $\typerule{s1}{\tau^2}{\bth}{\brh}$, by typing \hyperlink{restr-let-f}{restrictions} above.	
	Therefore, (3) follows from the \hyperlink{subst-lemma}{substitution lemma} above. 
 \end{proof}

It remains to prove that $\inlletstar$ always terminate, and results in a term $s \in S$.  
	To do so, we define the set of normal forms $\inlletNF$, and prove the following statements below: 
\begin{definition}[(Normal Forms $\inlletstar$)]
	%The set of normal forms for $\inlletplus$,  $\inlletNF$, is defined as follows: \quad	
 $ \inlletNF \triangleq \{ t | t \in \inlT \bwedge 
 \forall t' \in \inlT. t~\cancel{\inlletarr}~t' \} $
\end{definition}

\begin{lemma}[(Normal Forms For $\inlletstar$)] 
$ \inlletNF  = \inlS.$
\label{inllet-nforms-l}
\end{lemma}
\begin{proof}
Obverse that, by definition of $\inlletarr$, we have immediately $\inlS \subseteq  \inlletNF$. The other direction can be proven by showing the contrapositive $t \not\in \inlS \Rightarrow t \not\in \inlletNF$ which is trivially true.
\end{proof}


\hypertarget{measure-idea}{}
 A possible way to prove the normalisation of $\ilarrt$ is to find a measure $\varphi$, such that for each step $\il{s_{i}}{s_{i+1}}$ of the reduction chain $ s_0 \ilarr ... \ilarr s_i \ilarr s_{i+1} \ilarr ...$, we get $\varphi(s_{i}) >  \varphi(s_{i+1}) $. 

Here such a measure can be found easily: we simply define $\varphi$ as the number of second-order local bindings of a given term. We start by proving the auxiliary lemma below:

\begin{lemma}[(Decreasing Measure for $\inlletplus$)] 
If $\il{t}{t'}$, then $\varphi(t) > \varphi(t')$.
\hypertarget{desc-meas-1}{}
\end{lemma}
\begin{proof}
By induction on the derivation $\il{t}{t'}$. \\

\noindent\textit{Case} (\textsc{I}$_{1}$-\textsc{Let}$_0$):
$\qquad \dfrac{}{\varphi(let~F~=~s_1~in~s_2) = 1 > 0 = \varphi(s_2 [F \mapsfrom s_1])}$\\[0.5cm]

\noindent\textit{Case} (\textsc{I}$_{1}$-\textsc{App}$_1$):

$  \hspace*{4cm}\dfrac{\il{t_1}{t_1'}}{\il{(t_1~v)}{(t_1'~v)}}$\\[0.6cm]
By induction hypothesis on $\il{t_1}{t_1'}$ we get $\varphi(t_1) > \varphi(t_1')$.
Thus, $${\varphi(t_1~v_2) = \varphi(t_1) + \varphi(v_2) > \varphi(t_1') + \varphi(v_2) = \varphi(t_1'~v_2)}$$

Other cases are similar to (\textsc{I}$_{1}$-\textsc{App}$_1$). \end{proof}
Now we can proof the termination of $\inlletplus$ as follows:
\begin{theorem}[(Termination of $\inlletplus$)] 
		$\forall t_0 \in \inlsrc ~
	 			\exists n \in \mathbb{N}:  
	 				  (\inllett{t_0}{t_n}) \bwedge (t_n \in \inlletNF).$
\label{inllet-term-l}
\end{theorem}
\begin{proof} 
	Let $t_0$ be a well-typed term of $\inlsrc$. 
	If	$t_0 \in S$, then by \cref{inllet-nforms-l}, we have immediately $t_0 \in \inlletNF$. \\
	
	Otherwise, let $t_0 \ilarr t_1 \ilarr ... \ilarr t_j$ be a reduction chain of inlining steps. 
	The auxiliary lemma \hyperlink{decr-meas-1}{above} applies for \textit{every} step of this chain.
	Consequently, the sequence $k = \varphi(t_0) > \varphi(t_1) >  ... > \varphi(t_i) > ... $ is strictly decreasing, so there exists some $n$, for which $\varphi = 0$, and $ t_{i} \inlletarr ... \inlletarr t_n \in \inlletNF$. 
	By \cref{inllet-determ-l}, the rewriting strategy $\inlletarr$ is deterministic, so the chain above is uniquely defined.
	Theferore, $n$ is uniquely defined too, which is exactly what we wanted to prove. 	
\end{proof}
The main result of this subsection follows immediately from the theorems above:
\begin{corr}[(Second-Order Local Bindings Inlining)] 
~\\
 $\forall t \in \inlT.~ t \not\in \inlS \brarr 
	 				  \exists !s \in \inlT: \inllett{t}{s}.$
\label{inllet-corr}
\end{corr}


\subsection{Inlining Second-Order Applications}

	In the previous subsection, we transform an arbitrary program \textit{P} $ \in \inlsrc$ into program of $S$, the subset of $\inlsrc$ where all second-order variables are eliminated. 
	Call this program ${I_1(\textit{P})}$.
	Our goal now is to inline all possible \textbf{second-order applications} that may appear in $I_1(\textit{P})$. Call the resulting program ${I_2(I_1(\textit{P}))}$. 

	We define inlining of second order applications as function $\icarrt : S \longrightarrow U$, where $U$ is a subset of first-order programs of $\inlsrc$.
	This function is the transitive, reflexive closure of the \textit{small-step rewriting strategy} $\icarr$ defined below: 
\label{icrule-macro}
\newcommandx{\icrule}[6]
{\infrule[$I_2$-#1]
	{\ic{\boldsymbol{#2}}{\boldsymbol{#3}} \quad #6}
	{\ic{#4}{#5}}}

\newcommandx{\icrulehead}[4]
{\infrule[$I_2$-#1]
	{#4}
	{\ic{#2}{#3}}}
	
\label{ictrule-macro}	
	\newcommandx{\ictrule}[6]
{\infrule[$I_1$-#1]
	{\ict{\boldsymbol{#2}}{\boldsymbol{#3}}\qquad #6}
	{\ict{#4}{#5}}}

%%%%%%%%%%%%%%%%%%%%%%%%%%%%%%%%%%%%%%%%%%%%%%%%%%%%%%%%%%%%%%%%%%%%%%%%%%%%%%%
\begin{definition}[(One Step Inlining Of Second-Order Calls, $\icarr$)]
	To fix the notation, we write $s_i,w_j$ for the terms and values of $S$, and $u_i, 		
	\vartheta_j$  for the terms and values of $U$. \\
	
	Let $s$ be a term of $S$ such that $s \not\in U$. 
  Then, $s$ contains at least one application $(s_1~u_2)$, with $s_1 : \tau^2$.
	
  \indent There are four possible cases to inline $s$ containing exactly one second-order application such that it is $s$ itself:	
	\begin{figure}[H]
		\begin{footnotesize}
	\begin{adjustwidth}{-10em}{0em}

	\begin{multicols}{2}
	\hypertarget{Rule-def-I$_2$-App$_0$}{}
	\icrulehead
		{App$_0$}
		{(\lambda f. u~\vartheta)}
		{\tmsbst{u}{f}{\vartheta}}
		{u,\vartheta \in U} 
	\vspace*{0.5cm}
	
	\hypertarget{Rule-def-I$_2$-If$_0$}{}
	\icrulehead
		{If$_0$}
		{(\boldsymbol{(}if~v~then~u_1~else~u_2{)} \boldsymbol{\vartheta}) \\}
		{if~v~then~\boldsymbol{(u_1~\vartheta)}~else~\boldsymbol{(u_2~\vartheta)}}
		{u_1, u_2, \vartheta \in U, ~~~ (u_1, u_2:\tau^2)}	

	\icrulehead
		{Let$_0$\hypertarget{Rule-def-I$_2$-Let$_0$}{}}
		{(\boldsymbol{(}\tmlet{l}{u_1}{u_2}\boldsymbol{)} \boldsymbol{\vartheta})}
		{\tmlet{l}{u_1}{\boldsymbol{(u_2~\vartheta)}}} 
		{u_1, u_2, \vartheta \in U ~~~ (u_1, u_2:\tau^2)}	
		\vspace*{0.5cm}
	\icrulehead
		{Match$_0$\hypertarget{Rule-def-I$_2$-Match$_0$}{}}
		{(\boldsymbol{(}match~v~with~Nil~->~ u_1~|~Cons~x~y -> u_2{)} \boldsymbol{\vartheta}) \\}
		{match~v~with~Nil~->~ \boldsymbol{(u_1~\vartheta)}~|~Cons~x~y -> \boldsymbol{(u_2~\vartheta)}}
		{u_1, u_2, \vartheta \in U, ~~~ (u_1, u_2:\tau^2)}				
	\end{multicols}
		\end{adjustwidth}	
		\caption{ \textbf{Head Reduction Inlining of Second-Order Applications}\hfill}
 	\label{fig:inl-app-h-d}
 		\end{footnotesize}
	\end{figure}

Otherwise, $s$ contains at least two second-order applications. 
So, there is at least one proper sub-term of $s$ of the from $(s_1 w_1)$ with $s_1 : \tau^2$. 
In that case, a step of inlining of $s$ is defined inductively as inlining of the leftmost immediate subterm of $s$ in which $(s_1 w_1)$ occurs. 
That is, a step $\ic{s}{s'}$ is defined by the inferences rules below:
	\begin{figure}[H]
	\begin{footnotesize}
	\begin{adjustwidth}{-8em}{-3em}
	\begin{multicols}{2}	
	\icrule{Fun}
		{s_1}{s_1'}
		{\lambda l. \boldsymbol{s_1}}{\lambda l. \boldsymbol{s_1'}}	
		{}\vspace*{0.5cm}
				
	\icrule{App$_1$}
		{s_1}{s_1'}
		{(\boldsymbol{s_1}w_2)}{(\boldsymbol{s'_1}w_2)}
		{}\vspace*{0.5cm}
		
	\icrule{Let$_1$}
		{s_1}{s'_1}
		{\tmlet{l}{\boldsymbol{s_1}}{s_2}}{\tmlet{l}{\boldsymbol{s'_1}}{s_2}}
		{}\vspace*{0.5cm}
		
\infrule[$I_2$-If$_1$]
		{ \boldsymbol{s_1 \icarr s_1'} }
			{ if~v~then~\boldsymbol{s_1}~else~s_2 \icarr
				if~v~then~\boldsymbol{s_1'}~else~s_2 }	
		{}\vspace*{0.5cm}				
			
	\infrule[$I_2$-Match$_1$]
		{ \boldsymbol{s_1 \icarr s_1'} }
			{~~~~~ match~v~with~Nil~->~ \boldsymbol{s_1}~|~Cons~x~y -> s_2 \icarr \\ 
				match~v~with~Nil~->~ \boldsymbol{s_1'}~|~Cons~x~y -> s_2}			
		{}
		
	\icrule{Rec}
		{s_1}{s_1'}
		{\tmrec{f}{x}{(\tau^1, \theta)}{\boldsymbol{s_1}}}{\tmrec{f}{x}{\tau}{\boldsymbol{s_1'}}}	
		{}\vspace*{0.5cm}
				
	\icrule
		{App$_2$}
		{w_2}{w_2'}
		{(u_1\boldsymbol{w_2})}
		{(u_1\boldsymbol{w_2'})}
		{\quad u_1 \in U}\vspace*{0.5cm}
		
	\icrule{Let$_2$}
		{s_2}{s'_2}
		{\tmlet{l}{u_1}{\boldsymbol{s_2}}}{\tmlet{l}{u_1}{\boldsymbol{s'_2}}}
		{\quad u_1 \in U}\vspace*{0.5cm}
		
	\infrule[$I_2$-If$_2$]
		{ \boldsymbol{s_2 \icarr s_2'}  \quad u_1 \in U }
			{ if~v~then~u_1~else~\boldsymbol{s_2} \icarr
				if~v~then~u_1'~else~\boldsymbol{s_2'} }\vspace*{0.5cm}					
			
	\infrule[$I_2$-Match$_2$]
		{ \boldsymbol{s_2 \icarr s_2'} \quad u_1 \in U}
			{~~~~~ match~v~with~Nil~->~ u_1~|~Cons~x~y -> \boldsymbol{s_2} \icarr \\ 
				match~v~with~Nil~->~ u_1~|~Cons~x~y -> \boldsymbol{s_2'}}					
		
	\end{multicols}
	\end{adjustwidth}
	\caption{ \textbf{Context Reduction Inlining of Second-Order Applications}\hfill}
 	\label{fig:inl-app-c-d}
 		\end{footnotesize}
	\end{figure}
\end{definition}
%
%	One can check that \cref{fig:inl-app-h-d} describes indeed the only possible cases to inline terms such as $s$, for which all sub-terms are to be already first-order expressions.  
%	  \\

It remains to check that inlining of applications verifies the properties below:

\begin{lemma}[(Determinacy of $\icarr$)] 
	$\forall s, s_1, s_2.
		(s \icarr {s_1} \bwedge s \icarr {s_2}) \brarr
			s_1 = s_2$.	\label{ic-determ-l}	
\end{lemma}
\begin{proof} Similar to the proof of \ref{inllet-determ-l}. 
\end{proof} 
\begin{lemma}[($\icarr$ Preservation Properties)]
For any well-typed term $s$ such that\\ $\typerule{s}{\tau}{\theta}{\rho}$ and $\ic{s}{s'}$, the following properties hold:
\begin{displaymath}
\begin{array}{l@{\hspace*{1cm}}l@{\hspace*{1cm}}l}
	(1)& FV(s') \subseteq FV(s) & \textsc{(free variables inclusion)} \\
	(2)& s' \in S & \textsc{(A-normal form preservation)}\\ 
	(3)& \typerule{s'}{\tau}{\theta}{\rho} &\textsc{(typing and effects preservation)}
	\end{array}
\end{displaymath}
 \label{ic-prop-l}
\end{lemma}	
\begin{proof}
	By induction on the derivation of $\ic{s}{s'}$.  \\
	
	\textsc{Case (I$_2$-App$_0$)}:
	\icrulehead
		{App$_0$}
		{(\lambda f. u~\vartheta)}
		{\tmsbst{u}{f}{\vartheta}}
		{u,\vartheta \in U} 
	\vspace*{0.3cm}
	Here, one can check that (1) holds by simply computing  FV(t) and FV(t').
	(2) holds, because A-Normal form is invariant under $\beta-$reduction.
	Finally, the typing of $\vartheta$ is $\typerule{\vartheta}{\tau^1}{\bth}{\brh}$ so, by the \hyperlink{subst-lemma}{substitution lemma}, (3) holds as well. 
 \end{proof}
 
 \begin{definition}[(Normal Forms $\icarrt$)]
 $ \icNF \triangleq \{ s | s \in S \bwedge 
 \forall s' \in S. s~\cancel{\icarrt}~s' \} $
\end{definition}

\begin{lemma}[(Normal Forms For $\icarrt$)] 
$ \icNF  = U.$
\label{ic-nforms-l}
\end{lemma}
\begin{proof} Same as the proof of \cref{inllet-nforms-l}.
\end{proof}
The proof of termination of $\icarrt$ relies on the same \hyperlink{measure-idea}{idea}  idea that in the case of  $\icarrt$.

%: find a measure $\varphi$ such that for each step $\ic{s_{i}}{s_{i+1}}$ of the reduction chain $ s_0 \icarr ... \icarr s_i \icarr s_{i+1} \icarr ...$, we get $\varphi(s_{i}) >  \varphi(s_{i+1}) $.  

	In the case of $\inlletarr$, we take for $\varphi (t_{i})$ the number of second-order let constructions. 
	Here, $\varphi$ is a bit more sophisticated:
%	Obviously, the raw number of second-order applications cannot be a decreasing measure
%	for $\icarr$: indeed, the application of \hyperlink{Rule-def-I$_2$-If$_0$}{\textsc{I$_2$-If$_0$}} and \hyperlink{Rule-def-I$_2$-Match$_0$}{\textsc{I$_2$-Match$_0$}} increases this number from one to two. 
%
%	On the other hand, during any evaluation of term $s_{i+1}$ in these rules,  \textbf{one and only one branch} branch of \texttt{"if"} or \texttt{"match with"} expression will be executed.	
%	
%	That is, in three of four base cases of $\icarr$ (\textsc{I$_2$-If$_0$}, \textsc{I$_2$-Match$_0$}, \textsc{I$_2$-Let$_0$}) the number of \textbf{\textit{eventually}} evaluated second-order applications of $s_{i+1}$ remains the same as for $s_{i}$. 
%	In the case of \textsc{I$_2$-App$_0$}, this number goes straight from one to zero. 
	
%	It remains to define $\varphi$ according to this observation:
	
\begin{definition}[(Decreasing measure for $\icarr$)]
	Let $s$ be a term $\in S$. 
	
	Then either $s \in U$. In that case we pose that $\varphi(s) = 0$.
	
	Otherwise,  $s \not\in U$, so there is some second-order application $(s' w')$ inside $s$. 
	By case analysis of all possible contexts in which $(s' w')$ may appear in $s$, we define $\varphi(s)$ by induction on the structure of $s$ as follows:
\begin{displaymath}
	\begin{array}{lll@{\hspace*{3cm}}l}
	\varphi(\lambda f. u~\vartheta) &=& 1 \\ 
	 
	\varphi(\boldsymbol{(}match~v~with~Nil~->~u_1~|~Cons~x~y -> u_2{)} \vartheta) \quad \text{with}~u_1, u_2:\tau^2 &=& 3 \\
	 
	\varphi(\boldsymbol{(}let~x~=~u_1~in~u_2{)} \vartheta) \quad \text{with}~u_1, u_2:\tau^2 &=& 2 \\
	 
	\varphi(\boldsymbol{(}if~v~then~u_1~else~u_2) \vartheta \boldsymbol{)}) \quad \text{with}~u_1, u_2:\tau^2 &=& 3 \\
	 
	\varphi(s_1 w_2) \quad \text{with}~s_1:\tau^2  &=& 1 + \varphi(s_1) + \varphi(w_2) \ \\
	
	\varphi(s_1 w_2) \quad \text{with}~s_1:\tau^1  &=& \varphi(s_1) + \varphi(w_2) \ \\	 
	 
	\varphi(\lambda l. s_1) &=& \varphi(s_1) \\  
	
	\varphi(\tmrec{f}{x}{(\tau^1, \theta)}{s_1}) &=& \varphi(s_1) \\ 	 
	 
	\varphi(let~x~=~s_1~in~s_2) &=& \varphi(s_1) + \varphi(w_2) \\
	 
	\varphi(if~v~then~s_1~else~s_2) &=& \varphi(s_1) + \varphi(s_2) \\

	\varphi(match~v~with~Nil~->~s_1~|~Cons~x~y -> s_2) &=& \varphi(s_1) + \varphi(s_2) \\
	\end{array}
\end{displaymath}	
	\end{definition}	

It remains to check it defines a decreasing measure for $\icarr$ indeed:
	\begin{lemma}[(Decreasing Measure for $\icarr$)] 
If $\il{s}{s'}$, then $\varphi(s) > \varphi(s')$.
\hypertarget{desc-meas-1}{}
\end{lemma}
\begin{proof}
By induction on the derivation of $\il{s}{s'}$. \\

\noindent\textit{Case} (\textsc{I}$_{2}$-\textsc{App}$_0$):
$\hspace*{2cm} \varphi(\lambda f. u~\vartheta) = 1 > 0 = \varphi(u [f \mapsfrom \vartheta])$\\

\noindent\textit{Case} (\textsc{I}$_{2}$-\textsc{Match}$_0$): 
\hspace*{2em} 
$\varphi(\boldsymbol{(}match~v~with~Nil~->~u_1~|~Cons~x~y -> u_2{)} \vartheta) = 3 >$\\
\hspace*{10em} 
$	\varphi(match~v~with~Nil~->~(u_1~\vartheta)~|~Cons~x~y -> (u_2~\vartheta){)} = 2$\\

\noindent\textit{Case} (\textsc{I}$_{2}$-\textsc{Let}$_0$): 
\hspace*{2em} 
$\varphi(\boldsymbol{(}let~x~=~u_1~in~u_2{)} \vartheta) = 3 > $\\
\hspace*{8em} 
$\varphi(let~x~=~u_1~in~(u_2~\vartheta)) = 1$\\

\noindent\textit{Case} (\textsc{I}$_{2}$-\textsc{If}$_0$): 
\hspace*{2em} 
$\varphi(\boldsymbol{(}if~v~then~u_1~else~u_2{)} \vartheta) = 3 > $\\
\hspace*{7em} 
$\varphi(if~v~then~(u_1~\vartheta)~else~(u_2~\vartheta)) = 2$\\

For the remaining rules, the result follows from a immediate use of the induction hypothesis on the sub-derivation of each rule.
\end{proof}
\begin{theorem}[(Termination of $\icarrt$)] 
		$\forall s_0 \in S ~
	 			\exists n \in \mathbb{N}:  
	 				  (\ict{s_0}{s_n}) \bwedge (s_n \in \icNF).$
\label{ic-term-l}
\end{theorem}
\begin{proof}
%	With the definition above, we can easily prove by induction on the derivation of $\ic{s_{i}}{s_{i+1}}$ that $\varphi(s_{i}) > \varphi(s_{i+1})$ for each step $\ic{s_{i}}{s_{i+1}}$ of the reduction chain $ s_0 \icarrt ... \icarrt s_i \icarrt s_{i+1} \icarrt ...$. 
%	It follows then that the sequence $k = \varphi(s_0) > \varphi(s_1) >  ... > \varphi(s_i) > ... $ is strictly decreasing, so that there exist some $n$, for which $\varphi(t_n) = 0$, and $ s_{i} \inlletarr ... \inlletarr s_n \in \icNF$. 
%	By \cref{ic-determ-l}, the rewriting strategy $\icarrt$ is deterministic, so the chain above is uniquely defined.
%	Theferore, $n$ is uniquely defined too, which is exactly what we wanted to prove. 
Same as the proof of \cref{inllet-term-l}.
\end{proof}
\begin{corr}[(Second-Order Application Inlining)] 
 $\forall s \in S.~ s \not\in U \brarr 
	 				  \exists !u \in U: \ict{s}{u}.$\end{corr} 
 Finally, from the corollary above and \cref{inllet-corr} we conclude that 
 \begin{corr}[(Second-Order Inlining)] 
 $\forall P \in \inlsrc.~ \exists !u \in U: u = I_2(I_1(P)) .$
\end{corr} 
%for each program $P \in $ inlining  terminating and always resulting in a first-order program is achieved. :
%%	
%Because second-order applications are the only second-order expressions that can appear in $I_1\textit{P})$, we eventually get an entirely first-order program. \\	
	
%%%%%%%%%%%%%%%%%%%%%%%%%%%%%%%%%%%%%%%%%%%%%%%%%%%%%%%%%%%%%%%%%%%%%%%%%%%%%%%


\subsection{Total Correctness of Inlining}

	Our final goal is to prove the correction of inlining, i.e. to prove the \textit{meaning} of source term is preserved. 
	
	This can be understood as proving the inlining procedure \textit{partial correctness} or as \textit{total correctness}.
	
	\textit{Partial correctness} of inlining guarantees that if the source \texttt{P} program evaluates giving some value \texttt{V}, then the inlining output \texttt{O} evaluation terminates too and gives the same value \texttt{V}. 
	
	 \textit{Total correctness} of inlining extends this statement for diverging programs: if the source \texttt{P} program does not terminate, then \texttt{O} does not terminates neither.
	
	As we defined each of two inlining phases, $\ilarrt$ and $\icarrt$ in a \textit{small-step} manner, we will prove the \textit{total correctness} of inlining.
	
	To do so, we start by formalizing $\thicksim$, the notion of semantic equivalence between two programs.
%	
%	Then, for each of two inlining steps $I_1$ (i.e. $\ilarrt$),$I_2$ (i.e. $\icarrt$), we prove that for each pair of the initial program P $\in \inlsrc$ and corresponding output $I_2(I_1(P))$,  $P \thicksim I_2(I1(P))$.
		 
	
	
%%%%%%%%%%%%%%%%%%%%%%%%%%%%%%%%%%%%%%%%%%%%%%%%%%%%%%%%%%%%%%%%%%%%%%%%%%%%%%%
\subsubsection*{Semantic Equivalence Between Closed Terms}

\textit{Logical relations} is a technique introduced in Friedman$^{\cite{Fri:75}}$, Plotkin$^{\cite{plotkin1973lambda}}$ and others.
	Various results in lambda-calculus and type theory were established by this technique$^{\cite{MitchellM85}}$.
	Here we use logical relations to define the semantics equivalence below: 

%%%%%%%%%%%%%%%%%%%%%%%%%%%%%%%%%%%%%%%%%%%%%%%%%%%%%%%%%%%%%%%%%%%%%%%%%%%%%%%
\label{semantic equivalence}
\begin{definition}[(Semantic Equivalence)]
	Let $t$ and $t'$ be two well-typed closed terms,	such that 
	
	$$ t,t' \in \inlsrc  \quad 
		\typerule{t}{\tau}{\theta}{\rho}{} \quad \typerule{t'}{\tau}{\theta}{\rho}{} 
	$$

	Then we define $\eqv{t}$, the \textit{semantic equivalence} between $t$ and $t'$, 
	by induction on the structure of type $\ty$ :
	
	\begin{itemize}
		\item[$(\alpha)$]

			if $\ty{} = \ty{}^0$ for some base type $\ty{}^0$, then $\eqv{t}$ iff
			$$	\forall \mu_0.(\evalinfty{t}{\mu_0} \bwedge \evalinfty{t'}{\mu_0})~
					\bvee ~ \exists \mu_1 \exists v: 
					\evalstar{t}{0}{v}{1} \bwedge \evalstar{t'}{0}{v}{1}, $$
			where $\forall \mu_0.\evalinfty{t}{\mu_0}$ means that for any initial state $\mu_0$, the evaluation $t$ diverges.
		\item[$(\beta)$]
			if $\ty = \tyarr$ for some types $\ty[1], \ty[2]$,
			then\\[0.2cm]
			$\hspace*{0em}\forall v_0, v'_0. ~ \eqv{v_0} ~ 
			\bwedge \typerule{v_0}{\ty[1]}{\bth}{\brh}{} ~~
			\bwedge \typerule{v'_0}{\ty[1]}{\bth}{\brh}{} $ \\[0.2cm]
			$\hspace*{1em}
			\Rightarrow \forall\mu_0.(\evalinfty{\tmapp{t}{v_0}}{\mu_0}
			\bwedge\evalinfty{\tmapp{t'}{v'_0}}{\mu_0})$\\[0.2cm]
			$\hspace*{2em}\bvee~(\exists \mu_1 \exists v_1, v'_1 : 
			 	\evalstar{\tmapp{t}{v_0}}{0}{{v_1}}{1} 
				\bwedge \evalstar{\tmapp{t'}{v'_0}}{0}{{v'_1}}{1}
				\bwedge~\eqv{v_1})$				
	\end{itemize}
	\end{definition}
\vspace*{0.5cm}
We need to check the following 	lemmas:
%%%%%%%%%%%%%%%%%%%%%%%%%%%%%%%%%%%%%%%%%%%%%%%%%%%%%%%%%%%%%%%%%%%%%%%%%%%%%%%
	\begin{lemma} 
		The semantic relation, $\thicksim$, is reflexive, 	
		symmetric and transitive relation.
	\end{lemma}
	
	\begin{proof}
		By straightforward induction on the structure of type $\tau$.
	\end{proof}

%%%%%%%%%%%%%%%%%%%%%%%%%%%%%%%%%%%%%%%%%%%%%%%%%%%%%%%%%%%%%%%%%%%%%%%%%%%%%%%
	\begin{lemma}
		$\forall t,t'.~ \eqv{t} \Leftrightarrow 
			(\forall v, v'.~\eqv{v}~\Rightarrow
			\tmapp{t}{v} \thicksim \tmapp{t'}{v'})$
	\label{equiv-def-l}
	\end{lemma}
	
	\begin{proof} By induction on the structure of type $\ty[t]$.
	For detailed proof, see \ref{equiv-def-p}.
	\end{proof}

%%%%%%%%%%%%%%%%%%%%%%%%%%%%%%%%%%%%%%%%%%%%%%%%%%%%%%%%%%%%%%%%%%%%%%%%%%%%%%%
	
	\begin{lemma} 
	$\forall \mu_0, \mu_1, t.\evalstar{{t_0}}{0}{{t_1}}{1} \bwedge \evalstar{{t_0'}}{0}{{t_1'}}{1} \bwedge \eqv{t_0} \Rightarrow \eqv{t_1}$.
		\label{equiv-red2-l}
	\end{lemma}
	
	\begin{proof}
		By induction on the structure of type of $t_0$. 
		For detailed proof, see \ref{equiv-red2-p}.
	\end{proof}		
	\begin{corr} 
		$\forall \mu_0, \mu_1.\evalstar{{t}}{0}{{v}}{1} \bwedge \evalstar{{t'}}{0}{{v'}}{1} \bwedge \eqv{t} \Rightarrow \eqv{v_1}$.
		\label{equivalence parallel preservation corr}
	\end{corr}
	
	
	\begin{lemma}[(Equivalence for Reduction Step)]
		For any state $\mu_0$, if $\evalstep{{t_0}}{0}{{t_1}}{0}$, then
		$t_0 \thicksim t_1$.
	\end{lemma}	
	\begin{proof}
		By straightforward observation that $\rightarrow$ is deterministic.
	\end{proof}		

	\begin{corr} 
		For any state $\mu_0$, if $\evalstar{{t}}{0}{{v}}{0}$, then		
		$t \thicksim v$.
	\end{corr}
%%%%%%%%%%%%%%%%%%%%%%%%%%%%%%%%%%%%%%%%%%%%%%%%%%%%%%%%%%%%%%%%%%%%%%%%%%%%%%%
\subsubsection*{Semantics Equivalence Between Parallel Substitutions}

  With the definition above, the total correctness of inlining is formulated as follows:
  
  \begin{theorem}[(Total Correctness)]
   $ \forall t \in \inlsrc. t \thicksim I_2(I_1(t)) $
  \end{theorem}

	To prove it, we need to show two lemmas below:
	\begin{lemma}[(Total Correctness for $\ilarr$)]
   $ \forall t,t' \in \inlsrc. \il{t}{t'} \Rightarrow t \thicksim t'$
  \end{lemma}

	\begin{lemma}[(Total Correctness for $\icarr$)]
   $ \forall s,s' \in S. \ic{s}{s'} \Rightarrow s \thicksim s'$
  \end{lemma}
  
 	To prove these lemmas, on their turn, we need to reason by induction on the derivation rules of inlining. 
  Consequently, whenever we apply an induction hypothesis on sub-derivation $\il{t}{t'}$ or $\ic{t}{t'}$, we have to be sure that $\eqv{t}$ is well-defined. 
  
  The technical difficulty is that in the case of abstractions and recursive functions, $t$ and $t'$ are not necessary closed, so $\eqv{t}$ may have no sense.  
	We can resolve this difficulty, if instead of proving the theorem's statement for closed terms only, we generalize it to all closed instances of an open term $t$, using the definition below:
 
	\begin{definition}[(parallel substitutions)]
	Let $t$ be any well-typed term, $t_1, \dots t_n$ well-typed 
	\textit{\textbf{closed}} terms, and $x_1, \dots, x_n$ distinct variables.
	To fix the notation, 
	let $\sigma = [x_1 \mapsfrom t_1, \dots, x_n \mapsfrom t_n] $ be
	a finite map such that :  	
	$$\forall x\in\mathfrak{Dom}(\sigma).~x\notin BV(t)\bwedge\tau_x=
	\tau_{\sigma(x)}$$

	Then the parallel substitution $(t\sigma)$ is defined by induction on $t$ 
	as follows :
	\begin{displaymath}
	\begin{array}{lll}
	 (x_i\sigma)& \triangleq & \sigma(x_i)\\
	 (y\sigma)& \triangleq & y \\
	 ((t_1t_2)\sigma) & \triangleq & ((t_1\sigma)(t_2\sigma))\\
	 (\lambda y_{\tau}. t)\sigma & \triangleq & \lambda y_\tau. (t\sigma)\\
	 (rec~f~y_{\tau_1} : \tau_2 = t)\sigma & \triangleq 
	   & rec~f~y_{\tau_1} : \tau_2 = (t\sigma) \\
	 ((let~y_\tau = t_1~ in~ t_2)\sigma) & \triangleq 
	   & let~y_\tau = (t_1\sigma)~in~ (t_2\sigma)\\
	  (r_\tau := v)\sigma)	& \triangleq & r_\tau := (v\sigma) \\
	 (!r_\tau\sigma) & \triangleq & !r_\tau \\
	     	 
	\end{array}
	\end{displaymath}
	\label{}
	\end{definition}

	\begin{definition}[(equivalent parallel substitutions)]
	Let $t$ be a well-typed term. Then we define the semantic equivalence
	between two parallel substitutions $\sigma$ and $\sigma'$, 
	$\sigma \thicksim_t \sigma'$ as follows:
	\begin{itemize}
	\item[(1)] 
	  $(t\sigma)$ and $(t\sigma')$ are well-defined parallel substitutions
	\item[(2)]
	  $ \mathfrak{Dom}_\sigma = \mathfrak{Dom}_\sigma' = FV(t)$
	\item[(3)]
		$ \forall x \in \mathfrak{Dom}_\sigma.~ \sigma(x) \thicksim_t \sigma'(x)$
	\end{itemize}	 
	\label{equiv-subst-d}
	\end{definition}

	Note that from (1) it follows, that both $(t\sigma)$ and 
	$(t\sigma')$ are terms that well-typed, and from (2) and (3) it follows 
	that both $(t\sigma)$ and	 $(t\sigma')$ are \textit{\textbf{closed}} terms.

	Also we have to prove the following lemma:	
	
	\begin{lemma} 
		$\forall t. \forall (\sigma, \sigma'). ~\eqvsbst{\sigma}{t}
		\Rightarrow \tmapp{t}{\sigma} \thicksim \tmapp{t}{\sigma'}.$ 
	\label{equiv-subst-l}
	\end{lemma}

	\begin{proof}
		By induction on the structure of term $t$. 
		%For detailed proof, see 		\ref{equiv-subst-p}.
	\end{proof}





\subsubsection*{Proof of Total Correctness}


	With the definitions above, we can formulate the total correctness lemmas as follows:
 \begin{lemma}[(Total Correctness for $\ilarr$, Strengthened)]
   $$ \forall t,t' \in \inlsrc. \il{t}{t'} \Rightarrow (\forall \sigma, \sigma'. \sigma \thicksim_t \sigma' \Rightarrow t\sigma \thicksim t'\sigma')$$
  \end{lemma}
	\begin{proof}
	 By induction on derivation $\il{t}{t'}$.
	\end{proof}	  
  
	\begin{lemma}[(Total Correctness for $\icarr$, Strengthened)]
     $$ \forall s,s' \in \inlsrc. \ic{s}{s'} \Rightarrow (\forall \sigma, \sigma'. \sigma \thicksim_s \sigma' \Rightarrow s\sigma \thicksim s'\sigma')$$
  \end{lemma}
	\begin{proof}
	 By induction on derivation $\ic{s}{s'}$.
	\end{proof}	  
	
  \begin{theorem}[(Total Correctness)]
   $ \forall t \in \inlsrc. t \thicksim I_2(I_1(t)) $
  \end{theorem}	
	\begin{proof}
	 Todo.
	\end{proof}	 
	
%TODO: Nice figure %

%\subsection{Inlining Higher-Order Programs}


%The correctness of inlining we establish is \textit{total}: source language is
%not confined to terminating programs and an input program's non-termination is
%preserved by inlining. As \textit{Kahn natural semantics} is limited to
%establishing the \textit{partial} correctness, that explains our choice of an
%overall small-step style of programs operational semantics and inlining. 
% 
% 
%	Finally, we give the proof of concentrate our attention on the \textit{total correctness} of inlining
%transformation: using a well-know \textit{logical relations} technique, we will
%formalize the notion semantic equivalence between programs and show that for
%every couple of source and target programs, these programs are semantically
%equivalent.
% 
%The correctness of inlining we establish is \textit{total}: source language is
%not confined to terminating programs and an input program's non-termination is
%preserved by inlining. As \textit{Kahn natural semantics} is limited to
%establishing the \textit{partial} correctness, that explains our choice of an
%overall small-step style of programs operational semantics and inlining.


%%%%%%%%%%%%%%%%%%%%%%%%%%%%%%%%%%%%%%%%%%%%%%%%%%%%%%%%%%%%%%%%%%%%%%%%%%%%%%%

\section{Experimental Evaluation}

 The first argument of \texttt{inv} is of type int, because it is a type of array indexes, so that we can express the invariance of the \texttt{inv}, applying \texttt{inv} in different contexts to different indexes.
  
	For instance, let us rewrite $\sum\_iter$ in $\inlsrc$.
	In $\inlsrc$, we do not have arrays. However, we have integer lists. 
 List structure is quite different from arrays. 
 In particular, lists are recursively constructed data, so instead of access them with indexes, we access them using pattern-matching and recursion:	
	of \texttt{sum\_iter} using lists: \hypertarget{sum-list-iter}{}
	
\begin{small}
	\begin{minipage}[t]{0.4\linewidth}	
	\begin{whycode}  
let list_iter ($\textcolor{red}{\text{f:int -> unit}}$) (b: int list)   
 = let rec loop (l: int list) = 
      match l with 
       | Nil       -> ()
       | Cons x xs -> ($\textcolor{red}{\text{f}}$ x; loop xs); 
   in loop 0 
   
   let sum_list_iter (a: array int) =		 
     let s = ref 0 in
     list_iter $\textcolor{red}{\text{(fun x -> s := !s + x)}}$ a     
 	\end{whycode}
 	\end{minipage}
 \end{small}	
%	\end{minipage}\hfill 
%	\begin{minipage}[t]{0.45\linewidth}
%	\begin{whycode}
%   let sum_list_iter (a: array int) =		 
%     let s = ref 0 in
%     list_iter $\textcolor{red}{\text{(fun x -> s := !s + x)}}$ a  
%	\end{whycode}      
%	\end{minipage}
%\end{small}		

One can easily check that this program can be wrttin with the syntax of $\inlsrc$.


 	



%section{Conclusion and Perspectives}


%%%%%%%%%%%%%%%%%%%%%%%%%%%%%%%%%%%%%%%%%%%%%%%%%%%%%%%%%%%%%%%%%%%%%%%%%%%%%%
\appendix
\section{Detailed proofs} 

%%%%%%%%%%%%%%%%%%%%%%%%%%%%%%%%%%%%%%%%%%%%%%%%%%%%%%%%%%%%%%%%%%%%%%%%%%%%%%





 \subsection{Programming Language}
\hypertarget{proof:preserv-prop-p}{} 
\begin{theorem}[(Preservation of Mini-ML, Strengthened)] 
if $\vdash t : \tau, \textcolor{red}{\bth}, \rho$ holds and $\evalstep{t}{1}{t'}{2}$ then $\mu_1 = \mu_2$ and $\vdash t' : \tau, \textcolor{red}{\bth}, \rho'$ for some $\rho'$.

\end{theorem}
\begin{proof}
\label{proof:preserv-prop-p}
  By induction on the derivation of $\vdash t : \tau, \textcolor{red}{\bth}, \rho$.\\
  
  \noindent\textit{Cases} \textsc{(T-Var), (T-Const), (T-Lam), (T-Rec)}  are trivially true, because there is no reduction step from $t$. \\
  
  \noindent\textit{Case} \textsc{(T-Assign)} are straightforward, because the hypothesis 
  $\vdash t : \tau, \textcolor{red}{\bth}, \rho $ does not hold.\\
  
   \noindent\textit{Case} \textsc{(T-Deref)}: The only reduction rule is textsc{(E-Deref)}:  \infax[E-Deref]{\ghead{{!r_\tau}_{\mem}} {\mu_{}(r_\tau)}}
    for which we have that $\mu_1 = \mu_2$.  
    Because references can be assigned only with values, it follows that $\vdash \mu(r_\tau) : \tau, \textcolor{red}{\bth}, \rho' $ holds. \\ 

  \noindent\textit{Case} \textsc{(T-If)}: $\qquad \dfrac
	{
		\typing{v}{bool}{\bth}{\brh}
		\typing{t_1}{\tau_1}{\theta_1}{\rho_1} \qquad
		\typing{t_2}{\tau_1}{\theta_2}{\rho_2}
	}
	{	\typing{if~v~then~t_1~else~t_2}
		{\tau_1}{\theta_1 \bvee \theta_2}{\rho_1 \bvee \rho_2}}
	{\textsc{  (T-if)}}$ \\
  
	From the hypothesis $\theta_{(if~v~then~t_1~else~t_2)} = \bth$ it follows that $\theta_1 = \theta_2 = \bth$. Thus, the result holds both for reduction step described by \textsc{(E-If-True)} or by \textsc{(E-If-False)}. \\
   
  \noindent\textit{Case} \textsc{(T-Match)}: \qquad Similar to \textsc{(T-If)}. \\
   
   \noindent\textit{Case} \textsc{(T-App)} : $\qquad \dfrac
	{
		\typing{t_1}{\tyarr[2][1][\theta_1][\rho_1]}{\theta_2}{\rho_2} \qquad
		\typing{v}{\tau_2}{\bth}{\brh}}
	{
		\typing{t_1~v}{\tau_1}{\theta_1 \bvee \theta_2}{\rho_1 \bvee \rho_2}}
	{
		\textsc{  (T-app)}} $ \\
		
Again, from the theorem's hypothesis, it follows that $\theta_1 = \theta_2 = \bth$.


Therefore, we have the following sub-cases: 

- \textit{Sub-Case} \textsc{(E-App-T)} :
 \infrule[E-App-T]	
	{{t_{1}}\mem \rightarrow {{t'}_{1}}\memp}
	{{(t_{1}~v_2)}\mem \rightarrow {({t'}_{1}~v_2)}\memp}
	
then, by induction hypothesis we have that $\typing{t'_1}{\tyarr[2][1][\bth][\rho_1']}{\bth}{\rho_2'}$ and $\mu = \mu'$, and that $\typing{t_1~v}{\tau_1}{\bth}{\rho_1' \bvee \rho_2'}$ by the rule \textsc{(T-App)}. \\

- \textit{Sub-Cases}	\textsc{(E-Op-$\delta$) and (E-Op-$\lambda$)} because in both cases, the reference store is not modified, and because the resulting term is value, so necessarily, its $theta$ is equal to $\bth$. \\
		
- \textit{Sub-Cases} (E-App-Fun) and  (E-App-Rec) follow from \hyperlink{subst-lemma}{substitution lemma}.

Finally, the proof of \textsc{(T-Let)} follows exactly the same scheme that in the case of \textsc{(T-App)}, where the reduction step takes place either inside the let expression, or on the top of it.
 In latter case, again the result follows from \hyperlink{subst-lemma}{substitution lemma}. Back to \ref{preserv-prop-d}.
\end{proof}  	

	\subsection{Inlining Second-Order Local Bindings}
	

%\begin{lemma}[(determinacy of $\inlletplus$)] 
%	$\forall n \in \mathbb{N}.~ \forall t, t_1, t_2.
%		(t \inlletarr^{n} {t_1} \bwedge t \inlletarr^{n} {t_2}) \brarr
%			t_1 = t_2$.	
%	\label{inllet-determ-p}	
%\end{lemma}

\begin{proof}
Start by proving the base case (\textsc{$I_1$-step}) by case analysis, for each possible form of $t$, on the leftmost immediate sub-term of $t$ containing an occurrence of some second-order variable $F$. 
	  Do the remaining  (\textsc{$I_1$-trans}) case	by strong induction on $n$.  Back to \ref{inllet-determ-l}.
\end{proof}	
	
\begin{lemma}[($\inlletarr$ Preservation Properties)]
For any well-typed term $t$ such that\\ $\typerule{t}{\tau}{\theta}{\rho}$, if $\inllet{t}{t'}$ then
\begin{displaymath}
\begin{array}{l@{\hspace*{1cm}}l@{\hspace*{1cm}}l}
	(1)& FV(t') \subseteq FV(t) & \textsc{(free variables inclusion)} \\
	(2)& \typerule{t'}{\tau}{\theta}{\rho} &\textsc{(typing preservation)} \\
	(3)& t' \in \inlT & \textsc{(A-normal form preservation)}
\end{array}
\end{displaymath}
 \label{inllet-prop-p}
\end{lemma}

\begin{proof}
 By induction on the derivation of $\inllet{t}{t'}$.
 \label{TODO-inllet-prop-p}
 Back to \ref{inllet-prop-l}.
 \end{proof}	
	
%  \begin{lemma}
% 	If $\inllet{t}{t'}$, then $FV(t') \subseteq FV(t)$.
% 	\label{inllet-fv-p}
%  \end{lemma}
%
%\begin{proof}
% By induction on the derivation of $\inllet{t}{t'}$.  \\
% 	\noindent\textit{Case} \textsc{(Let$_0$)}\quad 
% 	$\tmlet{F}{s_1}{s_2} \hookdownarrow \tmsbst{s_2}{F}{s_1}$. 
% 	If $F \in FV(s_2)$, then 
% 	$$FV(t) = FV(s_1) \cup (FV(s_2) \backslash \{F\}) = FV(t') $$ 
% 	Otherwise  
% 	$$ FV(t') =  FV(s_2) \backslash \{F\} \subseteq (FV(s_1) \cup FV(s_2) \backslash \{F\}) =  FV(t)$$.
% Other cases are hold by straightforward induction. Back to \ref{inllet-fv-l}.
%\end{proof}	
%	
%	\begin{lemma}[($\inlletarr$, typing preservation)] For any well-typed term $t$
% such that\\ $\typerule{t}{\tau}{\theta}{\rho}$, if $\inllet{t}{t'}$
% then $\typerule{t'}{\tau}{\theta}{\rho}$.
% \label{inllet-ty-p}
%\end{lemma}
%
%\begin{proof} Observe that for the base case:
%$$\tmlet{F}{s_1}{s_2} \hookdownarrow \tmsbst{s_2}{F}{s_1} \quad \textsc{(Let$_0$)},$$ the restriction typing for term $s_1$, $\typerule{s_1}{\tau_{s_1}}{\bth}{\brh}$ guarantees that the only source of side effects or potential non-termination of $t$ is $s_2$, and the result follows immediately by LEMMA \label{TODO:typ-subst-lemma}. Other cases follow by straightforward induction on 
%the derivation of $\inllet{t}{t'}$. Back to \ref{inllet-ty-l}.
%\end{proof}
%	
%\begin{lemma}[($\inlletarr$, \textit{A-form})] 
% If $\inllet{t}{t'}$, then $t' \in \inlT$.
% \label{inllet-aform-p}
%\end{lemma}
%
%\begin{proof} Back to \ref{inllet-aform-l}.
%\end{proof}	
	
	
\begin{lemma}([Normal Forms For $\inlletarr^\star$]) = Output language S
 \label{inllet-nforms-p}.
\end{lemma}
\begin{proof}
Back to \ref{inllet-nforms-l}.
\end{proof}


\begin{theorem}([Termination of $\inlletarr^\star$])
\label{inllet-term-p}.
\end{theorem}
\begin{proof}
  Back to \ref{inllet-term-l}.
\end{proof}	
	
	
	
	\subsection{Inlining Second-Order Applications}



	\subsection{Semantic Equivalence Relation}
	\begin{lemma}
		$\forall t,t'.~ \eqv{t} \Leftrightarrow 
			(\forall v, v'.~\eqv{v}~\Rightarrow
			\tmapp{t}{v} \thicksim \tmapp{t'}{v'})$
	\label{equiv-def-p}
	\end{lemma}

	\begin{proof}
		Let $v, v'$ be two arbitrary values satisfying $\eqv{v}$.
		From the formulation of lemma, it follows that
		$$	\typerule{t}{\tyarr[1][2][\theta_0][\rho_0]}{\theta}{\rho}{} ~~ 
				\typerule{t'}{\tyarr[1][2][\theta_0][\rho_0]}{\theta}{\rho}{}.$$
		where $\ty[2]$ can be either some first-order type $\tyord{0}[2]$ or
		some arrow type $\tyarr[21][22][\theta_1][\rho_1].$ 
		Then, we prove the lemma by induction on the structure of type $\ty[t]$.
	\begin{itemize}		
	
		\item[$(\Rightarrow)$] Assume $\eqv{t}$. Thus 		
		
		\begin{itemize}
		
		\item[$(\alpha)$] if $\ty[\tmapp{t'}{v'}] = \tyord{0}[2]$, then
		by definition of $\eqv{t}$, for any initial state $\mu_0$ either 
		$\tmapp{t'}{v'}$ and $\tmapp{t'}{v'}$ diverge both, 
		or there is a pair of values $v_1, v'_1$ and a state $\mu_1$ such that 
		$$\evalstar{\tmapp{t}{v}}{0}{{v_1}}{1} 
			\bwedge \evalstar{\tmapp{t'}{v'}}{0}{{v'_1}}{1} \bwedge~\eqv{v_1}.$$
		
		By the preservation of typing, $v_1$ and $v'_1$ are of some base 
		type $\tyord{0}[2]$, and from the fact that they are values, we deduce
		that $v_1 = v'_1$. Therefore,  $\tmapp{t}{v} \thicksim \tmapp{t'}{v'})$.
			
		\item[$(\beta)$] Otherwise, 
		$\ty[\tmapp{t}{v}] = \tyarr[21][22][\theta_1][\rho_1]$.
		Let $v_0, v'_0$ be two arbitrary values satisfying $\eqv{v_0}$ and 
		$\ty[v_0] = \ty[21]$. We must show that for any initial state $\mu_0$,
		either both 
		$\tmapp{\tmapp{t}{v}}{v_0}$ and $\tmapp{\tmapp{t'}{v'}}{v'_0}$ diverge
		or there is a pair of values $v_1, v'_1$ and a state $\mu_1$ such that 
		$$\evalstar{\tmapp{\tmapp{t}{v}}{v_0}}{0}{{v_1}}{1} \bwedge
		\evalstar{\tmapp{\tmapp{t'}{v'}}{v'_0}}{0}{{v'_1}}{1} \bwedge~\eqv{v_1}.$$
		
		If both $t$ and $t'$ both diverge,
		then so do 
		$\tmapp{\tmapp{t'}{v'}}{v'_0}$ and $\tmapp{\tmapp{t'}{v'}}{v'_0}$. 
		Otherwise, there are
		some values $v_2$, $v'_2$ and intermediate state $\mu_2$ such that
		$$\evalstar{\tmapp{\tmapp{t}{v}}{v_0}}{0}{\tmapp{v_2}{v_0}}{2} \bwedge
		\evalstar{\tmapp{\tmapp{t'}{v'}}{v'_0}}{0}{\tmapp{v'_2}{v'_0}}{2} 
		\bwedge~\eqv{v_2}$$	
		Then by induction hypothesis on $\ty[\tmapp{t}{v}]=\ty[v_2]$, we 
		get	$$ \tmapp{v_2}{v_0} \thicksim \tmapp{v'_2}{v'_0}.$$ Therefore, 
		$\tmapp{v_2}{v_0}$ diverges if and only if $\tmapp{v_2'}{v_0'}$ diverges 
		too, so again, $\tmapp{\tmapp{t'}{v'}}{v'_0}$ diverges if and only if 
		$\tmapp{\tmapp{t'}{v'}}{v'_0}$ diverges. Otherwise there exist some values $v_1, v'_1$ and state $\mu_1$ such that:
		$$\evalstar{\tmapp{v_2}{v_0}}{2}{{v_1}}{1} \bwedge
		\evalstar{\tmapp{v_2}{v'_0}}{2}{{v'_1}}{1} 
		\bwedge~\eqv{v_1},$$ which allows us to deduce that $\tmapp{t}{v} \thicksim \tmapp{t'}{v'}.$ 
		\end{itemize}
		
	
		\item[$(\Leftarrow)$] Assume $(\forall v_0, v'.~\eqv{v}~\Rightarrow
		\tmapp{t}{v} \thicksim \tmapp{t'}{v'})$. 
		Then we have $\eqv{t}$ simply by definition of $\thicksim$. Back to \ref{equiv-def-l}.
				
		\end{itemize}	
	\end{proof}

	\begin{lemma}
		For any pair of states $\mu_0$, $\mu_1$ such that
		$\evalstar{{t_0}}{0}{{t_1}}{1}$ and $\evalstar{{t'_0}}{0}{{t'_1}}{1}$
		if $\eqv{t_0}$ then $\eqv{t_1}$.
		\label{equiv-red2-p}
	\end{lemma}
	
	\begin{proof}
		By induction on the structure of type of $t_0$. 
		\begin{itemize}
		\item[$(\alpha)$] if $\ty = \ty^0$ for some base type $\ty^0$, then
		
			\begin{itemize}
			\item[$(\alpha_1)$] if $t_0$ diverges, then by definition of 
			$\eqv{t_0}$, $t'_0$ diverges too. $\rightarrow^\star$ being 
			deterministic, both $t_1$ and $t'_1$ diverge. Symmetrically, if 
			$t'_0$ diverges, $t_1$ and $t'_1$ diverge both as well. 
						
			\item[$(\alpha_2)$] Otherwise, neither $t_0$ nor $t_{0}'$ diverges. 
			 By hypothesis we have that $\eqv{t_0}$. Thus,  
					$$ \exists \mu_2 \exists v: 
					\evalstar{{t_0}}{0}{v}{2} \bwedge \evalstar{{t'_0}}{0}{v}{2}.$$
				from which it follows that 
					$$\evalstar{{t_1}}{1}{v}{2} \bwedge \evalstar{{t'_1}}{1}{v}{2}.$$
				As $\mu_1$ depends uniquely on the arbitrarily chosen 
				$\mu_0$, we deduce again that $$\eqv{t_1}.$$					
			\end{itemize}			
			
		\item[$(\beta)$] Otherwise $\ty$ is an arrow type $\tyarr$ for some types 
		$\ty[1], \ty[2]$.	
		
		let $v_0, v'_0$ be two arbitrary values satisfying:
		$$~\eqv{v_0}~\bwedge \typerule{v_0}{\ty[1]}{\bth}{\brh}{}.$$
		By hypothesis, we have the reduction steps:
		$$\evalstar{{t_0}}{0}{{t_1}}{1}\quad\evalstar{{t'_0}}{0}{{t'_1}}{1}$$						which induce respectively: 
		$$\evalstar{\tmapp{t_0}{v_0}}{0}{\tmapp{t_1}{v_0}}{1} \quad
		\evalstar{\tmapp{t'_0}{v'_0}}{0}{\tmapp{t'_1}{v'_0}}{1}$$	
		From the definition of $\eqv{t_0}$, it follows that
		${\tmapp{t_0}{v_0}} \thicksim {\tmapp{t'_0}{v'_0}}.$
		
		Therefore, the induction hypothesis on $\ty[2]$ yields 
		${\tmapp{t_1}{v_0}}\thicksim{\tmapp{t'_1}{v'_0}}.$
		
		Therefore by lemma \ref{equiv-def-l},
		$\eqv{t_1}.$ Back to \ref{equiv-red2-l}.	
		\end{itemize}	
	\end{proof}		
%	\begin{lemma}
%	\label{equiv-subst-p}
%	\end{lemma}

%	\begin{proof}
%	Back to \ref{equiv-subst-l}.
%	\end{proof}
	
%%%%%%%%%%%%%%%%%%%%%%%%%%%%%%%%%%%%%%%%%%%%%%%%%%%%%%%%%%%%%%%%%%%%%%%%%%%%%%
\newpage
%\addcontentsline{toc}{chapter}{Bibliography}
\bibliographystyle{plain}
\bibliography{abbrevs,demons,demons2,demons3,team,crossrefs,./biblio}

\end{document}

(*
Local Variables:
compile-command: "rubber -d main"
End:
*)



%
%	It is important to see Together with the idea of higher-order program inlining, we introduce the idea of \textit{partial specification} of higher-order definitions. 
%	A specification of program is partial, if it does not suffice to prove a desired correctness property about that program. 
%	Typically, the following specification of \texttt{Array.iter} is partial, in that sense t
%%
%	Let us illustrate this on the example of \texttt{sum\_iter program}. 
%	First off all, the specification of \texttt{Array.iter} function is partial, in that sense that it provides no specification for its formal parameter $f$:
%	\begin{small}
%	\begin{whycode}  
%   let array_iter (f: int -> unit) (a: int array) $\textcolor{OliveGreen}{\text{(inv: int -> prop)}}$ 
%     $\textcolor{OliveGreen}{\text{requires \{ \text{inv } 0  \}}}$
%     $\textcolor{OliveGreen}{\text{ensures~~}\{ \text{ inv a.length }\}}$ =  
%     let rec loop i	
%        $\textcolor{OliveGreen}{ \text{requires \{~inv i } \bwedge \text{0 <= i <= a.length}~\}}$
%        $\textcolor{OliveGreen}{\text{ensures~~}\{ \text{ inv a.length }\}}$ =  
%        = if i < a.length then (f a[i] ; loop (i + 1) 
%     in loop 0
% 	\end{whycode}
% \end{small}
%	
%	
%		\begin{small}
%	\begin{whycode}  
%  let sum_iter (a: array int) =	
%     let s = ref 0 in
%     let array_iter (f: int -> unit) (a: int array) $\textcolor{OliveGreen}{\text{(inv: int -> prop)}}$ 
%       $\textcolor{OliveGreen}{\text{requires \{ \text{inv } 0  \}}}$
%       $\textcolor{OliveGreen}{\text{ensures~~}\{ \text{ inv a.length }\}}$ =  
%       let rec loop i	
%         $\textcolor{OliveGreen}{ \text{requires \{~inv i } \bwedge \text{0 <= i <= a.length}~\}}$
%         $\textcolor{OliveGreen}{\text{ensures~~}\{ \text{ inv a.length }\}}$ =  
%         = if i < a.length then (f a[i] ; loop (i + 1) 
%       in loop 0
%     in Array.iter (fun x -> s := !s + x) a;
% 	\end{whycode}
% \end{small}
%	
	

%	The verification condition \texttt{vc1} ensures that variable \texttt{i} used as an array index is within the bounds of \texttt{a}. \texttt{vc2} checks that  the invariant holds before entering into the loop. 
%
%	The verification condition \texttt{vc3} checks that for each iteration of loop, \textit{if the invariant holds for the preceding iteration step $i$}, for some $i<n$, then it still holds for the step $i+1$.   
%	
	


