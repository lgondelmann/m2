\documentclass[11pt]{article}

\usepackage[english]{babel} %
% \usepackage[T1]{fontenc} %
\usepackage[latin1]{inputenc} %
\usepackage[a4paper,colorlinks=true,urlcolor=blue,pdfstartview=FitH]{hyperref}
% \usepackage[applemac]{inputenc} %
% \usepackage{a4wide} %

% \setlength{\parskip}{0.3\baselineskip}

\begin{document}

\title{Inlining of Higher-Order Programs}

\author{L�on Gondelman, Jean-Christophe Filli�tre,  LRI, �quipe Toccata}

\date{\today}

\maketitle

\pagestyle{empty} %
\thispagestyle{empty}

%% Attention: pas plus d'un recto-verso!
% Ne conservez pas les questions


\subsection*{The general context}

The goal of this work is to present the inlining of higher-order programs \textit{inlining} in the context of deductive program verification.

	Deductive software verification is the process where the correctness of
programs is proved, using computational logic and an axiomatic formal
semantics approach$^{\cite{filliatre11sttt}}$. 

	Until recently, this approach was mainly applied to the verification of 
imperative programs written in languages such as C and Java.
In the last years, however, deductive verification of functional programs became an active research domain. 
	The work of Regis-Gianas and Pottier$^{\cite{regis-gianas-pottier-08}}$ defines a Hoare logic for a call-by-value programming language equipped with recursive, higher-order functions, algebraic data types, and a polymorphic type system in the style of Hindley and Milner. 
 	The work of Kanig$^{~\cite{kanig2010specification}}$  is dedicated to the specification and proof of \textit{stateful} Higher-Order Programs, thus generalizing deductive verification approach for functional programs that can be written in languages with imperative features such as OCaml.
	

\subsection*{The research problem}
	The main difficulty of the higher-order program deductive verification is to verify these programs automatically, using tools like automated theorems provers and SMT-solvers: in most cases, the solving power these tools is limited to first-order.
	However, many of higher-order definitions of every-day programming practice are functions such as \textit{iterators} \textit{mappings}, and \textit{folders}, used in most of the cases, to apply some first-order function to the elements of some primitive data such as arrays, lists, etc. 	
 The main question is thus to know if in practice we can do the verification of programs using these definitions in a similar way to the verification of a programs using simple imperative loop.
  
 
 
\subsection*{Your contribution}

We have presented a pragmatic approach to the verification of stateful
second-order programs, 
%which consists in defining the inlining trasformation of second-order definitions. 
%First we inline higher-order functions and then we use an existing tool for the
%verification of first-order programs (namely Why3). Obviously, this
%approach does not apply to all ML programs. Yet we think it is a
%valuable approach in many cases, as higher-order functions are often
%simply used to encode loops of various kinds.
% First, is to restrain higher-order stateful programs to the subset of second-order programs, for which it becomes possible to transform each second-order program into semantically equivalent first-order program, by simply \textit{inlining} all higher-order definitions inside. 
Concretely, we define the inlining of higher-order functions and then we use an existing tool for the
verification of first-order programs (namely Why3). Obviously, this
approach does not apply to all ML programs. Yet we think it is a
valuable approach in many cases, as higher-order functions are often
simply used to encode loops of various kinds.

We have given the abstract syntax, the semantics, and a type system
with effects for a language Mini-ML with global references. We have
formally defined an inlining procedure on some second-order fragment
of this language. While working out some preliminary examples, 
we have identified the need for ghost code in
the writing of specifications for higher-order functions such as map,
iter, or fold. Thus we have extended Mini-ML with ghost code, with a
type system that checks for non-interference between ghost code and
regular code, and ghost code erasure operation.

\subsection*{Arguments supporting its validity}

To guarantee the soundness of the inlining procedure, we stated and proved its total correctness, using a well-know technique of \textit{logical relations}.


\subsection*{Summary and future work}

By lack of time, we haven't implemented a prototype. Hence our
experimental evaluation is currently limited to a few examples where
inlining was manually performed, the resulting programs being written in
Why3 syntax. The results being definitely promising,
in the future, we intend to implement a prototype and work out other examples.
Also we would like to investigate the theoretical scope of this work in the case of second-order programs with polymorphic types. 
\begin{footnotesize}
\bibliographystyle{plain}

\bibliography{abbrevs,demons,demons2,demons3,team,crossrefs,./biblio}
\end{footnotesize}
\end{document}

(*
Local Variables:
compile-command: "rubber -d main"
End:
*)

