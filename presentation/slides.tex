\documentclass{beamer}

\usepackage[utf8]{inputenc}
\usepackage[french]{babel}

\begin{document}

\title{Deductive Verification of Stateful Second-Order Programs}
\author{Léon Gondelman}
\date{M2 internship \\ \today}

\begin{frame}
  \maketitle
\end{frame}

% 1. contexte, problème, exemple introductif (array_iter)
% sum_loop et sum_iter en parallèle


\begin{frame}\frametitle{}
 
\end{frame}

\begin{frame}\frametitle{}
  \begin{columns}
  \column{0.5\textwidth}

		\begin{whycode}  
 let sum_loop (a: array int) =		
  let s = ref 0 in
  let i = ref 0 in
  while !i < a.length do
    s := !s + a[!i];
    i := !i + 1
  done; !s 
 (*Array sum with while loop*)
		\end{whycode}

  \column{0.5\textwidth}


		\begin{whycode} 
  let sum_iter (a: array int) =		 
    let s = ref 0 in
    Array.iter (fun x -> s := !s + x) a;

    (*Array sum with iterator*)
		\end{whycode}	

  et la colonne deux
  \end{columns}
\end{frame}

\begin{frame}\frametitle{}
\end{frame}

% solution technique = inlining, dérouler sur l'exemple de sum_iter

% 2. quelles difficultés
%    ce qui nous amène à Mini-ML = ML + références globales
%    syntaxe, sémantique, système de types avec effets, preuve de type soundness
%    identification d'un fragment de Mini-ML + définition inlining + preuve

\begin{frame}\frametitle{}
\end{frame}

\begin{frame}\frametitle{}
\end{frame}

\begin{frame}\frametitle{}
\end{frame}

% 3. maintenant list_iter, servant à introduire l'idée de code ghost

%  d'abord le code de list_iter + le code client sum_list

%  puis la spec avec un argument supplémentaire

% => introduit la notion de code ghost

% un pgm avec du code ghost :
%  - pour le prouver, je prends *tout* le code, sans distinction
%  - pour l'exécuter, j'oublie le code ghost
% tout est dans le typage, qui garantit la non-interférence

% 4. formalisation de Ghost-ML
%    syntaxe, sémantique, typage avec effets (montrer le jugement),
%      bien expliquer qu'il y a des choix à faire (mais aussi des*)
%      obligations), l'opération erasure et la preuve

\end{document}

% Local Variables:
% compile-command: "rubber -d slides"
% ispell-local-dictionary: "francais"
% End:
